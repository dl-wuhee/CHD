%\section{气体基本概念}
\subsection{状态方程}
\begin{frame}{完全气体状态方程}
  \only<1>{
    \begin{description}
      \item[\color{blue}理想气体] 假想的没有粘滞性的流体,即$\mu=0$
      \item[\color{blue}完全气体] 假想气体,仅考虑分子热运动,忽略分子间内聚力与分子体积
    \end{description}
  }
  完全气体状态方程——形式1:
  \begin{equation*}
    pV = nR_{u}T
  \end{equation*}
  \only<1>{
    \begin{itemize}
      \item $p$:气体绝对压强
      \item $V$:气体体积
      \item $n$:气体摩尔数
      \item $R_{u}=8.314\mathrm{J/(mol\cdot K)}$:通用气体常数
      \item $T$:气体绝对温度
    \end{itemize}
  }
  \uncover<2>{
    \vspace*{-1em}
    \begin{equation*}
      p
      =
      \frac{nM}{V}
      \frac{R_{u}}{M}T
      =
      \frac{m}{V}
      RT
      =
      \rho RT
    \end{equation*}
    完全气体状态方程——形式2:
    \begin{equation*}
      \frac{p}{\rho}
      =
      RT
    \end{equation*}
    \begin{equation*}
      p\bar{v}=RT
    \end{equation*}
    \vspace*{-0.5em}
    \begin{itemize}
      \item $M$:气体分子摩尔质量
      \item $m=nM$:气体质量
      \item $\displaystyle R=\frac{R_{u}}{M}$:气体常数,单位为$\mathrm{J/(kg\cdot K)}$
      \item $\bar{v}$:比体积$\bar{v}=1/\rho$
    \end{itemize}
  }
\end{frame}


\subsection{热动力性质}
\begin{frame}{比热、焓、熵}
  \begin{itemize}[<+-|alert@+>]
    \item 单位质量的物质温度每升高一度所需的热量为{\color{blue}比热\color{red}$c$}。
      %\item 气体比热取决于伴随温度变化的过程
    \item 温度变化过程中体积保持不变的比热为{\color{blue}定容比热
      \color{red}$\displaystyle c_{v} = \frac{\mathrm{d}e}{\mathrm{d}T}$},
      $e=c_{v}T$
    \item 温度变化过程中压强保持不变的比热为{\color{blue}定压比热
      \color{red}$\displaystyle c_{p}=\frac{\mathrm{d}h}{\mathrm{d}T} $},
      $h=c_{p}T$
    \item {\color{blue}焓\color{red}$h= e + p\bar{v} = e + RT$},$\displaystyle
      \frac{\mathrm{d} h}{\mathrm{d} T}
      =
      \frac{\mathrm{d} e + \mathrm{d}(RT)}{\mathrm{d} T}
      =
      \frac{\mathrm{d} e}{\mathrm{d} T}
      +
      R
      $
      \begin{equation*}
        c_{p} - c_{v} = R
      \end{equation*}
    \item {\color{blue}比热比\color{red}$\displaystyle \gamma=\frac{c_{p}}{c_{v}}$}。对完全气体绝热过程,$\gamma$等于绝热指数$K$
    \item $\displaystyle c_{p} = \frac{\gamma}{\gamma-1}R$,$\displaystyle c_{v}=\frac{1}{\gamma-1}R$
    \item {\color{blue}熵\color{red}$\displaystyle s = c_{v}\ln{\frac{p}{\rho^{\gamma}}}$}
  \end{itemize}
\end{frame}

\subsection{过程方程}
\begin{frame}{等熵过程}
  \begin{itemize}
    \item 等熵过程方程
      \begin{equation*}
        \frac{p}{\rho^{\gamma}}
        =
        p\bar{v}^{\gamma}
        =
        \mathrm{C}
      \end{equation*}
    \item 初、终态参数间关系
      \begin{equation*}
        \begin{aligned}
          \frac{p_{2}}{p_{1}}
           &=
           \left(\frac{\rho_{2}}{\rho_{1}}\right)^{\gamma}
           \\
           \frac{p_{2}}{p_{1}}
           &=
           \left(\frac{\bar{v}_{1}}{\bar{v}_{2}}\right)^{\gamma}
           \\
           \frac{T_{2}}{T_{1}}
           &=
           \left(\frac{\rho_{2}}{\rho_{1}}\right)^{\gamma-1}
           \\
           \frac{T_{2}}{T_{1}}
           &=
           \left(\frac{\bar{v}_{1}}{\bar{v}_{2}}\right)^{\gamma-1}
           \\
           \frac{T_{2}}{T_{1}}
           &=
           \left(\frac{p_{2}}{p_{1}}\right)^{\frac{\gamma-1}{\gamma}}
        \end{aligned}
      \end{equation*}
  \end{itemize}
\end{frame}

\subsection{气体压缩性}
\begin{frame}{体积弹性模量}
  \vspace*{-0.5em}
  \begin{definition}[体积弹性模量]
    \begin{equation*}
      E_{v}
      =
      -\frac{\mathrm{d}p}{\mathrm{d}V/V}
      \only<5>{
        =
        \frac{\mathrm{d}p}{\mathrm{d}\rho/\rho}
      }
    \end{equation*}
    \vspace*{-1em}
    \begin{itemize}
      \item $\mathrm{d}p$:压强变化量
      \item $\mathrm{d}V$:体积变化量
      \item $V$:初始体积
      \item 上式中负号是因为当$\mathrm{d}p$为正时,$\mathrm{d}V$为负
      \item $E_{v}$越大,气体越难被压缩;$E_{v}$越小,气体越易被压缩;
    \end{itemize}
  \end{definition}
  \vspace*{-1.5em}
  \begin{equation*}
    \only<2->{
      m = \rho V
    }
  \end{equation*}
  \begin{equation*}
    \only<3->{
      \mathrm{d}m
      =
      \rho\mathrm{d}V
      +
      V\mathrm{d}\rho
      =
      0
    }
  \end{equation*}
  \begin{equation*}
    \only<4->{
      \frac{\mathrm{d}\rho}{\rho}
      =
      -\frac{\mathrm{d}V}{V}
    }
  \end{equation*}
\end{frame}

\begin{frame}{体积弹性模量——续}
  对完全气体等温过程:
  \begin{equation*}
    \frac{\mathrm{d}p}{\mathrm{d}\rho}
    =
    RT
  \end{equation*}
  \begin{equation*}
    E_{v}
    =
    \rho\frac{\mathrm{d}p}{\mathrm{d}\rho}
    =
    \rho RT
    =
    p
  \end{equation*}
  对绝热过程:
  \begin{equation*}
    \frac{p}{\rho^{\gamma}} = \mathrm{C}
  \end{equation*}
  \begin{equation*}
    \mathrm{d}p
    =
    \mathrm{C}\gamma \rho^{\gamma-1}\mathrm{d}\rho
    =
    \frac{p}{\rho^{\gamma}}\gamma \rho^{\gamma-1}\mathrm{d}\rho
  \end{equation*}
  \begin{equation*}
    \frac{\mathrm{d}p}{\mathrm{d}\rho}
    =
    \gamma\frac{p}{\rho}
    =
    \gamma RT
  \end{equation*}
  \begin{equation*}
    E_{v}
    =
    \rho\frac{\mathrm{d}p}{\mathrm{d}\rho}
    =
    \gamma\rho RT
    =\gamma p
  \end{equation*}

\end{frame}
