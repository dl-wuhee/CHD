\chapter{有限差分法}
上一章推导的控制方程的解析解,如果可以得出,将可以给出所有因变量在求解域上连续函
数。然而,除了个别极其简单的流动情况,这类方程的解析解往往很难获得。因此,我们只
能退而求其次,通过将连续的空间和时间分割成不连续的散点(分别称为网格点和时间节点
),并将控制方程在这些散点上离散,得出与原方程近似但可求解的代数方程组,最终得出
原方程在这些散点上的近似解。

在实际的数值模拟过程中,网格点或时间节点的划分是不均匀的。为了后续推导方便,本章
假定网格点和时间节点的间距都是均匀分布的。此外,本章只关注结构网格,对非结构网格
可参见其他参考书。

\section{泰勒展开}
考虑一个$x$的连续函数$f(x)$在$x$处的所有阶导数都存在。那么,$f$在点$x+\Delta x$
处的值可以通过在点$x$处的泰勒级数展开来估算。
\begin{equation}
  f(x+\Delta x)
  =
  f(x)
  +
  \frac{\partial f}{\partial x}\Delta x
  +
  \frac{\partial^{2} f}{\partial x^{2}}\frac{\Delta x^{2}}{2}
  +
  \cdots
  +
  \frac{\partial^{n} f}{\partial x^{n}}\frac{\Delta x^{n}}{n!}
  +
  \cdots
\end{equation}

假定$f(x)=\sin{2\pi x}$,且已知$x=0.2$处,函数值$f=0.9511$。我们希望估算
$x+\Delta x=0.22$处的函数值。根据方程表达式,该点的精确值为0.9823。首先,我们利
用泰勒级数展开的第一项来估算,有
\begin{equation}
  \begin{aligned}
    f(x+\Delta x) \approx f(x)  \\
    f(0.22) \approx f(0.2) = 0.9511
  \end{aligned} 
\end{equation}
估算值的相对误差为$(0.9823-0.9511)/0.9823=3.176\%$。接着,我们利用展开式的前两项
来估算,有
\begin{equation}
  \begin{aligned}
    f(x+\Delta x) &\approx f(x) + \frac{\partial f}{\partial x}\Delta x
  \\
    f(0.22) &\approx f(0.2) + 2\pi\cos{[2\pi(0.2)]}(0.02)
    \\
            &\approx 0.9511 + 0.388 = 0.9899
\end{aligned}
\end{equation}
估算值的相对误差为$(0.9899-0.9823)/0.9823=0.775\%$,相比于第一次的估算值更加接近
精确值。最后,为了获得更加精确的估算值,我们可以利用展开式的前三项,有
\begin{equation}
  \begin{aligned}
    f(x+\Delta x) &\approx f(x) + \frac{\partial f}{\partial x}\Delta x
    +
    \frac{\partial^{2} f}{\partial x^{2}}\frac{(\Delta x)^{2}}{2}
  \\
    f(0.22) &\approx f(0.2) + 2\pi\cos{[2\pi(0.2)]}(0.02) -
    4\pi^{2}\sin{[2\pi(0.2)]}\frac{0.02^{2}}{2}
    \\
            &\approx 0.9511 + 0.0388 - 0.0075
    \\
            &\approx 0.9824
\end{aligned}
\end{equation}
估算值的相对误差为$(0.9824-0.9823)/0.9823=0.01\%$,即仅用泰勒展开式的前三项就可
以得到一个非常接近精确值的估算值。

\section{离散基础知识}
图中给出了一个结构网格的例子。图中,网格点,即两条网格线的交点,可以用$(i,j)$进
行标示,其中,$i$是沿$x$方向的索引,$j$是沿$y$方向的索引。假定图中$P$的索引为
$(i,j)$,那么,紧邻$P$右边的点的索引为$(i+1,j)$,紧邻$P$左边的点的索引为
$(i-1,j)$,紧邻$P$上边的点的索引为$(i,j+1)$,紧邻$P$下边的点的索引为$(i,j-1)$。

\subsection{一阶偏导数的差分表达式}
点$(i,j)$上$x$方向速度用$u_{i,j}$表示,点$(i+1,j)$上$x$方向速度$u_{i+1,j}$可以在
点$(i,j)$上通过泰勒级数展开为:
\begin{equation}
  u_{i+1,j}
  =
  u_{i,j} + 
  \left(
    \frac{\partial u}{\partial x}
  \right)_{i,j}
  \Delta x
  +
  \left(
    \frac{\partial^{2} u}{\partial x^{2}}
  \right)_{i,j}
  \frac{(\Delta x)^{2}}{2}
  +
  \left(
    \frac{\partial^{3} u}{\partial x^{3}}
  \right)_{i,j}
  \frac{(\Delta x)^{3}}{3!}
  +
  \cdots
  \label{EqBD_Tsp_F}
\end{equation}
用上式求解$(\partial u/\partial x)_{i,j}$,可得
\begin{equation}
  \left(
    \frac{\partial u}{\partial x}
  \right)_{i,j}
  =
  \underbrace{
  \frac{u_{i+1,j}-u_{i,j}}{\Delta x}
}_{\mbox{差分表达式}}
  -
  \underbrace{
  \left(
    \frac{\partial^{2} u}{\partial x^{2}}
  \right)_{i,j}
  \frac{\Delta x}{2}
  -
  \left(
    \frac{\partial^{3} u}{\partial x^{3}}
  \right)_{i,j}
  \frac{(\Delta x)^{2}}{6}
  +
  \cdots
}_{\mbox{截断误差}}
\label{EqBD_Tes}
\end{equation}
上式中,如果我们用等号右边的差商$(u_{i+1,j}-u_{i,j})/\Delta x$来近似
$(\partial u/\partial x)_{i,j}$,则该项为偏导数的差分表达式,等号右边的其他项为截断误
差,可以省略。即,
\begin{equation}
  \left(
    \frac{\partial u}{\partial x}
  \right)_{i,j}
  \approx
  \frac{u_{i+1,j}-u_{i,j}}{\Delta x}
\end{equation}

式\eqref{EqBD_Tes}也可写成
\begin{equation}
  \left(
    \frac{\partial u}{\partial x}
  \right)_{i,j}
  =
  \frac{u_{i+1,j}-u_{i,j}}{\Delta x}
  +
  O(\Delta x)
  \label{EqBD_1P1AF}
\end{equation}
从上式可看出,被省略的截断误差中$\Delta x$的幂次最小的那一项决定了上式的精度
。截断误差中$\Delta x$的幂次即为式\eqref{EqBD_1P1AF}的精度。上式中截断误差中
$\Delta x$的最小幂次为1。因而,式\eqref{EqBD_1P1AF}为一阶精度,此外,式\eqref{EqBD_1P1AF}
中差分表达式使用了网格点$(i,j)$及其右侧相邻网格点$(i+1,j)$的信息,没有使用
$(i,j)$左侧网格点信息。因此,式\eqref{EqBD_1P1AF}也叫做向前差分。综上,式
\eqref{EqBD_1P1AF}为一阶偏导数$(\partial u/\partial x)_{i,j}$的一阶向前差分。

同理,$u_{i-1,j}$在$(i,j)$上进行泰勒级数展开为
\begin{equation}
  u_{i-1,j}
  =
  u_{i,j} - 
  \left(
    \frac{\partial u}{\partial x}
  \right)_{i,j}
  \Delta x
  +
  \left(
    \frac{\partial^{2} u}{\partial x^{2}}
  \right)_{i,j}
  \frac{(\Delta x)^{2}}{2}
  -
  \left(
    \frac{\partial^{3} u}{\partial x^{3}}
  \right)_{i,j}
  \frac{(\Delta x)^{3}}{3!}
  +
  \cdots
  \label{EqBD_Tsp_B}
\end{equation}
利用上式求解$(\partial u/\partial x)_{i,j}$,得
\begin{equation}
  \left(
    \frac{\partial u}{\partial x}
  \right)_{i,j}
  =
  \frac{u_{i,j}-u_{i-1,j}}{\Delta x}
  +
  O(\Delta x)
  \label{EqBD_1P1AB}
\end{equation}
上式中没有使用网格点$(i,j)$右侧网格点信息。因而,式\eqref{EqBD_1P1AB}为向后差分
。此外,上式中,截断误差中$\Delta x$的最低幂次为1。因此,式\eqref{EqBD_1P1AB}为
一阶偏导数$(\partial u/\partial x)_{i,j}$的一阶向后差分。

在计算水动力学、计算流体力学的实际应用中,一阶精度往往是不够的。为了构造一个二阶
精度的差分,将式\eqref{EqBD_Tsp_B}从式\eqref{EqBD_Tsp_F}中减去:
\begin{equation}
  u_{i+1,j} - u_{i-1,j}
  =
  2
  \left(
    \frac{\partial u}{\partial x}
  \right)_{i,j}
  \Delta x
  +
  2
  \left(
    \frac{\partial^3 u}{\partial x^3}
  \right)_{i,j}
  \frac{(\Delta x)^3}{3!}
  +
  \cdots
\end{equation}
上式可写成
\begin{equation}
  \left(
    \frac{\partial u}{\partial x}
  \right)_{i,j}
  =
  \frac{u_{i+1,j} - u_{i-1,j}}{2\Delta x}
  +
  O(\Delta x)^2
  \label{EqBD_1P2AC}
\end{equation}
式\eqref{EqBD_1P2AC}中使用了点$(i,j)$两侧节点$(i+1,j)$和$(i-1,j)$的信息。另外,
式\eqref{EqBD_1P2AC}中截断误差的最低幂次为2。因此,式\eqref{EqBD_1P2AC}是一阶偏
导数$(\partial u/\partial x)_{i,j}$的二阶中心差分。

同理,可以得到一阶偏导数$(\partial u/\partial y)_{i,j}$
的差分格式。总结如下:
\begin{equation}
  \left(
    \frac{\partial u}{\partial x}
  \right)_{i,j}
  =
  \left\{
    \begin{aligned}
      &\frac{u_{i+1,j}-u_{i,j}}{\Delta x} + O(\Delta x) & \mbox{一阶向前差分} \\
      &\frac{u_{i,j}-u_{i-1,j}}{\Delta x} + O(\Delta x) & \mbox{一阶向后差分} \\
      &\frac{u_{i+1,j}-u_{i-1,j}}{2\Delta x} + O(\Delta x)^{2} \quad& \mbox{二阶中心差分} \\
    \end{aligned}
  \right.
\end{equation}

\begin{equation}
  \left(
    \frac{\partial u}{\partial y}
  \right)_{i,j}
  =
  \left\{
    \begin{aligned}
      &\frac{u_{i,j+1}-u_{i,j}}{\Delta y} + O(\Delta y) & \mbox{一阶向前差分} \\
      &\frac{u_{i,j}-u_{i,j-1}}{\Delta y} + O(\Delta y) & \mbox{一阶向后差分} \\
      &\frac{u_{i,j+1}-u_{i,j-1}}{2\Delta y} + O(\Delta y)^{2} \quad& \mbox{二阶中心差分} \\
    \end{aligned}
  \right.
\end{equation}

\subsection{二阶偏导数的差分表达式}
将式\eqref{EqBD_Tsp_F}和式\eqref{EqBD_Tsp_B}相加,可得
\begin{equation}
  u_{i+1,j} + u_{i-1,j}
  =
  2u_{i,j}+
  \left(
    \frac{\partial^{2} u}{\partial x^{2}}
  \right)_{i,j}(\Delta x)^{2}
  +
  \left(
    \frac{\partial^{4} u}{\partial x^{4}}
  \right)_{i,j}\frac{(\Delta x)^{4}}{12}
  +
  \cdots
\end{equation}
利用上式求解出$(\partial^{2}u/\partial x^{2})_{i,j}$,
\begin{equation}
  \left(
    \frac{\partial^{2} u}{\partial x^{2}} 
  \right)_{i,j}
  =
  \frac{u_{i+1,j}-2u_{i,j}+u_{i-1,j}}{(\Delta x)^{2}}
  +
  O(\Delta x)^2
  \label{EqBD_2P2AC}
\end{equation}
式\eqref{EqBD_2P2AC}中,等号右边第一项是二阶偏导数$(\partial^{2}u/\partial x^{2})_{i,j}$的中心差分
截断误差中$\Delta x$的最小幂次为2,因此该式为二阶精度。同理,我们可以得到
$(\partial^{2}u/\partial y^{2})_{i,j}$的二阶精度中心差分,
\begin{equation}
  \left(
    \frac{\partial^{2} u}{\partial y^{2}} 
  \right)_{i,j}
  =
  \frac{u_{i,j+1}-2u_{i,j}+u_{i,j-1}}{(\Delta y)^{2}}
  +
  O(\Delta y)^2
  \label{EqBD_2Py2AC}
\end{equation}

对于二阶混合偏导,如$\partial^{2} u/\partial x\partial y$,可以采用与上面类似的
处理方式得到。首先,将式\eqref{EqBD_Tsp_F}对$y$求偏导,得
\begin{equation}
  \left(
    \frac{\partial u}{\partial y}
  \right)_{i+1,j}
  =
  \left(
    \frac{\partial u}{\partial y}
  \right)_{i,j}
  +
  \left(
    \frac{\partial^{2} u}{\partial x\partial y}
  \right)_{i,j}\Delta x
  +
  \left(
    \frac{\partial^{3} u}{\partial x^{2}\partial y}
  \right)_{i,j}\frac{(\Delta x)^{2}}{2} 
  +
  \left(
    \frac{\partial^{4} u}{\partial x^{3}\partial y}
  \right)_{i,j}\frac{(\Delta x)^{3}}{6} 
  +
  \cdots
  \label{EqBD_Tsp_F_py}
\end{equation}
然后,将式\eqref{EqBD_Tsp_B}对$y$求偏导,得
\begin{equation}
  \left(
    \frac{\partial u}{\partial y}
  \right)_{i-1,j}
  =
  \left(
    \frac{\partial u}{\partial y}
  \right)_{i,j}
  -
  \left(
    \frac{\partial^{2} u}{\partial x\partial y}
  \right)_{i,j}\Delta x
  +
  \left(
    \frac{\partial^{3} u}{\partial x^{2}\partial y}
  \right)_{i,j}\frac{(\Delta x)^{2}}{2} 
  -
  \left(
    \frac{\partial^{4} u}{\partial x^{3}\partial y}
  \right)_{i,j}\frac{(\Delta x)^{3}}{6} 
  +
  \cdots
  \label{EqBD_Tsp_B_py}
\end{equation}
将式\eqref{EqBD_Tsp_F_py}中减去式\eqref{EqBD_Tsp_B_py},得
\begin{equation}
  \left(
    \frac{\partial u}{\partial y}
  \right)_{i+1,j}
  -
  \left(
    \frac{\partial u}{\partial y}
  \right)_{i-1,j}
  =
  2
  \left(
    \frac{\partial^{2} u}{\partial x\partial y}
  \right)_{i,j}\Delta x
  +
  \left(
    \frac{\partial^{4} u}{\partial x^{3}\partial y}
  \right)_{i,j}\frac{(\Delta x)^{3}}{3} 
  +
  \cdots
\end{equation}
从上式中求解$(\partial^{2}u/\partial x\partial y)$,得
\begin{equation}
  \left(
    \frac{\partial^{2} u}{\partial x\partial y}
  \right)_{i,j}
  =
  \frac{(\partial u/\partial y)_{i+1,j}-(\partial u/\partial y)_{i-1,j}}{2\Delta x}
  -
  \left(
    \frac{\partial^{4} u}{\partial x^{3}\partial y}
  \right)_{i,j}\frac{(\Delta x)^{3}}{6} 
  +
  \cdots
  \label{EqBD_2PM2A}
\end{equation}
上式中等式右侧第一项中需要求解$(\partial u/\partial y)_{i+1,j}$和$(\partial
u/\partial y)_{i-1,j}$。利用上一小节所得到一阶偏导数的中心差分,得
\begin{equation}
  \left(
    \frac{\partial u}{\partial y}
  \right)_{i+1,j}
  =
  \frac{u_{i+1,j+1}-2u_{i+1,j}-u_{i+1,j-1}}{2\Delta y} + O(\Delta y)^{2}
\end{equation}
\begin{equation}
  \left(
    \frac{\partial u}{\partial y}
  \right)_{i-1,j}
  =
  \frac{u_{i-1,j+1}-2u_{i-1,j}-u_{i-1,j-1}}{2\Delta y} + O(\Delta y)^{2}
\end{equation}
将上两式代入式\eqref{EqBD_2PM2A},可得
\begin{equation}
  \left(
    \frac{\partial^{2} u}{\partial x\partial y}
  \right)_{i,j}
  =
  \frac{u_{i+1,j+1}-u_{i+1,j-1}-u_{i-1,j+1}+u_{i-1,j-1}}{4\Delta x\Delta y}
  +
  O[(\Delta x)^{2},(\Delta y)^{2}]
\end{equation}

本节所列出的差分格式只是所有差分格式的一小部分。同一个导数可以有许多不同的差分
格式,特别是更高精度的差分格式。高精度的差分格式通常需要引入更多网格点的信息。例
如,下式给出的$\partial^{2} u/\partial x^{2}$的四阶精度中心差分格式:
\begin{equation}
  \left(
    \frac{\partial^{2} u}{\partial x^{2}}
  \right)_{i,j}
  =
  \frac{-u_{i+2,j}+16u_{i+1,j}-30u_{i,j}+16u_{i-2,j}-u_{i-2,j}}{12(\Delta x)^{2}}
  +
  O(\Delta x)^{4}
\end{equation}

\subsection{待定系数法}
高阶格式的推导可用待定系数法得出。以上述的四阶精度中心差分格式推导为例。
我们需要利用$(i-2,j), (i-1,j), (i,j), (i+1,j),
(i+2,j)$这几个节点来构造出四阶精度的差分格式,即
\begin{equation}
  \left(
    \frac{\partial^{2} u}{\partial x^{2}}
  \right)_{i,j}
  =
  Au_{i+2,j}+Bu_{i+1,j}+Cu_{i,j}+Du_{i-2,j}+Eu_{i-2,j}
  +
  O(\Delta x)^{4}
\end{equation}
其中,系数$A, B, C, D, E$分别为待定系数。

首先,利用泰勒级数展开
将这$u_{i-2,j}, u_{i-1,j}, u_{i+1,j}, u_{i+2,j}$在$(i,j)$上展开,得:
\begin{equation}
  \begin{aligned}
    &\begin{aligned}
      u_{i-2,j}  =
      u_{i,j} &+
      \left(
        \frac{\partial u}{\partial x}
      \right)_{i,j}
      (-2\Delta x)
      +
      \left(
        \frac{\partial^{2} u}{\partial x^{2}}
      \right)_{i,j}
      \frac{(-2\Delta x)^{2}}{2}
      +
      \left(
        \frac{\partial^{3} u}{\partial x^{3}}
      \right)_{i,j}
      \frac{(-2\Delta x)^{3}}{3!}
      \\
              & 
              +
              \left(
                \frac{\partial^{4} u}{\partial x^{4}}
              \right)_{i,j}
              \frac{(-2\Delta x)^{4}}{4!}
              +
              \left(
                \frac{\partial^{5} u}{\partial x^{5}}
              \right)_{i,j}
              \frac{(-2\Delta x)^{5}}{5!}
              +
              \left(
                \frac{\partial^{6} u}{\partial x^{6}}
              \right)_{i,j}
              \frac{(-2\Delta x)^{6}}{6!}
              +
              \cdots
    \end{aligned}
    \\
    &\begin{aligned}
      u_{i-1,j}  =
      u_{i,j} &+
      \left(
        \frac{\partial u}{\partial x}
      \right)_{i,j}
      (-\Delta x)
      +
      \left(
        \frac{\partial^{2} u}{\partial x^{2}}
      \right)_{i,j}
      \frac{(-\Delta x)^{2}}{2}
      +
      \left(
        \frac{\partial^{3} u}{\partial x^{3}}
      \right)_{i,j}
      \frac{(-\Delta x)^{3}}{3!}
      \\
              & 
              +
              \left(
                \frac{\partial^{4} u}{\partial x^{4}}
              \right)_{i,j}
              \frac{(-\Delta x)^{4}}{4!}
              +
              \left(
                \frac{\partial^{5} u}{\partial x^{5}}
              \right)_{i,j}
              \frac{(-\Delta x)^{5}}{5!}
              +
              \left(
                \frac{\partial^{6} u}{\partial x^{6}}
              \right)_{i,j}
              \frac{(-\Delta x)^{6}}{6!}
              +
              \cdots
    \end{aligned}
    \\
    &u_{i,j} = u_{i, j}
    \\
    &\begin{aligned}
      u_{i+1,j}  =
      u_{i,j} &+
      \left(
        \frac{\partial u}{\partial x}
      \right)_{i,j}
      (\Delta x)
      +
      \left(
        \frac{\partial^{2} u}{\partial x^{2}}
      \right)_{i,j}
      \frac{(\Delta x)^{2}}{2}
      +
      \left(
        \frac{\partial^{3} u}{\partial x^{3}}
      \right)_{i,j}
      \frac{(\Delta x)^{3}}{3!}
      \\
              & 
              +
              \left(
                \frac{\partial^{4} u}{\partial x^{4}}
              \right)_{i,j}
              \frac{(\Delta x)^{4}}{4!}
              +
              \left(
                \frac{\partial^{5} u}{\partial x^{5}}
              \right)_{i,j}
              \frac{(\Delta x)^{5}}{5!}
              +
              \left(
                \frac{\partial^{6} u}{\partial x^{6}}
              \right)_{i,j}
              \frac{(\Delta x)^{6}}{6!}
              +
              \cdots
    \end{aligned}
    \\
    &\begin{aligned}
      u_{i+2,j}  =
      u_{i,j} &+
      \left(
        \frac{\partial u}{\partial x}
      \right)_{i,j}
      (2\Delta x)
      +
      \left(
        \frac{\partial^{2} u}{\partial x^{2}}
      \right)_{i,j}
      \frac{(2\Delta x)^{2}}{2}
      +
      \left(
        \frac{\partial^{3} u}{\partial x^{3}}
      \right)_{i,j}
      \frac{(2\Delta x)^{3}}{3!}
      \\
              & 
              +
              \left(
                \frac{\partial^{4} u}{\partial x^{4}}
              \right)_{i,j}
              \frac{(2\Delta x)^{4}}{4!}
              +
              \left(
                \frac{\partial^{5} u}{\partial x^{5}}
              \right)_{i,j}
              \frac{(2\Delta x)^{5}}{5!}
              +
              \left(
                \frac{\partial^{6} u}{\partial x^{6}}
              \right)_{i,j}
              \frac{(2\Delta x)^{6}}{6!}
              +
              \cdots
    \end{aligned}
  \end{aligned}
\end{equation}

对上式各式分别乘以系数$A,B,C,D,E$得:
\begin{equation}
\begin{aligned}
  &Au_{i-2,j}+Bu_{i-1,j}+Cu_{i,j}+Du_{i+1,j}+Eu_{i+2,j} 
  \\
  =
  &(A+B+C+D+E)u_{i,j}
  +
  \\
  &
  (-2A-B+D+2E)
  \left(
    \frac{\partial u}{\partial x}
  \right)_{i,j}
  \Delta x
  +
  \\
  &
  \left(2A+\frac{B}{2}+\frac{D}{2}+2E\right)
  \left(
    \frac{\partial^{2} u}{\partial x^{2}}
  \right)_{i,j}
  (\Delta x)^{2}
  +
  \\
  &
  \left(-\frac{4}{3}A-\frac{B}{6}+\frac{D}{6}+\frac{4}{3}E\right)
  \left(
    \frac{\partial^{3} u}{\partial x^{3}}
  \right)_{i,j}
  (\Delta x)^{3}
  +
  \\
  &
  \left(\frac{2}{3}A+\frac{B}{24}+\frac{D}{24}+\frac{2}{3}E\right)
  \left(
    \frac{\partial^{4} u}{\partial x^{4}}
  \right)_{i,j}
  (\Delta x)^{4}
  +
  \\
  &
  \left(-\frac{4}{15}A-\frac{B}{60}+\frac{D}{60}+\frac{4}{15}E\right)
  \left(
    \frac{\partial^{5} u}{\partial x^{5}}
  \right)_{i,j}
  (\Delta x)^{5}
  +
  \\
  &
  \left(\frac{4}{45}A+\frac{B}{120}+\frac{D}{120}+\frac{4}{45}E\right)
  \left(
    \frac{\partial^{6} u}{\partial x^{6}}
  \right)_{i,j}
  (\Delta x)^{6}
  +
  \cdots
\end{aligned}
\end{equation}
根据四阶精度要求,可得
\begin{equation}
  \begin{aligned}
    A+B+C+D+E = 0 \\
    -2A-B+D+2E = 0 \\
    \left(2A+\frac{B}{2} + \frac{D}{2} + 2E\right)(\Delta x)^{2} = 1 \\
    -\frac{4}{3}A-\frac{B}{6}+\frac{D}{6}+\frac{4}{3}E = 0 \\
    \frac{2}{3}A+\frac{B}{24}+\frac{D}{24}+\frac{2}{3}E = 0 \\
    -\frac{4}{15}A-\frac{B}{60}+\frac{D}{60}+\frac{4}{15}E = 0 \\
  \end{aligned}
\end{equation}
且
\begin{equation}
    \frac{4}{45}A+\frac{B}{120}+\frac{D}{120}+\frac{4}{45}E \neq 0
\end{equation}
从上两式可以解出:
\begin{equation}
  A = \frac{-1}{12(\Delta x)^{2}}
  ,
  B = \frac{16}{12(\Delta x)^{2}}
  ,
  C = \frac{-30}{12(\Delta x)^{2}}
  ,
  D = \frac{16}{12(\Delta x)^{2}}
  ,
  E = \frac{-1}{12(\Delta x)^{2}}
\end{equation}

\subsection{多项式拟合法}

\section{差分方程}
\section{相容性、稳定性和收敛性}

