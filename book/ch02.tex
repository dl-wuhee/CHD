\chapter{控制方程}
\section{纳维-斯托克斯方程}
%\subsection{模型}
\subsection{连续性微分方程}
连续性微分方程是质量守恒定律在流场中固定的无穷小微元控制体所导出的,即单位时间内
净流入控制体内的质量等于单位时间内控制体内质量的增加量。如图所示,在流场中任取
一无穷小正交六面体控制体,各边分别与直角坐标系各轴平行。设各边边长分别为$\mathrm{d}x$、
$\mathrm{d}y$和$\mathrm{d}z$。中心点坐标为$(x,y,z)$,密度为$\rho(x,y,z)$,速度为
$(u(x,y,z), v(x,y,z), w(x,y,z))$。

以$x$方向为例,流体从控制体的左侧和右侧两个面进出控制体。左侧面的中心点坐标为
$(x-\frac{\mathrm{d}x}{2},y,z)$。根据泰勒展开并略去高阶无穷小量,单位时间内从左侧面
流入控制体的质量为
\begin{equation*}
  \left[\rho u-\frac{\partial (\rho u)}{\partial x}\frac {\mathrm{d}x}{2}\right]\mathrm{d}y\mathrm{d}z
\end{equation*}

单位时间内从右侧面流出控制体的质量为
\begin{equation*}
  \left[\rho u+\frac{\partial (\rho u)}{\partial x}\frac {\mathrm{d}x}{2}\right]\mathrm{d}y\mathrm{d}z
\end{equation*}

因此,单位时间沿$x$方向净流入控制体的质量为
\begin{equation}
  -\frac{\partial (\rho u)}{\partial x}\mathrm{d}x\mathrm{d}y\mathrm{d}z
\end{equation}

同理,单位时间内沿$y$、$z$方向净流入控制体的质量分别为
\begin{equation*}
  -\frac{\partial (\rho v)}{\partial y}\mathrm{d}x\mathrm{d}y\mathrm{d}z
  \mbox{和}
  -\frac{\partial (\rho w)}{\partial z}\mathrm{d}x\mathrm{d}y\mathrm{d}z
\end{equation*}

因而单位时间内净流入控制体内的质量为
\begin{equation}
  -
  \left[
    \frac {\partial (\rho u)} {\partial x}
    +
    \frac {\partial (\rho v)} {\partial y}
    +
    \frac {\partial (\rho w)} {\partial z}
  \right]
  \mathrm{d}x\mathrm{d}y\mathrm{d}z
\end{equation}

控制体的质量为$\rho\mathrm{d}x\mathrm{d}y\mathrm{d}z$,单位时间内控制体内质量的增加量为
\begin{equation}
  \frac{\partial}{\partial t}(\rho\mathrm{d}x\mathrm{d}y\mathrm{d}z) =
  \frac{\partial \rho}{\partial  t}\mathrm{d}x\mathrm{d}y\mathrm{d}z
\end{equation}

根据质量守恒定律,有
\begin{equation}
  \frac{\partial \rho}{\partial  t}\mathrm{d}x\mathrm{d}y\mathrm{d}z
  =
  -
  \left[
    \frac {\partial (\rho u)} {\partial x}
    +
    \frac {\partial (\rho v)} {\partial y}
    +
    \frac {\partial (\rho w)} {\partial z}
  \right]
  \mathrm{d}x\mathrm{d}y\mathrm{d}z
\end{equation}
上式两边同除以$\mathrm{d}x\mathrm{d}y\mathrm{d}z$后,将等号右边所有项移到左边,即得连续性
微分方程
\begin{equation}
  \frac{\partial \rho}{\partial  t}
  +
  \frac{\partial (\rho u)}{\partial  x}
  +
  \frac{\partial (\rho v)}{\partial  y}
  +
  \frac{\partial (\rho w)}{\partial  z}
  =
  0
\end{equation}
对均质不可压缩流体,$\rho$为常数,连续性方程可写成
\begin{equation}
  \frac{\partial u}{\partial  x}
  +
  \frac{\partial v}{\partial  y}
  +
  \frac{\partial w}{\partial  z}
  =
  0
  \label{EqCGe_NS_Ce}
\end{equation}

\subsection{动量方程}
动量方程是牛顿第二定律或动量守恒定律得数学表达式。考虑一个随流运动的无穷小正交六
面体流体微团。微团中心点所在坐标为$(x,y,z)$,中心点密度为$\rho$,中心点速度为
$(u, v, w)$。则该微团的质量
为
\begin{equation}
  m = \rho\mathrm{d}x\mathrm{d}y\mathrm{d}z
\end{equation}

$x$方向的牛顿第二定律为
\begin{equation}
  F_{x} = ma_{x}
  \label{EqCGe_Nt_x}
\end{equation}
式中,$F_{x}$和$a_{x}$分别是微团所受外力和加速度在$x$方向的分量。

随流运动的流体微团的$a_{x}$等于$u_{x}$随时间的变化率,即速度的
随体导数
\begin{equation}
  a_{x} =
  \frac{\mathrm{D}u}{\mathrm{D}t}
  \label{EqCGe_Ac_x}
\end{equation}

流体微团在$x$方向所受外力有两类:体积力$F_{x}$和表面力$T_{x}$。假定作用在微团上的单位质量力为
$\mathbf{f}=(f_{x}, f_{y}, f_{z})$。$x$方向所受的体积力为:
\begin{equation}
  F_{x} = \rho f_{x}\mathrm{d}x\mathrm{d}y\mathrm{d}z
  \label{EqCGe_Bf_x}
\end{equation}
表面力是直接作用在六个外表面上的力,包括:(1)作用在外表面上的压强;(2)作用在
外表面上正应力和切应力。

\begin{equation}
  \begin{aligned}
    T_{x} =&
    \left[
      p -
      \left(
        p +
        \frac{\partial p}{\partial x}\mathrm{d}x
      \right)
    \right]\mathrm{d}y\mathrm{d}z
    +
    \left[
      \left(
        \tau_{xx} +
        \frac{\partial \tau_{xx}}{\partial x}\mathrm{d}x
      \right)
      - \tau_{xx}
    \right]\mathrm{d}y\mathrm{d}z \\
           & +
           \left[
             \left(
               \tau_{yx} +
               \frac{\partial \tau_{yx}}{\partial y}\mathrm{d}y
             \right)
             - \tau_{yx}
           \right]\mathrm{d}x\mathrm{d}z
           +
           \left[
             \left(
               \tau_{zx} +
               \frac{\partial \tau_{zx}}{\partial z}\mathrm{d}z
             \right)
             - \tau_{zx}
           \right]\mathrm{d}x\mathrm{d}y
           \\
    =&
    \left(
      -\frac{\partial p}{\partial x}
      +\frac{\partial \tau_{xx}}{\partial x}
      +\frac{\partial \tau_{yx}}{\partial y}
      +\frac{\partial \tau_{zx}}{\partial z}
    \right)
    \mathrm{d}x\mathrm{d}y\mathrm{d}z
  \end{aligned}
  \label{EqCGe_Sf_x}
\end{equation}
将式\eqref{EqCGe_Ac_x}、\eqref{EqCGe_Bf_x}和\eqref{EqCGe_Sf_x}代入式
\eqref{EqCGe_Nt_x},可得到$x$方向的动量方程。同理可得,$y$和$z$方向的动量方程。
最终得到的动量方程为
\begin{equation}
  \begin{aligned}
    \rho \frac{\mathrm{D}u}{\mathrm{D}t} =
    \rho f_{x}
    -\frac{\partial p}{\partial x}
    +\frac{\partial \tau_{xx}}{\partial x}
    +\frac{\partial \tau_{yx}}{\partial y}
    +\frac{\partial \tau_{zx}}{\partial z}
    \\
    \rho \frac{\mathrm{D}v}{\mathrm{D}t} =
    \rho f_{y}
    -\frac{\partial p}{\partial y}
    +\frac{\partial \tau_{xy}}{\partial x}
    +\frac{\partial \tau_{yy}}{\partial y}
    +\frac{\partial \tau_{zy}}{\partial z}
    \\
    \rho \frac{\mathrm{D}w}{\mathrm{D}t} =
    \rho f_{z}
    -\frac{\partial p}{\partial z}
    +\frac{\partial \tau_{xz}}{\partial x}
    +\frac{\partial \tau_{yz}}{\partial y}
    +\frac{\partial \tau_{zz}}{\partial z}
  \end{aligned}
  \label{EqCGe_NS_Me_ori}
\end{equation}

根据数学知识,
\begin{equation}
  \begin{aligned}
    \rho \frac{\mathrm{D}u}{\mathrm{D}t}
    =
    \frac{\partial (\rho u)}{\partial t}
    +
    \nabla\cdot(\rho u\mathbf{u})
    =
    \frac{\partial (\rho u)}{\partial t}
    +
    \frac{\partial \rho (uu)}{\partial x}
    +
    \frac{\partial \rho (uv)}{\partial y}
    +
    \frac{\partial \rho (uw)}{\partial z}
    \\
    \rho \frac{\mathrm{D}v}{\mathrm{D}t}
    =
    \frac{\partial (\rho v)}{\partial t}
    +
    \nabla\cdot(\rho v\mathbf{u})
    =
    \frac{\partial (\rho v)}{\partial t}
    +
    \frac{\partial \rho (vu)}{\partial x}
    +
    \frac{\partial \rho (vv)}{\partial y}
    +
    \frac{\partial \rho (vw)}{\partial z}
    \\
    \rho \frac{\mathrm{D}w}{\mathrm{D}t}
    =
    \frac{\partial (\rho w)}{\partial t}
    +
    \nabla\cdot(\rho w\mathbf{u})
    =
    \frac{\partial (\rho w)}{\partial t}
    +
    \frac{\partial \rho (wu)}{\partial x}
    +
    \frac{\partial \rho (wv)}{\partial y}
    +
    \frac{\partial \rho (ww)}{\partial z}
  \end{aligned}
  \label{EqCGe_div}
\end{equation}
将式\eqref{EqCGe_div}代入式\eqref{EqCGe_NS_Me_ori}中,
\begin{equation}
  \begin{aligned}
    \frac{\partial (\rho u)}{\partial t}
    +
    \frac{\partial \rho (uu)}{\partial x}
    +
    \frac{\partial \rho (uv)}{\partial y}
    +
    \frac{\partial \rho (uw)}{\partial z}
    =
    \rho f_{x}
    -\frac{\partial p}{\partial x}
    +\frac{\partial \tau_{xx}}{\partial x}
    +\frac{\partial \tau_{yx}}{\partial y}
    +\frac{\partial \tau_{zx}}{\partial z}
    \\
    \frac{\partial (\rho v)}{\partial t}
    +
    \frac{\partial \rho (vu)}{\partial x}
    +
    \frac{\partial \rho (vv)}{\partial y}
    +
    \frac{\partial \rho (vw)}{\partial z}
    =
    \rho f_{y}
    -\frac{\partial p}{\partial y}
    +\frac{\partial \tau_{xy}}{\partial x}
    +\frac{\partial \tau_{yy}}{\partial y}
    +\frac{\partial \tau_{zy}}{\partial z}
    \\
    \frac{\partial (\rho w)}{\partial t}
    +
    \frac{\partial \rho (wu)}{\partial x}
    +
    \frac{\partial \rho (wv)}{\partial y}
    +
    \frac{\partial \rho (ww)}{\partial z}
    =
    \rho f_{z}
    -\frac{\partial p}{\partial z}
    +\frac{\partial \tau_{xz}}{\partial x}
    +\frac{\partial \tau_{yz}}{\partial y}
    +\frac{\partial \tau_{zz}}{\partial z}
  \end{aligned}
  \label{EqCGe_NS_Me_general}
\end{equation}
在17世纪晚期,牛顿给出了牛顿内摩擦定律,即流体内部切应力正比于剪切变形速率。1845
年,斯托克斯给出如下关系式
\begin{equation}
  \begin{aligned}
  &\tau_{xx} = \lambda(\nabla\cdot\mathbf{u}) + 2\mu\frac{\partial u}{\partial x}
\\&
\tau_{yy} = \lambda(\nabla\cdot\mathbf{u}) + 2\mu\frac{\partial v}{\partial y}
\\&
\tau_{zz} = \lambda(\nabla\cdot\mathbf{u}) + 2\mu\frac{\partial w}{\partial z}
\\&
\tau_{xy} = \tau_{yx} =
\mu
\left(
  \frac{\partial v}{\partial x}+\frac{\partial u}{\partial y}
\right)
\\&
\tau_{xz} = \tau_{zx} =
\mu
\left(
  \frac{\partial u}{\partial z}+\frac{\partial w}{\partial x}
\right)
\\&
\tau_{yz} = \tau_{zy} =
\mu
\left(
  \frac{\partial w}{\partial y}+\frac{\partial v}{\partial z}
\right)
  \end{aligned}
  \label{EqCGe_Stokes}
\end{equation}
式中,$\mu$为流体动力粘滞系数,$\lambda$是第二粘滞系数。斯托克斯假定
\begin{equation}
  \lambda = -\frac{2}{3}\mu
  \label{EqCGe_Stokes_labmda}
\end{equation}
将式\eqref{EqCGe_Stokes}和\eqref{EqCGe_Stokes_labmda}代入式
\eqref{EqCGe_NS_Me_general},得
\begin{subequations}
  \begin{align}
    \begin{split}
      \frac{\partial (\rho u)}{\partial t}
      +&
      \frac{\partial \rho (uu)}{\partial x}
      +
      \frac{\partial \rho (uv)}{\partial y}
      +
      \frac{\partial \rho (uw)}{\partial z}
      =
      \rho f_{x}
      -\frac{\partial p}{\partial x} \\
       &+\frac{\partial }{\partial x}
       \left[
         \lambda(\nabla\cdot\mathbf{u}) + 2\mu\frac{\partial u}{\partial x}
       \right]
       +\frac{\partial }{\partial y}
       \left[
         \mu
         \left(
           \frac{\partial v}{\partial x}+\frac{\partial u}{\partial y}
         \right)
       \right]
       +\frac{\partial }{\partial z}
       \left[
         \mu
         \left(
           \frac{\partial u}{\partial z}+\frac{\partial w}{\partial x}
         \right)
       \right]
    \end{split}
    \label{EqCGe_NS_Me_general_expand_1}
    \\
    \begin{split}
      \frac{\partial (\rho v)}{\partial t}
      +&
      \frac{\partial \rho (vu)}{\partial x}
      +
      \frac{\partial \rho (vv)}{\partial y}
      +
      \frac{\partial \rho (vw)}{\partial z}
      =
      \rho f_{y}
      -\frac{\partial p}{\partial y} \\
       &+\frac{\partial }{\partial x}
       \left[
         \mu
         \left(
           \frac{\partial v}{\partial x}+\frac{\partial u}{\partial y}
         \right)
       \right]
       +\frac{\partial }{\partial y}
       \left[
         \lambda(\nabla\cdot\mathbf{u}) + 2\mu\frac{\partial v}{\partial y}
       \right]
       +\frac{\partial }{\partial z}
       \left[
         \mu
         \left(
           \frac{\partial w}{\partial y}+\frac{\partial v}{\partial z}
         \right)
       \right]
    \end{split}
    \\
    \begin{split}
      \frac{\partial (\rho w)}{\partial t}
    &+
    \frac{\partial \rho (wu)}{\partial x}
    +
    \frac{\partial \rho (wv)}{\partial y}
    +
    \frac{\partial \rho (ww)}{\partial z}
    =
    \rho f_{z}
    -\frac{\partial p}{\partial z} \\
    &+\frac{\partial }{\partial x}
    \left[
      \mu
      \left(
        \frac{\partial u}{\partial z}+\frac{\partial w}{\partial x}
      \right)
    \right]
    +\frac{\partial }{\partial y}
    \left[
      \mu
      \left(
        \frac{\partial w}{\partial y}+\frac{\partial v}{\partial z}
      \right)
    \right]
    +\frac{\partial }{\partial z}
    \left[
      \lambda(\nabla\cdot\mathbf{u}) + 2\mu\frac{\partial w}{\partial z}
    \right]
    \end{split}
  \end{align}
  \label{EqCGe_NS_Me_general_expand}
\end{subequations}
式\eqref{EqCGe_NS_Me_general_expand_1}中第二行的各项可以进一步展开为
\begin{equation}
  \begin{aligned}
&\frac{\partial }{\partial x}
\left[
  \lambda(\nabla\cdot\mathbf{u}) + 2\mu\frac{\partial u}{\partial x}
\right]
+\frac{\partial }{\partial y}
\left[
  \mu
  \left(
    \frac{\partial v}{\partial x}+\frac{\partial u}{\partial y}
  \right)
\right]
+\frac{\partial }{\partial z}
\left[
  \mu
  \left(
    \frac{\partial u}{\partial z}+\frac{\partial w}{\partial x}
  \right)
\right]
\\
&=
\mu
\left(
  \frac{\partial^{2} u}{\partial x^{2}} +
  \frac{\partial^{2} u}{\partial y^{2}} +
  \frac{\partial^{2} u}{\partial z^{2}}
\right)
+
\frac{\partial}{\partial x}
\left(
  \frac{\partial u}{\partial x} +
  \frac{\partial v}{\partial y} +
  \frac{\partial w}{\partial z}
\right)
+
\frac{\partial}{\partial x}[\lambda(\nabla\cdot\mathbf{u})]
\\
&=
\mu
\left(
  \frac{\partial^{2} u}{\partial x^{2}} +
  \frac{\partial^{2} u}{\partial y^{2}} +
  \frac{\partial^{2} u}{\partial z^{2}}
\right)
+
\frac{\partial}{\partial x}
\left(
  \nabla\cdot\mathbf{u}
\right)
+
\frac{\partial}{\partial x}[\lambda(\nabla\cdot\mathbf{u})]
  \end{aligned}
\end{equation}
对均质不可压缩流体,其连续性方程为$\nabla\cdot\mathbf{u}=0$,上式最后得到
\begin{equation}
  \mu
  \left(
    \frac{\partial^{2} u}{\partial x^{2}} +
    \frac{\partial^{2} u}{\partial y^{2}} +
    \frac{\partial^{2} u}{\partial z^{2}}
  \right)
\end{equation}
对\eqref{EqCGe_NS_Me_general_expand}的后两个公式采用同样的处理方式,并代入均质不
可压缩条件,式\eqref{EqCGe_NS_Me_general_expand}可以写成
\begin{subequations}
  \begin{align}
    \frac{\partial u}{\partial t}
    +
    \frac{\partial (uu)}{\partial x}
    +
    \frac{\partial (uv)}{\partial y}
    +
    \frac{\partial (uw)}{\partial z}
    =
    f_{x}
    -\frac{1}{\rho}\frac{\partial p}{\partial x}
    +
    \nu
    \left(
      \frac{\partial^{2} u}{\partial x^{2}} +
      \frac{\partial^{2} u}{\partial y^{2}} +
      \frac{\partial^{2} u}{\partial z^{2}}
    \right)
    \label{EqCGe_NS_Me_x}
    \\
    \frac{\partial v}{\partial t}
    +
    \frac{\partial (vu)}{\partial x}
    +
    \frac{\partial (vv)}{\partial y}
    +
    \frac{\partial (vw)}{\partial z}
    =
    f_{y}
    -\frac{1}{\rho}\frac{\partial p}{\partial y}
    +
    \nu
    \left(
      \frac{\partial^{2} v}{\partial x^{2}} +
      \frac{\partial^{2} v}{\partial y^{2}} +
      \frac{\partial^{2} v}{\partial z^{2}}
    \right)
    \\
    \frac{\partial w}{\partial t}
    +
    \frac{\partial (wu)}{\partial x}
    +
    \frac{\partial (wv)}{\partial y}
    +
    \frac{\partial (ww)}{\partial z}
    =
    f_{z}
    -\frac{1}{\rho}\frac{\partial p}{\partial z}
    +
    \nu
    \left(
      \frac{\partial^{2} w}{\partial x^{2}} +
      \frac{\partial^{2} w}{\partial y^{2}} +
      \frac{\partial^{2} w}{\partial z^{2}}
    \right)
  \end{align}
  \label{EqCGe_NS_Me}
\end{subequations}
上式写成矢量形式为
\begin{equation}
  \frac{\partial \mathbf{u}}{\partial x} +
  \nabla\cdot(\mathbf{u}\mathbf{u})
  =
  \mathbf{f} -
  \frac{1}{\rho}\nabla p +
  \nu\nabla^{2}\mathbf{u}
\end{equation}


\section{雷诺时均方程}
\subsection{紊流物理量时均值定义及性质}
根据大量相关实验观测,紊流具有有涡性、不规则性、耗能性、连续性、三维性以及非定常
性等特征。因此,紊流运动要素都具有随机性。为了分析方便,紊流运动要素的瞬时值
常被分解成时均值和脉动值。任一运动要素$\phi$的时均值定义为:
\begin{equation}
  \overline{\phi}
  =
  \frac{1}{\Delta t}
  \int_{t}^{t+\Delta t}\!
  \phi(t)
  \mathrm{d}t
  \label{EqCGe_RA}
\end{equation}
其中,时间间隔$\Delta t$相对于紊流的随机脉动周期而言足够大,但相对于流场的各种时
均量的缓慢变化周期来说,应足够小。如果时均值随时间变化,成为非稳态的时均紊流;如
果时均值不随时间变化,成为准稳态紊流,简称稳态紊流。

运动要素的瞬时值$\phi$,时均值$\overline{\phi}$及脉动值$\phi^{\prime}$之间有如下
关系:
\begin{equation}
  \phi = \overline{\phi} + \phi^{\prime}
  \label{EqCGe_RA_Comp}
\end{equation}

设$\phi$和$f$是两个瞬时值,$\overline{\phi}$和$\overline{f}$为相应的时均值,
$\phi^{\prime}$和$f^{\prime}$为相应的脉动值。按照式\eqref{EqCGe_RA}和
\eqref{EqCGe_RA_Comp},下列基本关系成立:
\begin{equation}
  \begin{aligned}
  &\overline{\phi^{\prime}} = 0
  \quad\quad
  \overline{\overline{\phi}} = \overline{\phi}
  \quad\quad
  \overline{\overline{\phi}+\phi^{\prime}} = \overline{\phi}
  \\
  & \overline{\overline{\phi}f} = \overline{\phi}\overline{f}
  \quad\quad
  \overline{\overline{\phi}\overline{f}} = \overline{\phi}\overline{f}
  \quad\quad
  \overline{\overline{\phi}f^{\prime}} = 0
  \quad\quad
  \overline{\phi f} = \overline{\phi}\overline{f} +
  \overline{\phi^{\prime}f^{\prime}}
  \\
  & \overline{\frac{\partial\phi}{\partial x_{i}}} = \frac{\partial \overline{\phi}}{\partial x_{i}}
  \quad\quad
  \overline{\frac{\partial^{2}\phi}{\partial x_{i}^{2}}} = \frac{\partial^{2}
  \overline{\phi}}{\partial x_{i}^{2}}
  \quad\quad
  \overline{\frac{\partial\phi^{\prime}}{\partial x_{i}}} = 0
  \quad\quad
  \overline{\frac{\partial^{2}\phi^{\prime}}{\partial x_{i}^{2}}} = 0
  \end{aligned}
  \label{EqCGe_RA_cal}
\end{equation}

\subsection{雷诺时均连续性方程}
将$u$、$v$和$w$表示成时均值和脉动值之和,并带入连续性方程\eqref{EqCGe_NS_Ce},并做时均运算
,
\begin{equation*}
  \overline{
    \frac{\partial(\overline{u}+u^{\prime})}{\partial x}
  }
  +
  \overline{
    \frac{\partial(\overline{v}+v^{\prime})}{\partial y}
  }
  +
  \overline{
    \frac{\partial(\overline{w}+w^{\prime})}{\partial z}
  }
  =
  \frac{\partial \overline{u}}{\partial x} +
  \frac{\partial \overline{v}}{\partial y} +
  \frac{\partial \overline{w}}{\partial z} +
  \frac{\partial \overline{u^{\prime}}}{\partial x} +
  \frac{\partial \overline{v^{\prime}}}{\partial y} +
  \frac{\partial \overline{w^{\prime}}}{\partial z}
  =0
\end{equation*}
运用式\eqref{EqCGe_RA_cal},可得
\begin{equation}
  \frac{\partial \overline{u}}{\partial x} +
  \frac{\partial \overline{v}}{\partial y} +
  \frac{\partial \overline{w}}{\partial z}
  =
  0
\end{equation}
上式表明,紊流速度的时均值仍满足连续性方程。

\subsection{雷诺时均运动方程}
以式\eqref{EqCGe_NS_Me_x}给出的$x$方向的运动方程为例,采取类似于连续性方程的处理,有
\begin{equation*}
  \begin{aligned}
    \overline{
      \frac{\partial (\overline{u}+u^{\prime})}{\partial t}
    }
    +
    \overline{
      \frac{\partial (\overline{u}+u^{\prime})^{2}}{\partial x}
    }
    +
    \overline{
      \frac{\partial (\overline{u}+u^{\prime})(\overline{v}+v^{\prime})}{\partial y}
    }
    +
    \overline{
      \frac{\partial (\overline{u}+u^{\prime})(\overline{w}+w^{\prime})}{\partial z}
    }
    \\
    =
    \overline{f_{x}}
    -\frac{1}{\rho}
    \overline{
      \frac{\partial (\overline{p}+p^{\prime})}{\partial x}
    }
    +
    \nu
    \left[
      \overline{
        \frac{\partial^{2} (\overline{u}+u^{\prime})}{\partial x^{2}}
      }
      +
      \overline{
        \frac{\partial^{2} (\overline{u}+u^{\prime})}{\partial y^{2}}
      }
      +
      \overline{
        \frac{\partial^{2} (\overline{u}+u^{\prime})}{\partial z^{2}}
      }
    \right ]
  \end{aligned}
\end{equation*}
运用式\eqref{EqCGe_RA_cal},可得
\begin{equation*}
  \frac{\partial \overline{u}}{\partial t} +
  \frac{\partial \widebar{u}\widebar{u}}{\partial x} +
  \frac{\partial \widebar{u}\widebar{v}}{\partial y} +
  \frac{\partial \widebar{u}\widebar{w}}{\partial z} +
  \frac{\partial \overline{u^{\prime}u^{\prime}}}{\partial x} +
  \frac{\partial \overline{u^{\prime}v^{\prime}}}{\partial y} +
  \frac{\partial \overline{u^{\prime}w^{\prime}}}{\partial z}
  =
  \overline{f_{x}}
  -\frac{1}{\rho}\frac{\partial \overline{p}}{\partial x} +
  \nu
  \left(
    \frac{\partial^{2} \overline{u}}{\partial x^{2}} +
    \frac{\partial^{2} \overline{u}}{\partial y x^{2}} +
    \frac{\partial^{2} \overline{u}}{\partial z x^{2}}
  \right)
\end{equation*}
将上式左侧脉动值乘积的时均值移到等号右侧,得:
\begin{equation}
  \frac{\partial \overline{u}}{\partial t} +
  \frac{\partial \widebar{u}\widebar{u}}{\partial x} +
  \frac{\partial \widebar{u}\widebar{v}}{\partial y} +
  \frac{\partial \widebar{u}\widebar{w}}{\partial z}
  =
  \overline{f_{x}}
  -\frac{1}{\rho}\frac{\partial \overline{p}}{\partial x} -
  \left(
    \frac{\partial \overline{u^{\prime}u^{\prime}}}{\partial x} +
    \frac{\partial \overline{u^{\prime}v^{\prime}}}{\partial y} +
    \frac{\partial \overline{u^{\prime}w^{\prime}}}{\partial z}
  \right)
  +
  \nu
  \left(
    \frac{\partial^{2} \overline{u}}{\partial x^{2}} +
    \frac{\partial^{2} \overline{u}}{\partial y x^{2}} +
    \frac{\partial^{2} \overline{u}}{\partial z x^{2}}
  \right)
  \label{EqCGe_Ra_Me_temp1}
\end{equation}
同理可得,$y$和$z$方向的雷诺平均方程为:
\begin{equation}
  \frac{\partial \overline{v}}{\partial t} +
  \frac{\partial \widebar{v}\widebar{u}}{\partial x} +
  \frac{\partial \widebar{v}\widebar{v}}{\partial y} +
  \frac{\partial \widebar{v}\widebar{w}}{\partial z}
  =
  \overline{f_{y}}
  -\frac{1}{\rho}\frac{\partial \overline{p}}{\partial y} -
  \left(
    \frac{\partial \overline{v^{\prime}u^{\prime}}}{\partial x} +
    \frac{\partial \overline{v^{\prime}v^{\prime}}}{\partial y} +
    \frac{\partial \overline{v^{\prime}w^{\prime}}}{\partial z}
  \right)
  +
  \nu
  \left(
    \frac{\partial^{2} \overline{v}}{\partial x^{2}} +
    \frac{\partial^{2} \overline{v}}{\partial y^{2}} +
    \frac{\partial^{2} \overline{v}}{\partial z^{2}}
  \right)
  \label{EqCGe_Ra_Me_temp2}
\end{equation}
\begin{equation}
  \frac{\partial \overline{w}}{\partial t} +
  \frac{\partial \widebar{w}\widebar{u}}{\partial x} +
  \frac{\partial \widebar{w}\widebar{v}}{\partial y} +
  \frac{\partial \widebar{w}\widebar{w}}{\partial z}
  =
  \overline{f_{z}}
  -\frac{1}{\rho}\frac{\partial \overline{p}}{\partial z} -
  \left(
    \frac{\partial \overline{w^{\prime}u^{\prime}}}{\partial x} +
    \frac{\partial \overline{w^{\prime}v^{\prime}}}{\partial y} +
    \frac{\partial \overline{w^{\prime}w^{\prime}}}{\partial z}
  \right)
  +
  \nu
  \left(
    \frac{\partial^{2} \overline{w}}{\partial x^{2}} +
    \frac{\partial^{2} \overline{w}}{\partial y^{2}} +
    \frac{\partial^{2} \overline{w}}{\partial z^{2}}
  \right)
  \label{EqCGe_Ra_Me_temp3}
\end{equation}

对式\eqref{EqCGe_Ra_Me_temp1}至\eqref{EqCGe_Ra_Me_temp3}分别乘上密度$\rho$,可得
\begin{equation}
  \begin{aligned}
    \frac{\partial \rho\overline{u}}{\partial t} +
  &\frac{\partial \rho\widebar{u}\widebar{u}}{\partial x} +
  \frac{\partial \rho\widebar{u}\widebar{v}}{\partial y} +
  \frac{\partial \rho\widebar{u}\widebar{w}}{\partial z}
  = \\
  &\rho \overline{f_{x}}
  -\frac{\partial \overline{p}}{\partial x} +
  \left[
    \frac{\partial (-\rho\overline{u^{\prime}u^{\prime}})}{\partial x} +
    \frac{\partial (-\rho\overline{u^{\prime}v^{\prime}})}{\partial y} +
    \frac{\partial (-\rho\overline{u^{\prime}w^{\prime}})}{\partial z}
  \right]
  +
  \mu
  \left(
    \frac{\partial^{2} \overline{u}}{\partial x^{2}} +
    \frac{\partial^{2} \overline{u}}{\partial y x^{2}} +
    \frac{\partial^{2} \overline{u}}{\partial z x^{2}}
  \right)
  \label{EqCGe_Ra_Me_1}
  \end{aligned}
\end{equation}
\begin{equation}
  \begin{aligned}
    \frac{\partial \rho\overline{v}}{\partial t} +
  &\frac{\partial \rho\widebar{v}\widebar{u}}{\partial x} +
  \frac{\partial \rho\widebar{v}\widebar{v}}{\partial y} +
  \frac{\partial \rho\widebar{v}\widebar{w}}{\partial z}
  = \\
  &\rho \overline{f_{y}}
  -\frac{\partial \overline{p}}{\partial y} +
  \left[
    \frac{\partial (-\rho\overline{v^{\prime}u^{\prime}})}{\partial x} +
    \frac{\partial (-\rho\overline{v^{\prime}v^{\prime}})}{\partial y} +
    \frac{\partial (-\rho\overline{v^{\prime}w^{\prime}})}{\partial z}
  \right]
  +
  \nu
  \left(
    \frac{\partial^{2} \overline{v}}{\partial x^{2}} +
    \frac{\partial^{2} \overline{v}}{\partial y^{2}} +
    \frac{\partial^{2} \overline{v}}{\partial z^{2}}
  \right)
  \label{EqCGe_Ra_Me_2}
  \end{aligned}
\end{equation}
\begin{equation}
  \begin{aligned}
    \frac{\partial \rho\overline{w}}{\partial t} +
  &\frac{\partial \rho\widebar{w}\widebar{u}}{\partial x} +
  \frac{\partial \rho\widebar{w}\widebar{v}}{\partial y} +
  \frac{\partial \rho\widebar{w}\widebar{w}}{\partial z}
  = \\
  &\rho \overline{f_{z}}
  -\frac{\partial \overline{p}}{\partial z} +
  \left[
    \frac{\partial (-\rho\overline{w^{\prime}u^{\prime}})}{\partial x} +
    \frac{\partial (-\rho\overline{w^{\prime}v^{\prime}})}{\partial y} +
    \frac{\partial (-\rho\overline{w^{\prime}w^{\prime}})}{\partial z}
  \right]
  +
  \mu
  \left(
    \frac{\partial^{2} \overline{w}}{\partial x^{2}} +
    \frac{\partial^{2} \overline{w}}{\partial y^{2}} +
    \frac{\partial^{2} \overline{w}}{\partial z^{2}}
  \right)
  \label{EqCGe_Ra_Me_3}
  \end{aligned}
\end{equation}
式\eqref{EqCGe_Ra_Me_1}至\eqref{EqCGe_Ra_Me_3}即紊流时均流动的运动微分方程,由雷
诺导出,通常称为雷诺时均方程。与纳维-斯托克斯方程相比,雷诺时均方程多出了紊动附
加应力(也称雷诺应力)项$-\rho \overline{u^{\prime}_{i}u^{\prime}_{j}}$,
$i,j=1,2,3$分别对应$x$,$y$和$z$方向。当$i=j$时,为紊动产生的时均附加正应力,
当$i\neq j$时,为紊动产生的时均附加切应力。

在雷诺时均方程组中,紊动附加应力共有9项,其中只有6个独立变量,另外还有4个变量(
$\widebar{u}$,$\widebar{v}$,$\widebar{w}$和$\widebar{p}$),而雷诺时均方程组只
有4个方程。因此,雷诺时均方程不封闭,必须附加方程或条件才能求解出上述10个变量。
根据附加方程或条件数目不同,用来封闭的紊流模型可分为零方程模型、一方程模型、二方
程模型和代数应力模型等。在水动力学数值模拟中,应用较多的是零方程模型。下面主要介
绍零方程模型。其他紊流模型可参考其他教材。

\subsection{零方程紊流模型}
布辛涅斯克类比层流粘性力等于动力粘滞系数$\mu$乘以旋转角速度的二倍,提出雷诺应力
等于紊动动力粘滞系数(或称涡粘性系数)$\eta$乘以时均角速度的二倍,即
\begin{equation}
  -\rho \overline{u^{\prime}_{i}u^{\prime}_{j}}
  =
  \eta
  \left(
    \frac{\partial \overline{u_{i}}}{\partial x_{j}} +
    \frac{\partial \overline{u_{j}}}{\partial x_{j}}
  \right)
\end{equation}

类似层流运动粘滞系数$\nu$,定义$\nu_{t}=\eta/\rho$为紊动运动粘滞系数。布辛涅斯
克将紊动附加应力与时均流速联系起来,使雷诺时均方程组封闭,为解决紊流问题开辟了一
条很好的途径。但是,布辛涅斯克假定$\eta$为常量,与实际有一定出入。普朗特于1952年
提出动量传递理论和掺长假设,给出了计算二维恒定均匀紊流附加应力的半经验公式
\begin{equation}
  \tau_{xy}
  =
  -\rho \overline{u^{\prime}v^{\prime}}
  =
  \rho l^{2}
  \left|
  \frac{\partial \overline{u}}{\partial y}
  \right|
  \frac{\partial \overline{u}}{\partial y}
\end{equation}
式中,$l$为掺长。普朗特认为,在近壁区,可以假定讨论点的掺长与该店至壁面的距离成
正比,即
\begin{equation}
  l = \kappa y
\end{equation}
式中,$\kappa$为卡门常数,由实验资料确定。普朗特掺长理论在近壁区给出的结果同实际
资料吻合较好,因而得到广泛应用。

根据布辛涅斯克假定,雷诺时均方程可以变为:
\begin{equation}
  \begin{gathered}
    \frac{\partial \overline{u}}{\partial x} +
    \frac{\partial \overline{v}}{\partial y} +
    \frac{\partial \overline{w}}{\partial z}
    =
    0
    \\
    \begin{split}
      \frac{\partial \overline{u}}{\partial t} +
  &\frac{\partial (\widebar{u}\widebar{u})}{\partial x} +
  \frac{\partial (\widebar{u}\widebar{v})}{\partial y} +
  \frac{\partial (\widebar{u}\widebar{w})}{\partial z}
  = \\
  &\overline{f_{x}} -
  \frac{1}{\rho}\frac{\partial \overline{p}}{\partial x} +
  \nu_{t}
  \left(
    \frac{\partial^{2} \overline{u}}{\partial x^{2}} +
    \frac{\partial^{2} \overline{u}}{\partial y^{2}} +
    \frac{\partial^{2} \overline{u}}{\partial z^{2}}
  \right)
  +
  \nu
  \left(
    \frac{\partial^{2} \overline{u}}{\partial x^{2}} +
    \frac{\partial^{2} \overline{u}}{\partial y^{2}} +
    \frac{\partial^{2} \overline{u}}{\partial z^{2}}
  \right)
    \end{split}
    \\
    \begin{split}
      \frac{\partial \overline{v}}{\partial t} +
  &\frac{\partial (\widebar{v}\widebar{u})}{\partial x} +
  \frac{\partial (\widebar{v}\widebar{v})}{\partial y} +
  \frac{\partial (\widebar{v}\widebar{w})}{\partial z}
  = \\
  &\overline{f_{y}} -
  \frac{1}{\rho}\frac{\partial \overline{p}}{\partial y} +
  \nu_{t}
  \left(
    \frac{\partial^{2} \overline{v}}{\partial x^{2}} +
    \frac{\partial^{2} \overline{v}}{\partial y^{2}} +
    \frac{\partial^{2} \overline{v}}{\partial z^{2}}
  \right)
  +
  \nu
  \left(
    \frac{\partial^{2} \overline{v}}{\partial x^{2}} +
    \frac{\partial^{2} \overline{v}}{\partial y^{2}} +
    \frac{\partial^{2} \overline{v}}{\partial z^{2}}
  \right)
    \end{split}
    \\
    \begin{split}
      \frac{\partial \overline{w}}{\partial t} +
  &\frac{\partial (\widebar{w}\widebar{u})}{\partial x} +
  \frac{\partial (\widebar{w}\widebar{v})}{\partial y} +
  \frac{\partial (\widebar{w}\widebar{w})}{\partial z}
  = \\
  &\overline{f_{z}} -
  \frac{1}{\rho}\frac{\partial \overline{p}}{\partial z} +
  \nu_{t}
  \left(
    \frac{\partial^{2} \overline{w}}{\partial x^{2}} +
    \frac{\partial^{2} \overline{w}}{\partial y^{2}} +
    \frac{\partial^{2} \overline{w}}{\partial z^{2}}
  \right)
  +
  \nu
  \left(
    \frac{\partial^{2} \overline{w}}{\partial x^{2}} +
    \frac{\partial^{2} \overline{w}}{\partial y^{2}} +
    \frac{\partial^{2} \overline{w}}{\partial z^{2}}
  \right)
    \end{split}
  \end{gathered}
\end{equation}

层流的动力粘滞系数$\mu$一般远小于紊动动力粘滞系数$\eta$,因此上式中层流阻力项可
以忽略,雷诺时均方程为
\begin{subequations}
  \begin{align}
    \frac{\partial \overline{u}}{\partial x} +
    \frac{\partial \overline{v}}{\partial y} +
    \frac{\partial \overline{w}}{\partial z}
  &
  =
  0
  \label{EqRaeC}
  \\
  \frac{\partial \overline{u}}{\partial t} +
  \frac{\partial (\widebar{u}\widebar{u})}{\partial x} +
  \frac{\partial (\widebar{u}\widebar{v})}{\partial y} +
  \frac{\partial (\widebar{u}\widebar{w})}{\partial z}
  &
  =
  \overline{f_{x}} -
  \frac{1}{\rho}\frac{\partial \overline{p}}{\partial x} +
  \nu_{t}
  \left(
    \frac{\partial^{2} \overline{u}}{\partial x^{2}} +
    \frac{\partial^{2} \overline{u}}{\partial y^{2}} +
    \frac{\partial^{2} \overline{u}}{\partial z^{2}}
  \right)
  \label{EqRaeMex}
  \\
  \frac{\partial \overline{v}}{\partial t} +
  \frac{\partial (\widebar{v}\widebar{u})}{\partial x} +
  \frac{\partial (\widebar{v}\widebar{v})}{\partial y} +
  \frac{\partial (\widebar{v}\widebar{w})}{\partial z}
  &
  =
  \overline{f_{y}} -
  \frac{1}{\rho}\frac{\partial \overline{p}}{\partial y} +
  \nu_{t}
  \left(
    \frac{\partial^{2} \overline{v}}{\partial x^{2}} +
    \frac{\partial^{2} \overline{v}}{\partial y^{2}} +
    \frac{\partial^{2} \overline{v}}{\partial z^{2}}
  \right)
  \label{EqRaeMey}
  \\
  \frac{\partial \overline{w}}{\partial t} +
  \frac{\partial (\widebar{w}\widebar{u})}{\partial x} +
  \frac{\partial (\widebar{w}\widebar{v})}{\partial y} +
  \frac{\partial (\widebar{w}\widebar{w})}{\partial z}
  &
  =
  \overline{f_{z}} -
  \frac{1}{\rho}\frac{\partial \overline{p}}{\partial z} +
  \nu_{t}
  \left(
    \frac{\partial^{2} \overline{w}}{\partial x^{2}} +
    \frac{\partial^{2} \overline{w}}{\partial y^{2}} +
    \frac{\partial^{2} \overline{w}}{\partial z^{2}}
  \right) \label{EqRaeMez}
  \end{align}
\end{subequations}


\section{平面二维浅水方程}
天然河道水流运动一般都属于三维流动,运动要素即沿程变化,又沿水深和河宽方向变化。
由于三维水流运动比较复杂,河流数值模拟常用的一种简化方法是将运动要素沿水深方向平
均,把三维问题转化为平面二维问题。本节基于一定条件将三维流动的雷诺时均运动微分方
程简化为平面二维浅水方程。
\subsection{浅水假设和水深平均积分法则}
\subsubsection{浅水假设}
在河道、湖泊或水库水流中,水平尺度一般远大于垂向尺度。如果垂向加速度与重力加速度
相比很小,则可以忽略垂向加速度,流速等水力参数沿垂向的变化常采用其垂向平均值,并
假定沿水深方向的动水压强分布符合静水压强分布。三维流动的雷诺时均微分方程式
\eqref{EqRaeC}-\eqref{EqRaeMez}可简化为:
\begin{subequations}
  \begin{align}
    \frac{\partial \overline{u}}{\partial x} +
    \frac{\partial \overline{v}}{\partial y} +
    \frac{\partial \overline{w}}{\partial z}
  &
  =
  0
  \label{EqRaeCSimp}
  \\
  \frac{\partial \overline{u}}{\partial t} +
  \frac{\partial (\widebar{u}\widebar{u})}{\partial x} +
  \frac{\partial (\widebar{u}\widebar{v})}{\partial y} +
  \frac{\partial (\widebar{u}\widebar{w})}{\partial z}
  &
  =
  -\frac{1}{\rho}\frac{\partial \overline{p}}{\partial x} +
  \nu_{t}
  \left(
    \frac{\partial^{2} \overline{u}}{\partial x^{2}} +
    \frac{\partial^{2} \overline{u}}{\partial y^{2}} +
    \frac{\partial^{2} \overline{u}}{\partial z^{2}}
  \right)
  \label{EqRaeMexSimp}
  \\
  \frac{\partial \overline{v}}{\partial t} +
  \frac{\partial (\overline{v}\overline{u})}{\partial x} +
  \frac{\partial (\overline{v}\overline{v})}{\partial y} +
  \frac{\partial (\overline{v}\overline{w})}{\partial z}
  &
  =
  -\frac{1}{\rho}\frac{\partial \overline{p}}{\partial y} +
  \nu_{t}
  \left(
    \frac{\partial^{2} \overline{v}}{\partial x^{2}} +
    \frac{\partial^{2} \overline{v}}{\partial y^{2}} +
    \frac{\partial^{2} \overline{v}}{\partial z^{2}}
  \right)
  \label{EqRaeMeySimp}
  \\
  \frac{\partial \overline{p}}{\partial z}
  =
  -\rho g
  \label{EqRaeMezSimp}
  \end{align}
  \label{EqRae}
\end{subequations}

\subsubsection{水深积分平均法则}
将式\eqref{EqRae}沿水深积分平均,即可得到沿水深平均的平面二维流动的基本方程。在沿
水深积分平均过程中,采用以下定义和公式:

(1)定义水深为
\begin{equation}
  H =  z_{s} - z_{b}
\end{equation}
式中,$H=H(x,y,t)$为水深,$ z_{s}= z_{s}(x,y,t)$、$z_{b}=z_{b}(x,y,t)$分别为某一基准面下的
水面高程和河床高程(见图)

\begin{figure}
  \begin{tikzpicture}
  \end{tikzpicture}
  \caption{水位基准示意图}
\end{figure}

(2)定义沿水深平均流速$U$和$V$
\begin{equation}
  U
  =
  \frac{1}{H}
  \int_{z_{b}}^{ z_{s}}\!\overline{u}\mathrm{d}z
  \quad
  \quad
  V
  =
  \frac{1}{H}
  \int_{z_{b}}^{ z_{s}}\!\overline{v}\mathrm{d}z
\end{equation}
式中,下标$i$取1,2和3分别对应$x$,$y$和$z$方向的速度分量。

(3)莱布尼兹公式
\begin{equation}
  \frac{\partial}{\partial x_{i}}
  \int_{a}^{b}\!f\mathrm{d}z
  =
  \int_{a}^{b}\!
  \frac{\partial f}{\partial x_{i}}\mathrm{d}z
  +
  \left.
    f
    \right|_{b}\frac{\partial b}{\partial x_{i}}
    -
    \left.f\right|_{a}\frac{\partial a}{\partial x_{i}}
      \label{EqCGeLeibnizLaw}
    \end{equation}
    式中,$x_{i}=x$,$y$,$t$。$a$、$b$和$f$都是$x_{i}$的函数。

    (4)自由表面及河床底部运动学条件为:
    \begin{equation}
      \left.\overline{w}\right|_{z_{s}}
        =
        \frac{D z_{s}}{Dt}
        =
        \frac{\partial z_{s}}{\partial t}
        +
        \left.\overline{u}\right|_{z_{s}}\frac{\partial z_{s}}{\partial x}
          +
          \left.\overline{v}\right|_{z_{s}}\frac{\partial z_{s}}{\partial y}
            \label{EqCGeSMC}
          \end{equation}
          \begin{equation}
            \left.\overline{w}\right|_{z_{b}}
              =
              \frac{Dz_{b}}{Dt}
              =
              \frac{\partial z_{b}}{\partial t}
              +
              \left.\overline{u}\right|_{z_{b}}\frac{\partial z_{b}}{\partial x}
                +
                \left.\overline{v}\right|_{z_{b}}\frac{\partial z_{b}}{\partial y}
                  \label{EqCGeBMC}
                \end{equation}

                \subsection{沿水深平均的连续性方程}
                采用上述定义和公式对连续性方程\eqref{EqRaeCSimp}沿水深积分平均得:
                \begin{equation}
                  \int_{z_{b}}^{ z_{s}}\!
                  \left(
                    \frac{\partial \overline{u}}{\partial x} +
                    \frac{\partial \overline{v}}{\partial y} +
                    \frac{\partial \overline{w}}{\partial z}
                  \right)
                  \mathrm{d}z
                  =
                  0
                  \label{EqCGeRaeCE_Int}
                \end{equation}
                根据式\eqref{EqCGeLeibnizLaw},式\eqref{EqCGeRaeCE_Int}中前两项分别可写成
                \begin{equation}
                  \int_{z_{b}}^{ z_{s}}\!
                  \frac{\partial \overline{u}}{\partial x}
                  \mathrm{d}z
                  =
                  \frac{\partial}{\partial x}
                  \int_{z_{b}}^{ z_{s}}\!
                  \overline{u}
                  \mathrm{d}z
                  -
                  \left.\overline{u}\right|_{z_{s}}
                    \frac{\partial z_{s}}{\partial z}
                    +
                    \left.\overline{u}\right|_{z_{b}}
                      \frac{\partial z_{b}}{\partial z}
                      \label{EqCGeRaeCE_Int_x}
                    \end{equation}
                    \begin{equation}
                      \int_{z_{b}}^{ z_{s}}\!
                      \frac{\partial \overline{v}}{\partial y}
                      \mathrm{d}z
                      =
                      \frac{\partial}{\partial y}
                      \int_{z_{b}}^{ z_{s}}\!
                      \overline{v}
                      \mathrm{d}z
                      -
                      \left.\overline{v}\right|_{z_{s}}
                        \frac{\partial z_{s}}{\partial z}
                        +
                        \left.\overline{v}\right|_{z_{b}}
                          \frac{\partial z_{b}}{\partial z}
                          \label{EqCGeRaeCE_Int_y}
                        \end{equation}
                        式\eqref{EqCGeRaeCE_Int}中最后一项
                        \begin{equation}
                          \int_{z_{b}}^{ z_{s}}\!
                          \frac{\partial \overline{w}}{\partial z}
                          \mathrm{d}z
                          =
                          \left.\overline{w}\right|_{z_{s}}
                            -
                            \left.\overline{w}\right|_{z_{b}}
                              \label{EqCGeRaeCE_Int_z}
                            \end{equation}
                            将式\eqref{EqCGeRaeCE_Int_x},\eqref{EqCGeRaeCE_Int_y}和\eqref{EqCGeRaeCE_Int_z}
                            代入\eqref{EqCGeRaeCE_Int},并利用自由表面及河床底部运动学条件式\eqref{EqCGeSMC}
                            和\eqref{EqCGeBMC},可得
                            \begin{equation*}
                              \begin{aligned}
  &\int_{z_{b}}^{ z_{s}}\!
  \left(
    \frac{\partial \overline{u}}{\partial x} +
    \frac{\partial \overline{v}}{\partial y} +
    \frac{\partial \overline{w}}{\partial z}
  \right)
  \mathrm{d}z
  \\
                                =&
                                \frac{\partial}{\partial x}
                                \int_{z_{b}}^{ z_{s}}\!
                                \overline{u}
                                \mathrm{d}z
                                -
                                \left.\overline{u}\right|_{z_{s}}
                                  \frac{\partial z_{s}}{\partial z}
                                  +
                                  \left.\overline{u}\right|_{z_{b}}
                                    \frac{\partial z_{b}}{\partial z}
                                    + \\
                                 &
                                 \frac{\partial}{\partial y}
                                 \int_{z_{b}}^{ z_{s}}\!
                                 \overline{v}
                                 \mathrm{d}z
                                 -
                                 \left.\overline{v}\right|_{z_{s}}
                                   \frac{\partial z_{s}}{\partial z}
                                   +
                                   \left.\overline{v}\right|_{z_{b}}
                                     \frac{\partial z_{b}}{\partial z}
                                     + \\
                                 &
                                 \left.\overline{w}\right|_{z_{s}}
                                   -
                                   \left.\overline{w}\right|_{z_{b}}
                                     \\
                                =&
                                \frac{\partial HU}{\partial x} +
                                \frac{\partial HV}{\partial x} +
                                \frac{\partial  z_{s}}{\partial t} -
                                \frac{\partial z_{b}}{\partial t}
                                =
                                0
                              \end{aligned}
                            \end{equation*}
                            最后得
                            \begin{equation}
                              \frac{\partial H}{\partial t} +
                              \frac{\partial HU}{\partial x} +
                              \frac{\partial HV}{\partial y}
                              =
                              0
                            \end{equation}
                            也可以写成
                            \begin{equation}
                              \frac{\partial H}{\partial t} +
                              \nabla\cdot(H\mathbf{U})
                              =
                              0
                            \end{equation}

                            \subsection{沿水深平均的运动方程}
                            以$x$方向为例,式\eqref{EqRaeMexSimp}沿水深积分为
                            \begin{equation}
                              \int_{z_{b}}^{ z_{s}}\!
                              \left[
                                \frac{\partial \overline{u}}{\partial t} +
                                \frac{\partial (\widebar{u}\widebar{u})}{\partial x} +
                                \frac{\partial (\widebar{u}\widebar{v})}{\partial y} +
                                \frac{\partial (\widebar{u}\widebar{w})}{\partial z} +
                                \frac{1}{\rho}\frac{\partial \overline{p}}{\partial x} -
                                \nu_{t}
                                \left(
                                  \frac{\partial^{2} \overline{u}}{\partial x^{2}} +
                                  \frac{\partial^{2} \overline{u}}{\partial y^{2}} +
                                  \frac{\partial^{2} \overline{u}}{\partial z^{2}}
                                \right)
                              \right]
                              \mathrm{d}z
                              =
                              0
                              \label{EqCGeMex_Int}
                            \end{equation}
                            式\eqref{EqCGeMex_Int}中包含了非恒定流项积分、对流项积分、压力项积分和阻力项积分
                            。接下来分项讨论。

                            (1)非恒定流项积分
                            \begin{equation}
                              \begin{aligned}
    &
    \int_{z_0}^{ z_{s}}\!
    \frac{\partial \overline{u}}{\partial t}
    \mathrm{d}z
    =
    \frac{\partial}{\partial t}
    \int_{z_0}^{ z_{s}}\!
    \overline{u}
    \mathrm{d}z
    -
    \left.\overline{u}\right|_{z_{s}}\frac{\partial  z_{s}}{\partial t}
      +
      \left.\overline{u}\right|_{z_{b}}\frac{\partial z_{b}}{\partial t}
        \\
                                =&
                                \frac{\partial HU}{\partial t}
                                -
                                \left.\overline{u}\right|_{z_{s}}\frac{\partial  z_{s}}{\partial t}
                                  +
                                  \left.\overline{u}\right|_{z_{b}}\frac{\partial z_{b}}{\partial t}
                              \end{aligned}
                              \label{EqCGe_Mex_US_Int}
                            \end{equation}

                            (2)对流项积分

                            首先将时均流速按照式\eqref{EqCGe_VelDecompse}进行分解
                            \begin{equation}
                              \begin{aligned}
                                \overline{u}
                                =
                                u+\Delta\overline{u}
                                \\
                                \overline{v}
                                =
                                v+\Delta\overline{v}
                              \end{aligned}
                              \label{EqCGe_VelDecompse}
                            \end{equation}
                            式中,$\Delta\overline{u}$和$\Delta\overline{v}$分别为$x$和$y$的时均流速与
                            垂线平均流速的差值。

                            对流项中第一项的积分
                            \begin{equation}
                              \int_{z_0}^{ z_{s}}\!
                              \frac{\partial \widebar{u}\widebar{u}}{\partial x}
                              \mathrm{d}z
                              =
                              \frac{\partial}{\partial x}
                              \int_{z_0}^{ z_{s}}\!
                              \widebar{u}\widebar{u}
                              \mathrm{d}z
                              -
                              \left.\widebar{u}\widebar{u}\right|_{z_{s}}\frac{\partial  z_{s}}{\partial x}
                                +
                                \left.\widebar{u}\widebar{u}\right|_{z_{b}}\frac{\partial z_{b}}{\partial x}
                                  \label{EqCGe_CT_1st}
                                \end{equation}
                                式中
                                \begin{equation*}
                                  \begin{aligned}
                                    \int_{z_0}^{ z_{s}}\!
                                    \widebar{u}\widebar{u}
                                    \mathrm{d}z
    &=
    \int_{z_0}^{ z_{s}}\!
    (u+\Delta\overline{u})(u+\Delta\overline{u})
    \mathrm{d}z
    \\
    &=
    \int_{z_0}^{ z_{s}}\!
    (uu+2u\Delta\overline{u}+\Delta\overline{u}\Delta\overline{u})
    \mathrm{d}z
    \\
    &=
    HUU
    +
    \int_{z_0}^{ z_{s}}\!
    \Delta\overline{u}\Delta\overline{u}
    \mathrm{d}z
    \\
    &=
    \beta_{xx}HUU
                                  \end{aligned}
                                \end{equation*}
                                其中
                                \begin{equation}
                                  \beta_{xx}
                                  =
                                  1 +
                                  \frac
                                  {
                                    \int_{z_0}^{ z_{s}}\!
                                    \Delta\overline{U}\Delta\overline{U}
                                    \mathrm{d}z
                                  }
                                  {Huu}
                                \end{equation}
                                是由于流速沿垂线分布不均匀而引入的修正系数,类似于水力学中的动量修正系数。
                                $\beta_{xx}$的取值一般在1.02与1.05之间,可近似取为1.0。因此
                                \begin{equation}
                                  \int_{z_0}^{ z_{s}}\!
                                  \frac{\partial \widebar{u}\widebar{u}}{\partial x}
                                  \mathrm{d}z
                                  =
                                  \frac{\partial HUU}{\partial x}
                                  -
                                  \left.\widebar{u}\widebar{u}\right|_{z_{s}}\frac{\partial  z_{s}}{\partial x}
                                    +
                                    \left.\widebar{u}\widebar{u}\right|_{z_{b}}\frac{\partial z_{b}}{\partial x}
                                    \end{equation}

                                    同理,对流项的第二项可写为
                                    \begin{equation}
                                      \int_{z_0}^{ z_{s}}\!
                                      \frac{\partial \widebar{u}\widebar{v}}{\partial y}
                                      \mathrm{d}z
                                      =
                                      \frac{\partial HUV}{\partial x}
                                      -
                                      \left.\widebar{u}\widebar{v}\right|_{z_{s}}\frac{\partial  z_{s}}{\partial y}
                                        +
                                        \left.\widebar{u}\widebar{v}\right|_{z_{b}}\frac{\partial z_{b}}{\partial y}
                                        \end{equation}
                                        对流项的第三项
                                        \begin{equation}
                                          \int_{z_0}^{ z_{s}}\!
                                          \frac{\partial \widebar{u}\widebar{w}}{\partial z}
                                          \mathrm{d}z
                                          =
                                          \left.\widebar{u}\widebar{w}\right|_{z_{s}}
                                            -
                                            \left.\widebar{u}\widebar{w}\right|_{z_{b}}
                                            \end{equation}

                                            将非恒定流项和对流项积分相加,并利用自由表面和河床底部运动学条件可得:
                                            \begin{equation}
                                              \int_{z_{b}}^{ z_{s}}\!
                                              \left[
                                                \frac{\partial \overline{u}}{\partial t} +
                                                \frac{\partial (\widebar{u}\widebar{u})}{\partial x} +
                                                \frac{\partial (\widebar{u}\widebar{v})}{\partial y} +
                                                \frac{\partial (\widebar{u}\widebar{w})}{\partial z}
                                              \right]
                                              \mathrm{d}z
                                              =
                                              \frac{\partial HU}{\partial t}
                                              +
                                              \frac{\partial HUU}{\partial x}
                                              +
                                              \frac{\partial HUV}{\partial y}
                                            \end{equation}

                                            (3)压力项积分
                                            \begin{equation}
                                              \begin{aligned}
                                                \int_{z_{b}}^{ z_{s}}\!
                                                \frac{\partial \overline{p}}{\partial x}
                                                \mathrm{d}z
    &=
    \frac{\partial}{\partial x}
    \int_{z_0}^{ z_{s}}\!
    \overline{p}
    \mathrm{d}z
    -
    \left.\overline{p}\right|_{z_{s}}\frac{\partial  z_{s}}{\partial x}
      +
      \left.\overline{p}\right|_{z_{b}}\frac{\partial z_{b}}{\partial x}
        \\
    &=
    \frac{\partial}{\partial x}
    \int_{z_0}^{ z_{s}}\!
    \rho g( z_{s}-z)
    \mathrm{d}z
    -
    \left.\rho g( z_{s}-z)\right|_{z_{s}}\frac{\partial  z_{s}}{\partial x}
      +
      \left.\rho g( z_{s}-z)\right|_{z_{b}}\frac{\partial z_{b}}{\partial x}
        \\
    &=
    \rho gH\frac{\partial H}{\partial x}
    +
    \rho gH\frac{\partial z_{b}}{\partial x}
    =
    \rho gH\frac{\partial  z_{s}}{\partial x}
                                              \end{aligned}
                                            \end{equation}

                                            (4)阻力项积分
                                            \begin{equation}
                                              \begin{aligned}
    &\int_{z_{b}}^{ z_{s}}\!
    \frac{\partial^{2} \overline{u}}{\partial x^{2}}
    \mathrm{d}z
    =
    \int_{z_{b}}^{ z_{s}}\!
    \frac{\partial}{\partial x}
    \left(\frac{\partial \overline{u}}{\partial x}\right)
    \mathrm{d}z
    =
    \frac{\partial}{\partial x}
    \int_{z_0}^{ z_{s}}\!
    \frac{\partial \overline{u}}{\partial x}
    \mathrm{d}z
    -
    \left.\frac{\partial \overline{u}}{\partial x}\right|_{z_{s}}\frac{\partial  z_{s}}{\partial x}
      +
      \left.\frac{\partial \overline{u}}{\partial x}\right|_{z_{b}}\frac{\partial z_{b}}{\partial x}
        \\
                                                =&
                                                \frac{\partial}{\partial x}
                                                \left(
                                                  \frac{\partial}{\partial x}
                                                  \int_{z_0}^{ z_{s}}\!
                                                  \overline{u}
                                                  \mathrm{d}z
                                                  -
                                                  \left.\overline{u}\right|_{z_{s}}\frac{\partial  z_{s}}{\partial x}
                                                    +
                                                    \left.\overline{u}\right|_{z_{b}}\frac{\partial z_{b}}{\partial x}
                                                    \right)
                                                    -
                                                    \left.\frac{\partial \overline{u}}{\partial x}\right|_{z_{s}}\frac{\partial  z_{s}}{\partial x}
                                                      +
                                                      \left.\frac{\partial \overline{u}}{\partial x}\right|_{z_{b}}\frac{\partial z_{b}}{\partial x}
                                                        \\
                                                =&
                                                \frac{\partial^{2} HU}{\partial x^{2}} +
                                                \frac{\partial}{\partial x}
                                                \left(
                                                  -
                                                  \left.\overline{u}\right|_{z_{s}}\frac{\partial  z_{s}}{\partial x}
                                                    +
                                                    \left.\overline{u}\right|_{z_{b}}\frac{\partial z_{b}}{\partial x}
                                                    \right)
                                                    -
                                                    \left.\frac{\partial \overline{u}}{\partial x}\right|_{z_{s}}\frac{\partial  z_{s}}{\partial x}
                                                      +
                                                      \left.\frac{\partial \overline{u}}{\partial x}\right|_{z_{b}}\frac{\partial z_{b}}{\partial x}
                                              \end{aligned}
                                              \label{EqCGe_Shear_1}
                                            \end{equation}

                                            类似地
                                            \begin{equation}
                                              \int_{z_{b}}^{ z_{s}}\!
                                              \frac{\partial^{2} \overline{u}}{\partial y^{2}}
                                              \mathrm{d}z
                                              =
                                              \frac{\partial^{2} HU}{\partial y^{2}} +
                                              \frac{\partial}{\partial y}
                                              \left(
                                                -
                                                \left.\overline{u}\right|_{z_{s}}\frac{\partial  z_{s}}{\partial y}
                                                  +
                                                  \left.\overline{u}\right|_{z_{b}}\frac{\partial z_{b}}{\partial y}
                                                  \right)
                                                  -
                                                  \left.\frac{\partial \overline{u}}{\partial y}\right|_{z_{s}}\frac{\partial  z_{s}}{\partial y}
                                                    +
                                                    \left.\frac{\partial \overline{u}}{\partial y}\right|_{z_{b}}\frac{\partial z_{b}}{\partial y}
                                                      \label{EqCGe_Shear_2}
                                                    \end{equation}
                                                    \begin{equation}
                                                      \int_{z_{b}}^{ z_{s}}\!
                                                      \frac{\partial^{2} \overline{u}}{\partial y^{2}}
                                                      \mathrm{d}z
                                                      =
                                                      \left.\frac{\partial \overline{u}}{\partial z}\right|_{z_{s}}
                                                        -
                                                        \left.\frac{\partial \overline{u}}{\partial z}\right|_{z_{b}}
                                                          \label{EqCGe_Shear_3}
                                                        \end{equation}
                                                        将式\eqref{EqCGe_Shear_1}至\eqref{EqCGe_Shear_3}相加,得
                                                        \begin{equation}
                                                          \begin{aligned}
    &\int_{z_{b}}^{ z_{s}}\!
    \nu_{t}
    \left(
      \frac{\partial^{2} \overline{u}}{\partial x^{2}} +
      \frac{\partial^{2} \overline{u}}{\partial y^{2}} +
      \frac{\partial^{2} \overline{u}}{\partial z^{2}}
    \right)
    \mathrm{d}z =\\
    &
    \nu_{t}
    \left(
      \frac{\partial^{2} HU}{\partial x^{2}} +
      \frac{\partial^{2} HU}{\partial y^{2}}
    \right)
    \\
                                                            -&\nu_{t}
                                                            \left[
                                                              \frac{\partial}{\partial x}
                                                              \left(
                                                                \frac{\partial  z_{s}}{\partial x}
                                                                \left.\overline{u}\right|_{z_{s}}
                                                                \right)
                                                                +
                                                                \frac{\partial}{\partial y}
                                                                \left(
                                                                  \frac{\partial  z_{s}}{\partial y}
                                                                  \left.\overline{u}\right|_{z_{s}}
                                                                  \right)
                                                                  +
                                                                  \left.\frac{\partial \overline{u}}{\partial x}\right|_{z_{s}}
                                                                    \frac{\partial  z_{s}}{\partial x}
                                                                    +
                                                                    \left.\frac{\partial \overline{u}}{\partial y}\right|_{z_{s}}
                                                                      \frac{\partial  z_{s}}{\partial y}
                                                                      -
                                                                      \left.\frac{\partial \overline{u}}{\partial z}\right|_{z_{s}}
                                                                      \right]
                                                                      \\
                                                            +&\nu_{t}
                                                            \left[
                                                              \frac{\partial}{\partial x}
                                                              \left(
                                                                \frac{\partial z_{b}}{\partial x}
                                                                \left.\overline{u}\right|_{z_{b}}
                                                                \right)
                                                                +
                                                                \frac{\partial}{\partial y}
                                                                \left(
                                                                  \frac{\partial z_{b}}{\partial y}
                                                                  \left.\overline{u}\right|_{z_{b}}
                                                                  \right)
                                                                  +
                                                                  \left.\frac{\partial \overline{u}}{\partial x}\right|_{z_{b}}
                                                                    \frac{\partial z_{b}}{\partial x}
                                                                    +
                                                                    \left.\frac{\partial \overline{u}}{\partial y}\right|_{z_{b}}
                                                                      \frac{\partial z_{b}}{\partial y}
                                                                      -
                                                                      \left.\frac{\partial \overline{u}}{\partial z}\right|_{z_{b}}
                                                                      \right]
                                                          \end{aligned}
                                                          \label{EqCGe_Shear_Total}
                                                        \end{equation}
                                                        式\eqref{EqCGe_Shear_Total}中右边后两项分别为由底部床面阻力和自由表面风阻力引起
                                                        的阻力项,通常可以式\eqref{EqCGe_Shear}表示:
                                                        \begin{equation}
                                                          g\frac{n^{2}U\sqrt{U^{2}+V^{2}}}{H^{1/3}}
                                                          -
                                                          C_{w}\frac{\rho_{a}}{\rho}\omega^{2}\cos\beta
                                                          \label{EqCGe_Shear}
                                                        \end{equation}
                                                        式中,$C_{w}$为无因此风应力系数,$\rho_{a}$为空气密度,$\omega$为风速,$\beta$为
                                                        风向与$x$方向的夹角。最后,$x$方向的运动方程为
                                                        \begin{equation}
                                                          \begin{aligned}
                                                            \frac{\partial HU}{\partial t} +
  &\frac{\partial HUU}{\partial x} +
  \frac{\partial HUV}{\partial y}
  =\\
  &-gH\frac{\partial  z_{s}}{\partial x}
  -g\frac{n^{2}U\sqrt{U^{2}+V^{2}}}{H^{1/3}}
  +
  \nu_{t}\left(
    \frac{\partial^{2}HU}{\partial x^{2}}+
    \frac{\partial^{2}HU}{\partial y^{2}}
  \right)
  +C_{w}\frac{\rho_{a}}{\rho}\omega^{2}\cos\beta
                                                          \end{aligned}
                                                          \label{EqCGe_Me_x}
                                                        \end{equation}
                                                        同理可得,$y$方向运动方程为
                                                        \begin{equation}
                                                          \begin{aligned}
                                                            \frac{\partial HV}{\partial t} +
  &\frac{\partial HUV}{\partial x} +
  \frac{\partial HVV}{\partial y}
  =\\
  &-gH\frac{\partial  z_{s}}{\partial y}
  -g\frac{n^{2}V\sqrt{U^{2}+V^{2}}}{H^{1/3}}
  +
  \nu_{t}\left(
    \frac{\partial^{2}HV}{\partial x^{2}}+
    \frac{\partial^{2}HV}{\partial y^{2}}
  \right)
  +C_{w}\frac{\rho_{a}}{\rho}\omega^{2}\sin\beta
                                                          \end{aligned}
                                                          \label{EqCGe_Me_y}
                                                        \end{equation}

                                                        当自由表面风应力影响较小时,风应力项可以忽略。此外,当模拟区域尺度较大,还要考虑
                                                        地球自转的影响,可在方程\eqref{EqCGe_Me_x}和\eqref{EqCGe_Me_y}右边分别加入科氏力。
                                                        \begin{equation}
                                                          \begin{aligned}
                                                            f_x = 2\omega\sin\varphi U
                                                            \\
                                                            f_y = 2\omega\sin\varphi V
                                                          \end{aligned}
                                                        \end{equation}
                                                        式中,$\omega$为地球自转角速度,$\varphi$为模拟区域所处纬度。

                                                        根据推导过程中所采用的假定条件,在使用二维浅水方程时应注意以下问题:
                                                        \begin{enumerate}
                                                          \item 方程推导中引用了牛顿流体所满足的本构关系式,因此上述方程只适用于牛顿流体
                                                            ,对类似高含沙水流的非牛顿流体不适用。
                                                          \item 方程推导中对流体做了均质不可压的假设,因此上述方程只能在含沙量较小的情况下近似使
                                                            用,当含沙量较大时,应考虑密度变化的影响。
                                                          \item 在垂向积分过程中,略去流速等水力参数沿垂直方向的变化,并假定沿水深方向的动水压强
                                                            分布符合静水压强分布。因此所研究问题的水平尺度应远大于垂向尺度,流速等水力参数沿
                                                            垂直方向的变化较之沿水平方向的变化要小得多。
                                                        \end{enumerate}

                                                        \subsection{二维平面浅水方程不同形式}
                                                        忽略风应力和科氏力的二维平面浅水方程为
                                                        \begin{equation}
                                                          \begin{gathered}
                                                            \frac{\partial H}{\partial t} +
                                                            \frac{\partial HU}{\partial x} +
                                                            \frac{\partial HV}{\partial y}
                                                            =
                                                            0
                                                            \\
                                                            \frac{\partial HU}{\partial t} +
                                                            \frac{\partial HUU}{\partial x} +
                                                            \frac{\partial HUV}{\partial y}
                                                            =
                                                            -gH\frac{\partial  z_{s}}{\partial x}
                                                            -g\frac{n^{2}U\sqrt{U^{2}+V^{2}}}{H^{1/3}}
                                                            +
                                                            \nu_{t}\left(
                                                              \frac{\partial^{2}HU}{\partial x^{2}}+
                                                              \frac{\partial^{2}HU}{\partial y^{2}}
                                                            \right)
                                                            \\
                                                            \frac{\partial HV}{\partial t} +
                                                            \frac{\partial HUV}{\partial x} +
                                                            \frac{\partial HVV}{\partial y}
                                                            =
                                                            -gH\frac{\partial  z_{s}}{\partial y}
                                                            -g\frac{n^{2}V\sqrt{U^{2}+V^{2}}}{H^{1/3}}
                                                            +
                                                            \nu_{t}\left(
                                                              \frac{\partial^{2}HV}{\partial x^{2}}+
                                                              \frac{\partial^{2}HV}{\partial y^{2}}
                                                            \right)
                                                          \end{gathered}
                                                        \end{equation}

                                                        当床面阻力占主导作用时,还可进一步忽略紊动阻力项。
                                                        \begin{equation}
                                                          \begin{gathered}
                                                            \frac{\partial H}{\partial t} +
                                                            \frac{\partial HU}{\partial x} +
                                                            \frac{\partial HV}{\partial y}
                                                            =
                                                            0
                                                            \\
                                                            \frac{\partial HU}{\partial t} +
                                                            \frac{\partial (HUU+gH^{2}/2)}{\partial x} +
                                                            \frac{\partial HUV}{\partial y}
                                                            =
                                                            -gH\frac{\partial z_{b}}{\partial x}
                                                            -g\frac{n^{2}U\sqrt{U^{2}+V^{2}}}{H^{1/3}}
                                                            \\
                                                            \frac{\partial HV}{\partial t} +
                                                            \frac{\partial HUV}{\partial x} +
                                                            \frac{\partial (HVV+gH^{2}/2)}{\partial y}
                                                            =
                                                            -gH\frac{\partial z_{b}}{\partial y}
                                                            -g\frac{n^{2}V\sqrt{U^{2}+V^{2}}}{H^{1/3}}
                                                          \end{gathered}
                                                        \end{equation}

                                                        此外,为了书写的简化,用$h$替代$H$,$u$替代$U$,$v$替代$V$,定义$x$和$y$
                                                        方向的流量分别为$q_{x}=hu$和$q_{y}=hv$,上式可写成
                                                        \begin{equation}
                                                          \begin{gathered}
                                                            \frac{\partial h}{\partial t} +
                                                            \frac{\partial q_{x}}{\partial x} +
                                                            \frac{\partial q_{y}}{\partial y}
                                                            =
                                                            0
                                                            \\
                                                            \frac{\partial q_{x}}{\partial t} +
                                                            \frac{\partial (uq_{x}+gh^{2}/2)}{\partial x} +
                                                            \frac{\partial vq_{y}}{\partial y}
                                                            =
                                                            -gh\frac{\partial z_{b}}{\partial x}
                                                            -g\frac{n^{2}u\sqrt{u^{2}+v^{2}}}{h^{1/3}}
                                                            \\
                                                            \frac{\partial q_{y}}{\partial t} +
                                                            \frac{\partial uq_{y}}{\partial x} +
                                                            \frac{\partial (vq_{y}+gh^{2}/2)}{\partial y}
                                                            =
                                                            -gh\frac{\partial z_{b}}{\partial y}
                                                            -g\frac{n^{2}v\sqrt{u^{2}+v^{2}}}{h^{1/3}}
                                                          \end{gathered}
                                                        \end{equation}
                                                        上式的向量形式为:
                                                        \begin{equation}
                                                          \frac{\partial \mathbf{q}}{\partial t} +
                                                          \frac{\partial \mathbf{f}}{\partial x} +
                                                          \frac{\partial \mathbf{g}}{\partial y}
                                                          =
                                                          \mathbf{S_{b}} + \mathbf{S_{f}}
                                                        \end{equation}
                                                        %或
                                                        %\begin{equation}
                                                        %\frac{\partial \mathbf{q}}{\partial t} +
                                                        %\nabla\cdot(\mathbf{F}) =
                                                        %\mathbf{S}
                                                        %\end{equation}
                                                        其中,$\mathbf{q}$为守恒变量向量,$\mathbf{f}$和$\mathbf{g}$分别为$x$和$y$方向的
                                                        通量向量,$\mathbf{S_{b}}$和$\mathbf{S_{f}}$分别为河床底坡和阻力源项向量。这些向
                                                        量的具体元素为:
                                                        \begin{equation}
                                                          \mathbf{q} =
                                                          \begin{bmatrix}
                                                            h \\
                                                            hu \\
                                                            hv \\
                                                          \end{bmatrix}
                                                          ,
                                                          \mathbf{f} =
                                                          \begin{bmatrix}
                                                            hu \\
                                                            hu^{2} + \frac{1}{2}gh^{2} \\
                                                            huv \\
                                                          \end{bmatrix}
                                                          ,
                                                          \mathbf{g} =
                                                          \begin{bmatrix}
                                                            hv \\
                                                            huv \\
                                                            hu^{2} + \frac{1}{2}gh^{2} \\
                                                          \end{bmatrix}
                                                          ,
                                                          \mathbf{S_{b}} =
                                                          \begin{bmatrix}
                                                            0 \\
                                                            -gh\frac{\partial z_{b}}{\partial x} \\
                                                            -gh\frac{\partial z_{b}}{\partial y} \\
                                                          \end{bmatrix}
                                                          ,
                                                          \mathbf{S_{f}} =
                                                          \begin{bmatrix}
                                                            0 \\
                                                            -g\frac{n^{2}u\sqrt{u^2+v^2}}{h^{1/3}} \\
                                                            -g\frac{n^{2}v\sqrt{u^2+v^2}}{h^{1/3}} \\
                                                          \end{bmatrix}
                                                        \end{equation}


                                                        \section{一维非恒定流基本控制方程}
                                                        \subsection{一维连续性方程}
                                                        如图\ref{FigCGe_}所示,在明槽非恒定流中,沿水流流动方向取长为$\mathrm{d}x$的微小
                                                        流段。流段进口1-1断面流量为$Q$,在$\Delta t$时段内,从1-1断面进入流段的液体
                                                        质量为$\rho Q\Delta t$。流段出口2-2断面流量为$Q+\frac{\partial Q}{\partial
                                                        x}\mathrm{d}x$,在$\Delta t$时段内,从2-2断面流出的液体质量为$\rho Q\Delta
                                                        t+\rho\frac{\partial Q}{\partial x}\mathrm{d}x\Delta t$。该时段内进出此流段的液体
                                                        质量差为$-\rho\frac{\partial Q}{\partial x}\mathrm{d}x\Delta t$。

                                                        时段内的质量差表现为流段内的槽蓄量变化。在起始时刻,流段内的槽蓄量为
                                                        $\rho\overline{A}\mathrm{d}x$。而经过$\Delta t$时段后,流段内的槽蓄量为
                                                        $\rho\left(\overline{A}+\frac{\partial\overline{A}}{\partial t}\Delta
                                                        t\right)\mathrm{d}x$。$\Delta t$时段内,流段内的槽蓄量变化量为
                                                        $\rho\frac{\partial \overline{A}}{\partial t}\Delta t\mathrm{d}x$。其中
                                                        $\overline{A}$为微小流段的平均过水面积。当流段内过水断面面积变化较小时,可直接用
                                                        1-1断面段面积$A$来替代$\overline{A}$。

                                                        因此,根据质量守恒原理,进出该流段的液体质量差等于流段内槽蓄量改变量,即
                                                        \begin{equation*}
                                                          -\rho\frac{\partial Q}{\partial x}\mathrm{d}x\Delta t
                                                          =
                                                          \rho\frac{\partial A}{\partial t}\mathrm{d}x\Delta t
                                                        \end{equation*}
                                                        化简后得到明槽一维非恒定流连续性方程
                                                        \begin{equation}
                                                          \frac{\partial A}{\partial t}
                                                          +
                                                          \frac{\partial Q}{\partial x}
                                                          =
                                                          0
                                                          \label{EqCGe_SVe_Ce}
                                                        \end{equation}

                                                        如果在该流段内有旁侧入流或出流,且单位长度旁侧入流流量为$q$($q>0$为入流,$q<0$为出流
                                                        ),考虑旁侧入流得明槽一维非恒定流连续性方程为
                                                        \begin{equation}
                                                          \frac{\partial A}{\partial t}
                                                          +
                                                          \frac{\partial Q}{\partial x}
                                                          =
                                                          q
                                                        \end{equation}

                                                        \subsection{一维运动方程}
                                                        设坐标轴$x$方向与水流流动方向一致,根据牛顿第二定律建立运动方程。为分析简单起见
                                                        ,首先考虑棱柱体明槽(如图)的情况。

                                                        作用在1-1断面的所有外力在$x$方向的分力有:

                                                        (1)总动水压力。

                                                        设压强分布服从静水压强分布,则作用在1-1断面的水压力
                                                        \begin{equation}
                                                          P
                                                          =
                                                          \int_{0}^{h}\! \rho g(h-y)\xi(y)\mathrm{d}y
                                                        \end{equation}
                                                        式中,$\xi(y)$为过水断面上距渠底$y$处的宽度。
                                                        作用于2-2断面的水压力为$P+\frac{\partial P}{\partial x}\mathrm{d}x$。则沿$x$方向
                                                        的总动水压力
                                                        \begin{equation}
                                                          \begin{aligned}
                                                            \sum P
                                                            =&
                                                            P - \left(P+\frac {\partial P} {\partial x}\mathrm{d}x\right)
                                                            \\
                                                            =&
                                                            -\gamma \mathrm{d}x
                                                            \left[
                                                              \frac{\partial h}{\partial x}
                                                              \int_{0}^{h}\!
                                                              z_{s}(y)
                                                              \mathrm{d}y
                                                              +
                                                              \int_{0}^{h}\!
                                                              (h-y)
                                                              \frac{\partial  z_{s}(y)}{\partial x}
                                                              \mathrm{d}y
                                                            \right]
                                                          \end{aligned}
                                                          \label{EqCGe_SVe_Me_Pressure}
                                                        \end{equation}
                                                        因假定明槽为棱柱体明槽,有$\frac{\partial \xi(y)}{\partial x}=0$。则有
                                                        \begin{equation}
                                                          \sum P
                                                          =
                                                          -\gamma A\frac{\partial h}{\partial x}\mathrm{d}x
                                                        \end{equation}

                                                        (2)重力
                                                        \begin{equation}
                                                          \mathrm{d}G_{x}
                                                          =
                                                          \mathrm{d}G\sin\alpha
                                                          =
                                                          -\gamma A\mathrm{d}x\frac{\partial z}{\partial x}
                                                        \end{equation}
                                                        式中,$\alpha$为坐标轴$x$与水平方向的夹角,$A$为过水断面面积。

                                                        (3)侧壁面上的阻力
                                                        \begin{equation}
                                                          \mathrm{d}T
                                                          =
                                                          \tau_{0}\chi\mathrm{d}x
                                                          =
                                                          \gamma RJ\chi\mathrm{d}x
                                                          =
                                                          \gamma AJ\mathrm{d}x
                                                        \end{equation}
                                                        式中,$\chi$为过水断面湿周,$R$为过水断面水力半径,$J$为水力坡度,
                                                        $\tau_{0}=\gamma RJ$为侧壁表面平均切应力。

                                                        其次,由于流速$U$是$x$和$t$的函数,则水流沿$x$方向的加速度$a_{x}$为
                                                        \begin{equation}
                                                          a_{x}
                                                          =
                                                          \frac{\mathrm{d} U}{\mathrm{d} t}
                                                          =
                                                          \frac{\partial U}{\partial t}
                                                          +
                                                          U
                                                          \frac{\partial U}{\partial x}
                                                        \end{equation}
                                                        微小流段内的水体质量为$\mathrm{d}m=\rho A\mathrm{d}x$。

                                                        根据牛顿第二定律,有$\sum F_{x}=\mathrm{d}ma_{x}$,即
                                                        \begin{equation}
                                                          -\gamma A\frac{\partial h}{\partial x}\mathrm{d}x
                                                          -\gamma A\frac{\partial z}{\partial x}\mathrm{d}x
                                                          -\gamma AJ\mathrm{d}x
                                                          =
                                                          \rho A\mathrm{d}x
                                                          \left(
                                                            \frac{\partial U}{\partial t}
                                                            +
                                                            U
                                                            \frac{\partial U}{\partial x}
                                                          \right)
                                                        \end{equation}
                                                        上式两边同除以$\gamma A\mathrm{d}x$并整理得:
                                                        \begin{equation}
                                                          \frac{\partial z}{\partial x}
                                                          +
                                                          \frac{1}{g}
                                                          \frac{\partial U}{\partial t}
                                                          +
                                                          \frac{U}{g}
                                                          \frac{\partial U}{\partial x}
                                                          +
                                                          J
                                                          =
                                                          0
                                                          \label{EqCGe_SVe_Me}
                                                        \end{equation}
                                                        式\eqref{EqCGe_SVe_Me}即为棱柱体明槽非恒定流运动方程得一般形式。对于非棱柱体明槽
                                                        (比如河槽向下游缩窄或展宽),则两岸壁将对微小流段水体作用一附加压力,该附加压力
                                                        可表示为
                                                        \begin{equation}
                                                          Vp^{\prime}
                                                          =
                                                          \int_{0}^{h}\!
                                                          \left[
                                                            \rho g(h-y)
                                                            \frac{\partial \xi(y)}{\partial x}
                                                            \mathrm{d}x
                                                          \right]
                                                          \mathrm{d}y
                                                        \end{equation}
                                                        将附加压力代式\eqref{EqCGe_SVe_Me_Pressure}入中,恰好与该式最后一项抵消。因此对
                                                        于非棱柱体明槽,式\eqref{EqCGe_SVe_Me}仍适用。

                                                        \subsection{圣维南方程不同形式}
                                                        连续性方程\eqref{EqCGe_SVe_Ce}和运动方程\eqref{EqCGe_SVe_Me}构成了描述明槽非恒定
                                                        渐变流的圣维南方程组。在实际应用中,为了使用方便,常对式\eqref{EqCGe_SVe_Ce}和
                                                        \eqref{EqCGe_SVe_Me}进行改写,得到不同因变量组合的圣维南方程组。

                                                        % Todo: 列出下列各式的具体推导过程。

                                                        (1)以水位$z$和流量$Q$为因变量的圣维南方程组
                                                        \begin{equation}
                                                          \begin{gathered}
                                                            B\frac{\partial z}{\partial t}
                                                            +
                                                            \frac{\partial Q}{\partial x}
                                                            =
                                                            q
                                                            \\
                                                            \frac{\partial Q}{\partial t}
                                                            +
                                                            \frac{2Q}{A}\frac{\partial Q}{\partial x}
                                                            +
                                                            \left[
                                                              gA -
                                                              B
                                                              \left(
                                                                \frac{Q}{A}
                                                              \right)^{2}
                                                            \right]
                                                            \frac{\partial z}{\partial x}
                                                            =
                                                            \left(
                                                              \frac{Q}{A}
                                                            \right)^{2}
                                                            \left.
                                                              \frac{\partial A}{\partial x}
                                                              \right|_{z}
                                                              -
                                                              gA\frac{Q^{2}}{K^{2}}
                                                            \end{gathered}
                                                            \label{EqCGe_SV_zQ}
                                                          \end{equation}

                                                          (2)以水位$h$和流量$Q$为因变量的圣维南方程组
                                                          \begin{equation}
                                                            \begin{gathered}
                                                              B\frac{\partial h}{\partial t}
                                                              +
                                                              \frac{\partial Q}{\partial x}
                                                              =
                                                              q
                                                              \\
                                                              \frac{\partial Q}{\partial t}
                                                              +
                                                              \frac{2Q}{A}\frac{\partial Q}{\partial x}
                                                              +
                                                              \left[
                                                                gA -
                                                                B
                                                                \left(
                                                                  \frac{Q}{A}
                                                                \right)^{2}
                                                              \right]
                                                              \frac{\partial h}{\partial x}
                                                              =
                                                              \left(
                                                                \frac{Q}{A}
                                                              \right)^{2}
                                                              \left.
                                                                \frac{\partial A}{\partial x}
                                                                \right|_{h}
                                                                -
                                                                gA\frac{Q^{2}}{K^{2}}
                                                              \end{gathered}
                                                              \label{EqCGe_SV_hQ}
                                                            \end{equation}



                                                            (3)以水深$z$和流量$U$为因变量的圣维南方程组
                                                            \begin{equation}
                                                              \begin{gathered}
                                                                \frac{\partial z}{\partial t}
                                                                +
                                                                U\frac{\partial z}{\partial x}
                                                                +
                                                                \frac{A}{B}\frac{\partial U}{\partial x}
                                                                =
                                                                \frac{1}{B}
                                                                \left(
                                                                  q - BiU - U\left.\frac{\partial A}{\partial x}\right|_{z}
                                                                  \right)
                                                                  \\
                                                                  \frac{\partial U}{\partial t}
                                                                  +
                                                                  U\frac{\partial U}{\partial x}
                                                                  +
                                                                  g\frac{\partial z}{\partial x}
                                                                  =
                                                                  -g\frac{U^{2}}{C^{2}R}
                                                                \end{gathered}
                                                                \label{EqCGe_SV_zU}
                                                              \end{equation}

                                                              (4)以水深$h$和流量$U$为因变量的圣维南方程组
                                                              \begin{equation}
                                                                \begin{gathered}
                                                                  \frac{\partial h}{\partial t}
                                                                  +
                                                                  U\frac{\partial h}{\partial x}
                                                                  +
                                                                  \frac{A}{B}\frac{\partial U}{\partial x}
                                                                  =
                                                                  \frac{1}{B}
                                                                  \left(
                                                                    q - U\left.\frac{\partial A}{\partial x}\right|_{h}
                                                                    \right)
                                                                    \\
                                                                    \frac{\partial U}{\partial t}
                                                                    +
                                                                    U\frac{\partial U}{\partial x}
                                                                    +
                                                                    g\frac{\partial z}{\partial x}
                                                                    =
                                                                    g
                                                                    \left(
                                                                      i-\frac{U^{2}}{C^{2}R}
                                                                    \right)
                                                                  \end{gathered}
                                                                  \label{EqCGe_SV_hU}
                                                                \end{equation}

                                                                (5)以$A$和$Q$为因变量的圣维南方程组
                                                                \begin{equation}
                                                                  \begin{gathered}
                                                                    \frac{\partial A}{\partial t}
                                                                    +
                                                                    \frac{\partial Q}{\partial x}
                                                                    =
                                                                    q
                                                                    \\
                                                                    \frac{\partial Q}{\partial t}
                                                                    +
                                                                    \frac{\partial}{\partial x}\left(\frac{Q^{2}}{A}\right)
                                                                    =
                                                                    -gA\frac{\partial h}{\partial x}
                                                                    -gA\frac{\partial z_{b}}{\partial x}
                                                                    -g\frac{n^{2}|U|}{R^{4/3}}Q
                                                                  \end{gathered}
                                                                  \label{EqCGe_SV_AQ_1}
                                                                \end{equation}
                                                                或
                                                                \begin{equation}
                                                                  \begin{gathered}
                                                                    \frac{\partial A}{\partial t}
                                                                    +
                                                                    \frac{\partial Q}{\partial x}
                                                                    =
                                                                    q
                                                                    \\
                                                                    \frac{\partial Q}{\partial t}
                                                                    +
                                                                    \frac{\partial}{\partial x}\left(\frac{Q^{2}}{A}\right)
                                                                    =
                                                                    -
                                                                    gA\frac{\partial Z}{\partial x}
                                                                    -
                                                                    g\frac{n^{2}Q|Q|}{AR^{4/3}}
                                                                  \end{gathered}
                                                                  \label{EqCGe_SV_AQ_2}
                                                                \end{equation}

                                                                \section{偏微分方程类型和性质}
                                                                \subsection{偏微分方程形式}
                                                                本章前几节所推导的方程都属于偏微分方程,形式各异。对同一模型建立的控制方程也有多
                                                                种不同的形式。在数值计算中,若控制方程的对流项采用散度的形式来表示,这类方程被称
                                                                为守恒型的控制方程,否则被称为非恒定形式。

                                                                理论上,从微元体角度来看,守恒型微分方程和非守恒型微分方程是等价的,都是同一物理
                                                                定律的数学表示。但是,数值计算是对有限大小的计算单元进行的,对有限大小的计算体积
                                                                ,两种形式的控制方程有不同的特性。守恒型微分方程允许流动参数在计算单元或控制体内
                                                                部存在间断;而非守恒型微分方程要求流动参数是可微的。因此,基于守恒型微分方程的数
                                                                值方法,可以直接用来计算有间断(如激波)的流场,且不用对间断进行任何特殊处理。这
                                                                类数值方法被称为激波捕捉方法。而基于非守恒型为微分方程的数值方法,一般无法正确的
                                                                计算有间断的流场。为了处理有间断的流动,这类方法必须与激波装配方法联合使用。简单
                                                                来说,激波装配方法是把间断从流场中分离出来,当作流场的边界来处理。

                                                                \subsection{控制方程的数学分类及其对数值解的影响}

                                                                \subsubsection{双曲型方程}

                                                                \subsubsection{抛物型方程}

                                                                \subsubsection{椭圆型方程}

                                                                \subsubsection{NS方程的数学性质}

                                                                \subsubsection{二维浅水方程的数学性质}

                                                                \subsubsection{一维浅水方程的数学性质}

