\chapter{控制方程}
\section{纳维-斯托克斯方程}
\subsection{模型}
\subsection{连续性方程}
\subsection{运动方程}
\section{雷诺应力平均方程}
\subsection{紊流基础特征}
\subsection{紊流连续性方程}
\subsection{紊流运动方程}
\subsection{紊流模型}
\section{平面二维浅水方程}
天然河道水流运动一般都属于三维流动,运动要素即沿程变化,又沿水深和河宽方向变化。
由于三维水流运动比较复杂,河流数值模拟常用的一种简化方法是将运动要素沿水深方向平
均,把三维问题转化为平面二维问题。本节基于一定条件将三维流动的雷诺平均运动微分方
程简化为平面二维浅水方程。
\subsection{浅水假设和水深平均积分法则}
\subsubsection{浅水假设}
三维流动的雷诺平均运动微分方程如式\eqref{EqRaeC}-\eqref{EqRaeMez}所示。

\begin{subequations}
\begin{align}
  \frac{\partial \widebar{u_{x}}}{\partial x} +
  \frac{\partial \widebar{u_{y}}}{\partial y} +
  \frac{\partial \widebar{u_{z}}}{\partial z} 
  &
  =
  0
 \label{EqRaeC}
  \\
  \frac{\partial \widebar{u_{x}}}{\partial t} +
  \frac{\partial (\widebar{u_{x}}\widebar{u_{x}})}{\partial x} +
  \frac{\partial (\widebar{u_{x}}\widebar{u_{y}})}{\partial y} +
  \frac{\partial (\widebar{u_{x}}\widebar{u_{z}})}{\partial z} 
  &
  =
  \widebar{f_{x}} -
  \frac{1}{\rho}\frac{\partial \widebar{p}}{\partial x} +
  \nu_{t}
  \left(
    \frac{\partial^{2} \widebar{u_{x}}}{\partial x^{2}} +
    \frac{\partial^{2} \widebar{u_{x}}}{\partial y^{2}} +
    \frac{\partial^{2} \widebar{u_{x}}}{\partial z^{2}}
  \right)
 \label{EqRaeMex}
  \\
  \frac{\partial \widebar{u_{y}}}{\partial t} +
  \frac{\partial (\widebar{u_{y}}\widebar{u_{x}})}{\partial x} +
  \frac{\partial (\widebar{u_{y}}\widebar{u_{y}})}{\partial y} +
  \frac{\partial (\widebar{u_{y}}\widebar{u_{z}})}{\partial z} 
  &
  =
  \widebar{f_{y}} -
  \frac{1}{\rho}\frac{\partial \widebar{p}}{\partial y} +
  \nu_{t}
  \left(
    \frac{\partial^{2} \widebar{u_{y}}}{\partial x^{2}} +
    \frac{\partial^{2} \widebar{u_{y}}}{\partial y^{2}} +
    \frac{\partial^{2} \widebar{u_{y}}}{\partial z^{2}}
  \right)
 \label{EqRaeMey}
  \\
  \frac{\partial \widebar{u_{z}}}{\partial t} +
  \frac{\partial (\widebar{u_{z}}\widebar{u_{x}})}{\partial x} +
  \frac{\partial (\widebar{u_{z}}\widebar{u_{y}})}{\partial y} +
  \frac{\partial (\widebar{u_{z}}\widebar{u_{z}})}{\partial z} 
  &
  =
  \widebar{f_{z}} -
  \frac{1}{\rho}\frac{\partial \widebar{p}}{\partial z} +
  \nu_{t}
  \left(
    \frac{\partial^{2} \widebar{u_{z}}}{\partial x^{2}} +
    \frac{\partial^{2} \widebar{u_{z}}}{\partial y^{2}} +
    \frac{\partial^{2} \widebar{u_{z}}}{\partial z^{2}}
  \right) \label{EqRaeMez}
\end{align}
\end{subequations}

在河道、湖泊或水库水流中,水平尺度一般远大于垂向尺度。如果垂向加速度与重力加速度
相比很小,则可以忽略垂向加速度,流速等水力参数沿垂向的变化常采用其垂向平均值,并
假定沿水深方向的动水压强分布符合静水压强分布。三维流动的运动微分方程可简化为:
\begin{subequations}
\begin{align}
  \frac{\partial \widebar{u_{x}}}{\partial x} +
  \frac{\partial \widebar{u_{y}}}{\partial y} +
  \frac{\partial \widebar{u_{z}}}{\partial z} 
  &
  =
  0
 \label{EqRaeCSimp}
  \\
  \frac{\partial \widebar{u_{x}}}{\partial t} +
  \frac{\partial (\widebar{u_{x}}\widebar{u_{x}})}{\partial x} +
  \frac{\partial (\widebar{u_{x}}\widebar{u_{y}})}{\partial y} +
  \frac{\partial (\widebar{u_{x}}\widebar{u_{z}})}{\partial z} 
  &
  =
  -\frac{1}{\rho}\frac{\partial \widebar{p}}{\partial x} +
  \nu_{t}
  \left(
    \frac{\partial^{2} \widebar{u_{x}}}{\partial x^{2}} +
    \frac{\partial^{2} \widebar{u_{x}}}{\partial y^{2}} +
    \frac{\partial^{2} \widebar{u_{x}}}{\partial z^{2}}
  \right)
 \label{EqRaeMexSimp}
  \\
  \frac{\partial \widebar{u_{y}}}{\partial t} +
  \frac{\partial (\widebar{u_{y}}\widebar{u_{x}})}{\partial x} +
  \frac{\partial (\widebar{u_{y}}\widebar{u_{y}})}{\partial y} +
  \frac{\partial (\widebar{u_{y}}\widebar{u_{z}})}{\partial z} 
  &
  =
  -\frac{1}{\rho}\frac{\partial \widebar{p}}{\partial y} +
  \nu_{t}
  \left(
    \frac{\partial^{2} \widebar{u_{y}}}{\partial x^{2}} +
    \frac{\partial^{2} \widebar{u_{y}}}{\partial y^{2}} +
    \frac{\partial^{2} \widebar{u_{y}}}{\partial z^{2}}
  \right)
 \label{EqRaeMeySimp}
  \\
  \frac{\partial \widebar{p}}{\partial z} 
  =
  -\rho g
  \label{EqRaeMezSimp}
\end{align}
\label{EqRae}
\end{subequations}

\subsubsection{水深积分平均法则}
将式\eqref{EqRae}沿水深积分平均,即可得到沿水深平均的平面二维流动的基本方程。在沿
水深积分平均过程中,采用以下定义和公式:

(1)定义水深为
\begin{equation}
  H = \zeta - z_{0}
\end{equation}
式中,$H$为水深,$\zeta=\zeta(x,y,t)$、$z_{0}=z_{0}(x,y,t)$分别为某一基准面下的
水面高程和河床高程(见图)

\begin{figure}
  \begin{tikzpicture}
  \end{tikzpicture}
  \caption{水位基准示意图}
\end{figure}

(2)定义沿水深平均流速$U_i$为
\begin{equation}
  U_i
  =
  \frac{1}{H}
\int_{z_{0}}^{\zeta}\!\overline{u_{i}}\mathrm{d}z
\end{equation}
式中,下标$i$取1,2和3分别对应$x$,$y$和$z$方向的速度分量。

(3)莱布尼兹公式
\begin{equation}
  \frac{\partial}{\partial x_{i}}
  \int_{a}^{b}\!f\mathrm{d}z
  =
  \int_{a}^{b}\!
  \frac{\partial f}{\partial x_{i}}\mathrm{d}z
  +
  \left.f\right|_{b}\frac{\partial b}{\partial x_{i}}
  -
  \left.f\right|_{a}\frac{\partial a}{\partial x_{i}}
    \label{EqCGeLeibnizLaw}
\end{equation}
式中,$x_{i}=x$,$y$,$t$。$a$、$b$和$f$都是$x_{i}$的函数。

(4)自由表面及河床底部运动学条件为:
\begin{equation}
  \left.\overline{u_{z}}\right|_{z=\zeta}
    =
    \frac{D\zeta}{Dt}
    =
    \frac{\partial\zeta}{\partial t}
    +
    \left.\overline{u_{x}}\right|_{z=\zeta}\frac{\partial\zeta}{\partial x}
      +
    \left.\overline{u_{y}}\right|_{z=\zeta}\frac{\partial\zeta}{\partial y}
    \label{EqCGeSMC}
\end{equation}
\begin{equation}
  \left.\overline{u_{z}}\right|_{z=z_{0}}
    =
    \frac{Dz_{0}}{Dt}
    =
    \frac{\partial z_{0}}{\partial t}
    +
    \left.\overline{u_{x}}\right|_{z=z_{0}}\frac{\partial z_{0}}{\partial x}
      +
    \left.\overline{u_{y}}\right|_{z=z_{0}}\frac{\partial z_{0}}{\partial y}
    \label{EqCGeBMC}
\end{equation}

\subsection{沿水深平均的连续性方程}
采用上述定义和公式对连续性方程\eqref{EqRaeCSimp}沿水深积分平均得:
\begin{equation}
  \int_{z_{0}}^{\zeta}\!
  \left(
  \frac{\partial \widebar{u_{x}}}{\partial x} +
  \frac{\partial \widebar{u_{y}}}{\partial y} +
  \frac{\partial \widebar{u_{z}}}{\partial z} 
  \right)
  \mathrm{d}z
  =
  0
  \label{EqCGeRaeCE_Int}
\end{equation}
根据式\eqref{EqCGeLeibnizLaw},式\eqref{EqCGeRaeCE_Int}中前两项分别可写成
\begin{equation}
  \int_{z_{0}}^{\zeta}\!
  \frac{\partial \widebar{u_{x}}}{\partial x} 
  \mathrm{d}z
  =
  \frac{\partial}{\partial x}
  \int_{z_{0}}^{\zeta}\!
  \widebar{u_{x}}
  \mathrm{d}z
  -
  \left.\widebar{u_{x}}\right|_{z=\zeta}
    \frac{\partial\zeta}{\partial z}
  +
  \left.\widebar{u_{x}}\right|_{z=z_{0}}
    \frac{\partial z_{0}}{\partial z}
\label{EqCGeRaeCE_Int_x}
\end{equation}
\begin{equation}
  \int_{z_{0}}^{\zeta}\!
  \frac{\partial \widebar{u_{y}}}{\partial y} 
  \mathrm{d}z
  =
  \frac{\partial}{\partial y}
  \int_{z_{0}}^{\zeta}\!
  \widebar{u_{y}}
  \mathrm{d}z
  -
  \left.\widebar{u_{y}}\right|_{z=\zeta}
    \frac{\partial\zeta}{\partial z}
  +
  \left.\widebar{u_{y}}\right|_{z=z_{0}}
    \frac{\partial z_{0}}{\partial z}
\label{EqCGeRaeCE_Int_y}
\end{equation}
式\eqref{EqCGeRaeCE_Int}中最后一项
\begin{equation}
  \int_{z_{0}}^{\zeta}\!
  \frac{\partial \widebar{u_{z}}}{\partial z} 
  \mathrm{d}z
  =
  \left.\widebar{u_{z}}\right|_{z=\zeta}
    -
    \left.\widebar{u_{z}}\right|_{z=z_{0}}
\label{EqCGeRaeCE_Int_z}
\end{equation}
将式\eqref{EqCGeRaeCE_Int_x},\eqref{EqCGeRaeCE_Int_y}和\eqref{EqCGeRaeCE_Int_z}
代入\eqref{EqCGeRaeCE_Int},并利用自由表面及河床底部运动学条件式\eqref{EqCGeSMC}
和\eqref{EqCGeBMC},可得
\begin{equation*}
  \begin{aligned}
  &\int_{z_{0}}^{\zeta}\!
  \left(
  \frac{\partial \widebar{u_{x}}}{\partial x} +
  \frac{\partial \widebar{u_{y}}}{\partial y} +
  \frac{\partial \widebar{u_{z}}}{\partial z} 
  \right)
  \mathrm{d}z
  \\
  =&
  \frac{\partial}{\partial x}
  \int_{z_{0}}^{\zeta}\!
  \widebar{u_{x}}
  \mathrm{d}z
  -
  \left.\widebar{u_{x}}\right|_{z=\zeta}
    \frac{\partial\zeta}{\partial z}
  +
  \left.\widebar{u_{x}}\right|_{z=z_{0}}
    \frac{\partial z_{0}}{\partial z}
   + \\
  &
  \frac{\partial}{\partial y}
  \int_{z_{0}}^{\zeta}\!
  \widebar{u_{y}}
  \mathrm{d}z
  -
  \left.\widebar{u_{y}}\right|_{z=\zeta}
    \frac{\partial\zeta}{\partial z}
  +
  \left.\widebar{u_{y}}\right|_{z=z_{0}}
    \frac{\partial z_{0}}{\partial z}
    + \\
  &
  \left.\widebar{u_{z}}\right|_{z=\zeta}
    -
    \left.\widebar{u_{z}}\right|_{z=z_{0}}
      \\
  =&
  \frac{\partial HU_{x}}{\partial x} +
  \frac{\partial HU_{y}}{\partial x} +
  \frac{\partial \zeta}{\partial t} -
  \frac{\partial z_{0}}{\partial t}
  =
  0
  \end{aligned}
\end{equation*}
最后得
\begin{equation}
  \frac{\partial H}{\partial t} +
  \frac{\partial HU_{x}}{\partial x} +
  \frac{\partial HU_{y}}{\partial y}
  =
  0
\end{equation}
也可以写成
\begin{equation}
  \frac{\partial H}{\partial t} +
  \nabla\cdot(H\mathbf{U})
  =
  0
\end{equation}

\subsection{沿水深平均的运动方程}
以$x$方向为例,式\eqref{EqRaeMexSimp}沿水深积分为
\begin{equation}
    \int_{z_{0}}^{\zeta}\!
    \left[
  \frac{\partial \widebar{u_{x}}}{\partial t} +
  \frac{\partial (\widebar{u_{x}}\widebar{u_{x}})}{\partial x} +
  \frac{\partial (\widebar{u_{x}}\widebar{u_{y}})}{\partial y} +
  \frac{\partial (\widebar{u_{x}}\widebar{u_{z}})}{\partial z} +
  \frac{1}{\rho}\frac{\partial \widebar{p}}{\partial x} -
  \nu_{t}
  \left(
    \frac{\partial^{2} \widebar{u_{x}}}{\partial x^{2}} +
    \frac{\partial^{2} \widebar{u_{x}}}{\partial y^{2}} +
    \frac{\partial^{2} \widebar{u_{x}}}{\partial z^{2}}
  \right)
  \right]
  \mathrm{d}z
  =
  0
  \label{EqCGeMex_Int}
\end{equation}
式\eqref{EqCGeMex_Int}中包含了非恒定流项积分、对流项积分、压力项积分和阻力项积分
。接下来分项讨论。

(1)非恒定流项积分
\begin{equation}
    \begin{aligned}
    &
    \int_{z_0}^{\zeta}\!
    \frac{\partial \widebar{u_{x}}}{\partial t} 
    \mathrm{d}z
    =
    \frac{\partial}{\partial t}
    \int_{z_0}^{\zeta}\!
    \widebar{u_{x}}
    \mathrm{d}z
    -
    \left.\widebar{u_{x}}\right|_{z=\zeta}\frac{\partial \zeta}{\partial t}
    +
    \left.\widebar{u_{x}}\right|_{z=z_{0}}\frac{\partial z_{0}}{\partial t}
        \\
        =&
    \frac{\partial HU_{x}}{\partial t}
    -
    \left.\widebar{u_{x}}\right|_{z=\zeta}\frac{\partial \zeta}{\partial t}
    +
    \left.\widebar{u_{x}}\right|_{z=z_{0}}\frac{\partial z_{0}}{\partial t}
    \end{aligned}
        \label{EqCGe_Mex_US_Int}
\end{equation}

(2)对流项积分

首先将时均流速按照式\eqref{EqCGe_VelDecompse}进行分解
\begin{equation}
\begin{aligned}
\widebar{u_{x}}
=
U_{x}+\Delta\widebar{u_{x}}
\\
\widebar{u_{y}}
=
U_{y}+\Delta\widebar{u_{y}}
\end{aligned}
\label{EqCGe_VelDecompse}
\end{equation}
式中,$\Delta\widebar{u_{x}}$和$\Delta\widebar{u_{y}}$分别为$x$和$y$的时均流速与
垂线平均流速的差值。

对流项中第一项的积分
\begin{equation}
    \int_{z_0}^{\zeta}\!
    \frac{\partial \widebar{u_{x}}\widebar{u_{x}}}{\partial x}
    \mathrm{d}z
    =
    \frac{\partial}{\partial x}
    \int_{z_0}^{\zeta}\!
    \widebar{u_{x}}\widebar{u_{x}}
    \mathrm{d}z
    -
    \left.\widebar{u_{x}}\widebar{u_{x}}\right|_{z=\zeta}\frac{\partial \zeta}{\partial x}
    +
    \left.\widebar{u_{x}}\widebar{u_{x}}\right|_{z=z_{0}}\frac{\partial z_{0}}{\partial x}
    \label{EqCGe_CT_1st}
\end{equation}
式中
\begin{equation*}
    \begin{aligned}
    \int_{z_0}^{\zeta}\!
    \widebar{u_{x}}\widebar{u_{x}}
    \mathrm{d}z
    &=
    \int_{z_0}^{\zeta}\!
    (U_{x}+\Delta\widebar{u_{x}})(U_{x}+\Delta\widebar{u_{x}})
    \mathrm{d}z
    \\
    &=
    \int_{z_0}^{\zeta}\!
    (U_{x}U_{x}+2U_{x}\Delta\widebar{u_{x}}+\Delta\widebar{u_{x}}\Delta\widebar{u_{x}})
    \mathrm{d}z
    \\
    &=
    HU_{x}U_{x}
    +
    \int_{z_0}^{\zeta}\!
    \Delta\widebar{u_{x}}\Delta\widebar{u_{x}}
    \mathrm{d}z
    \\
    &=
    \beta_{xx}HU_{x}U_{x}
    \end{aligned}
\end{equation*}
其中
\begin{equation}
\beta_{xx}
=
1 +
\frac
{
\int_{z_0}^{\zeta}\!
    \Delta\widebar{u_{x}}\Delta\widebar{u_{x}}
\mathrm{d}z
}
{HU_{x}U_{x}}
\end{equation}
是由于流速沿垂线分布不均匀而引入的修正系数,类似于水力学中的动量修正系数。
$\beta_{xx}$的取值一般在1.02与1.05之间,可近似取为1.0。因此
\begin{equation}
    \int_{z_0}^{\zeta}\!
    \frac{\partial \widebar{u_{x}}\widebar{u_{x}}}{\partial x}
    \mathrm{d}z
    =
    \frac{\partial HU_{x}U_{x}}{\partial x}
    -
    \left.\widebar{u_{x}}\widebar{u_{x}}\right|_{z=\zeta}\frac{\partial \zeta}{\partial x}
    +
    \left.\widebar{u_{x}}\widebar{u_{x}}\right|_{z=z_{0}}\frac{\partial z_{0}}{\partial x}
\end{equation}

同理,对流项的第二项可写为
\begin{equation}
    \int_{z_0}^{\zeta}\!
    \frac{\partial \widebar{u_{x}}\widebar{u_{y}}}{\partial y}
    \mathrm{d}z
    =
    \frac{\partial HU_{x}U_{y}}{\partial x}
    -
    \left.\widebar{u_{x}}\widebar{u_{y}}\right|_{z=\zeta}\frac{\partial \zeta}{\partial y}
    +
    \left.\widebar{u_{x}}\widebar{u_{y}}\right|_{z=z_{0}}\frac{\partial z_{0}}{\partial y}
\end{equation}
对流项的第三项
\begin{equation}
    \int_{z_0}^{\zeta}\!
    \frac{\partial \widebar{u_{x}}\widebar{u_{z}}}{\partial z}
    \mathrm{d}z
    =
    \left.\widebar{u_{x}}\widebar{u_{z}}\right|_{z=\zeta}
    -
    \left.\widebar{u_{x}}\widebar{u_{z}}\right|_{z=z_{0}}
\end{equation}

将非恒定流项和对流项积分相加,并利用自由表面和河床底部运动学条件可得:
\begin{equation}
    \int_{z_{0}}^{\zeta}\!
    \left[
  \frac{\partial \widebar{u_{x}}}{\partial t} +
  \frac{\partial (\widebar{u_{x}}\widebar{u_{x}})}{\partial x} +
  \frac{\partial (\widebar{u_{x}}\widebar{u_{y}})}{\partial y} +
  \frac{\partial (\widebar{u_{x}}\widebar{u_{z}})}{\partial z}
  \right]
  \mathrm{d}z
=
\frac{\partial HU_{x}}{\partial t}
+
\frac{\partial HU_{x}U_{x}}{\partial x}
+
\frac{\partial HU_{x}U_{y}}{\partial y}
\end{equation}

(3)压力项积分
\begin{equation}
  \begin{aligned}
    \int_{z_{0}}^{\zeta}\!
    \frac{\partial \widebar{p}}{\partial x}
    \mathrm{d}z
    &=
    \frac{\partial}{\partial x}
    \int_{z_0}^{\zeta}\!
    \widebar{p}
    \mathrm{d}z
    -
    \left.\widebar{p}\right|_{z=\zeta}\frac{\partial \zeta}{\partial x}
    +
    \left.\widebar{p}\right|_{z=z_{0}}\frac{\partial z_{0}}{\partial x}
      \\
    &=
    \frac{\partial}{\partial x}
    \int_{z_0}^{\zeta}\!
    \rho g(\zeta-z)
    \mathrm{d}z
    -
    \left.\rho g(\zeta-z)\right|_{z=\zeta}\frac{\partial \zeta}{\partial x}
    +
    \left.\rho g(\zeta-z)\right|_{z=z_{0}}\frac{\partial z_{0}}{\partial x}
      \\
    &=
    \rho gH\frac{\partial H}{\partial x} 
    +
    \rho gH\frac{\partial z_{0}}{\partial x} 
    =
    \rho gH\frac{\partial \zeta}{\partial x} 
  \end{aligned}
\end{equation}

(4)阻力项积分
\begin{equation}
  \begin{aligned}
    &\int_{z_{0}}^{\zeta}\!
    \frac{\partial^{2} \widebar{u_{x}}}{\partial x^{2}}
    \mathrm{d}z
    =
    \int_{z_{0}}^{\zeta}\!
    \frac{\partial}{\partial x}
    \left(\frac{\partial \widebar{u_{x}}}{\partial x}\right)
    \mathrm{d}z
    =
    \frac{\partial}{\partial x}
    \int_{z_0}^{\zeta}\!
    \frac{\partial \widebar{u_{x}}}{\partial x}
    \mathrm{d}z
    -
    \left.\frac{\partial \widebar{u_{x}}}{\partial x}\right|_{z=\zeta}\frac{\partial \zeta}{\partial x}
    +
    \left.\frac{\partial \widebar{u_{x}}}{\partial x}\right|_{z=z_{0}}\frac{\partial z_{0}}{\partial x}
    \\
    =&
    \frac{\partial}{\partial x}
    \left(
    \frac{\partial}{\partial x}
    \int_{z_0}^{\zeta}\!
    \widebar{u_{x}}
    \mathrm{d}z
    -
    \left.\widebar{u_{x}}\right|_{z=\zeta}\frac{\partial \zeta}{\partial x}
    +
    \left.\widebar{u_{x}}\right|_{z=z_{0}}\frac{\partial z_{0}}{\partial x}
    \right)
    -
    \left.\frac{\partial \widebar{u_{x}}}{\partial x}\right|_{z=\zeta}\frac{\partial \zeta}{\partial x}
    +
    \left.\frac{\partial \widebar{u_{x}}}{\partial x}\right|_{z=z_{0}}\frac{\partial z_{0}}{\partial x}
      \\
    =&
    \frac{\partial^{2} HU_{x}}{\partial x^{2}} +
    \frac{\partial}{\partial x}
    \left(
    -
    \left.\widebar{u_{x}}\right|_{z=\zeta}\frac{\partial \zeta}{\partial x}
    +
    \left.\widebar{u_{x}}\right|_{z=z_{0}}\frac{\partial z_{0}}{\partial x}
    \right)
    -
    \left.\frac{\partial \widebar{u_{x}}}{\partial x}\right|_{z=\zeta}\frac{\partial \zeta}{\partial x}
    +
    \left.\frac{\partial \widebar{u_{x}}}{\partial x}\right|_{z=z_{0}}\frac{\partial z_{0}}{\partial x}
  \end{aligned}
  \label{EqCGe_Shear_1}
\end{equation}

类似地
\begin{equation}
    \int_{z_{0}}^{\zeta}\!
    \frac{\partial^{2} \widebar{u_{x}}}{\partial y^{2}}
    \mathrm{d}z
    =
    \frac{\partial^{2} HU_{x}}{\partial y^{2}} +
    \frac{\partial}{\partial y}
    \left(
    -
    \left.\widebar{u_{x}}\right|_{z=\zeta}\frac{\partial \zeta}{\partial y}
    +
    \left.\widebar{u_{x}}\right|_{z=z_{0}}\frac{\partial z_{0}}{\partial y}
    \right)
    -
    \left.\frac{\partial \widebar{u_{x}}}{\partial y}\right|_{z=\zeta}\frac{\partial \zeta}{\partial y}
    +
    \left.\frac{\partial \widebar{u_{x}}}{\partial y}\right|_{z=z_{0}}\frac{\partial z_{0}}{\partial y}
  \label{EqCGe_Shear_2}
\end{equation}
\begin{equation}
    \int_{z_{0}}^{\zeta}\!
    \frac{\partial^{2} \widebar{u_{x}}}{\partial y^{2}}
    \mathrm{d}z
    =
    \left.\frac{\partial \widebar{u_{x}}}{\partial z}\right|_{z=\zeta}
      -
      \left.\frac{\partial \widebar{u_{x}}}{\partial z}\right|_{z=z_{0}}
  \label{EqCGe_Shear_3}
\end{equation}
将式\eqref{EqCGe_Shear_1}至\eqref{EqCGe_Shear_3}相加,得
\begin{equation}
  \begin{aligned}
    &\int_{z_{0}}^{\zeta}\!
    \nu_{t}
    \left(
      \frac{\partial^{2} \widebar{u_{x}}}{\partial x^{2}} +
      \frac{\partial^{2} \widebar{u_{x}}}{\partial y^{2}} +
      \frac{\partial^{2} \widebar{u_{x}}}{\partial z^{2}}
    \right)
    \mathrm{d}z =\\
     &
    \nu_{t}
    \left(
    \frac{\partial^{2} HU_{x}}{\partial x^{2}} +
    \frac{\partial^{2} HU_{x}}{\partial y^{2}}
  \right)
  \\
      -&\nu_{t}
      \left[
    \frac{\partial}{\partial x}
    \left(
    \frac{\partial \zeta}{\partial x}
\left.\widebar{u_{x}}\right|_{z=\zeta}
\right)
+
    \frac{\partial}{\partial y}
    \left(
    \frac{\partial \zeta}{\partial y}
\left.\widebar{u_{x}}\right|_{z=\zeta}
\right)
+
\left.\frac{\partial \widebar{u_{x}}}{\partial x}\right|_{z=\zeta}
    \frac{\partial \zeta}{\partial x}
  +
\left.\frac{\partial \widebar{u_{x}}}{\partial y}\right|_{z=\zeta}
    \frac{\partial \zeta}{\partial y}
    -
    \left.\frac{\partial \widebar{u_{x}}}{\partial z}\right|_{z=\zeta}
        \right]
        \\
      +&\nu_{t}
      \left[
    \frac{\partial}{\partial x}
    \left(
    \frac{\partial z_{0}}{\partial x}
\left.\widebar{u_{x}}\right|_{z=z_{0}}
\right)
+
    \frac{\partial}{\partial y}
    \left(
    \frac{\partial z_{0}}{\partial y}
\left.\widebar{u_{x}}\right|_{z=z_{0}}
\right)
+
\left.\frac{\partial \widebar{u_{x}}}{\partial x}\right|_{z=z_{0}}
    \frac{\partial z_{0}}{\partial x}
  +
\left.\frac{\partial \widebar{u_{x}}}{\partial y}\right|_{z=z_{0}}
    \frac{\partial z_{0}}{\partial y}
    \left.\frac{\partial \widebar{u_{x}}}{\partial z}\right|_{z=z_{0}}
        \right]
  \end{aligned}
  \label{EqCGe_Shear_Total}
\end{equation}
式\eqref{EqCGe_Shear_Total}中右边后两项分别为由底部床面阻力和自由表面风阻力引起
的阻力项,通常可以式\eqref{EqCGe_Shear}表示:
\begin{equation}
  g\frac{n^{2}U_{x}\sqrt{U_{x}^{2}+U_{y}^{2}}}{H^{1/3}}
  -
  C_{w}\frac{\rho_{a}}{\rho}\omega^{2}\cos\beta
  \label{EqCGe_Shear}
\end{equation}
式中,$C_{w}$为无因此风应力系数,$\rho_{a}$为空气密度,$\omega$为风速,$\beta$为
风向与$x$方向的夹角。最后,$x$方向的运动方程为
\begin{equation}
  \begin{aligned}
  \frac{\partial HU_{x}}{\partial t} +
  &\frac{\partial HU_{x}U_{x}}{\partial x} +
  \frac{\partial HU_{x}U_{y}}{\partial y} 
  =\\
  &-gH\frac{\partial \zeta}{\partial x}
  -g\frac{n^{2}U_{x}\sqrt{U_{x}^{2}+U_{y}^{2}}}{H^{1/3}}
  +
  \nu_{t}\left(
    \frac{\partial^{2}HU_{x}}{\partial x^{2}}+
    \frac{\partial^{2}HU_{x}}{\partial y^{2}}
\right)
  +C_{w}\frac{\rho_{a}}{\rho}\omega^{2}\cos\beta
  \end{aligned}
  \label{EqCGe_Me_x}
\end{equation}
同理可得,$y$方向运动方程为
\begin{equation}
  \begin{aligned}
  \frac{\partial HU_{y}}{\partial t} +
  &\frac{\partial HU_{x}U_{y}}{\partial x} +
  \frac{\partial HU_{y}U_{y}}{\partial y} 
  =\\
  &-gH\frac{\partial \zeta}{\partial y}
  -g\frac{n^{2}U_{y}\sqrt{U_{x}^{2}+U_{y}^{2}}}{H^{1/3}}
  +
  \nu_{t}\left(
    \frac{\partial^{2}HU_{y}}{\partial x^{2}}+
    \frac{\partial^{2}HU_{y}}{\partial y^{2}}
\right)
  +C_{w}\frac{\rho_{a}}{\rho}\omega^{2}\sin\beta
  \end{aligned}
  \label{EqCGe_Me_y}
\end{equation}

当自由表面风应力影响较小时,风应力项可以忽略。此外,当模拟区域尺度较大,还要考虑
地球自转的影响,可在方程\eqref{EqCGe_Me_x}和\eqref{EqCGe_Me_y}右边分别加入科氏力。
\begin{equation}
  \begin{aligned}
    f_x = 2\omega\sin\varphi U_{x} 
    \\
    f_y = 2\omega\sin\varphi U_{y} 
  \end{aligned}
\end{equation}
式中,$\omega$为地球自转角速度,$\varphi$为模拟区域所处纬度。

根据推导过程中所采用的假定条件,在使用二维浅水方程时应注意以下问题:
\begin{enumerate}
  \item 方程推导中引用了牛顿流体所满足的本构关系式,因此上述方程只适用于牛顿流体
,对类似高含沙水流的非牛顿流体不适用。
 \item 方程推导中对流体做了均质不可压的假设,因此上述方程只能在含沙量较小的情况下近似使
用,当含沙量较大时,应考虑密度变化的影响。
\item 在垂向积分过程中,略去流速等水力参数沿垂直方向的变化,并假定沿水深方向的动水压强
分布符合静水压强分布。因此所研究问题的水平尺度应远大于垂向尺度,流速等水力参数沿
垂直方向的变化较之沿水平方向的变化要小得多。
\end{enumerate}

\subsection{二维平面浅水方程不同形式}

\section{一维非恒定流基本控制方程}
\subsection{一维连续性方程}
如图\ref{FigCGe_}所示,在明槽非恒定流中,沿水流流动方向取长为$\mathrm{d}x$的微小
流段。流段进口1-1断面流量为$Q$,在$\Delta t$时段内,从1-1断面进入流段的液体
质量为$\rho Q\Delta t$。流段出口2-2断面流量为$Q+\frac{\partial Q}{\partial
x}\mathrm{d}x$,在$\Delta t$时段内,从2-2断面流出的液体质量为$\rho Q\Delta
t+\rho\frac{\partial Q}{\partial x}\mathrm{d}x\Delta t$。该时段内进出此流段的液体
质量差为$-\rho\frac{\partial Q}{\partial x}\mathrm{d}x\Delta t$。

时段内的质量差表现为流段内的槽蓄量变化。在起始时刻,流段内的槽蓄量为
$\rho\overline{A}\mathrm{d}x$。而经过$\Delta t$时段后,流段内的槽蓄量为
$\rho\left(\overline{A}+\frac{\partial\overline{A}}{\partial t}\Delta
t\right)\mathrm{d}x$。$\Delta t$时段内,流段内的槽蓄量变化量为
$\rho\frac{\partial \overline{A}}{\partial t}\Delta t\mathrm{d}x$。其中
$\overline{A}$为微小流段的平均过水面积。当流段内过水断面面积变化较小时,可直接用
1-1断面段面积$A$来替代$\overline{A}$。

因此,根据质量守恒原理,进出该流段的液体质量差等于流段内槽蓄量改变量,即
\begin{equation*}
  -\rho\frac{\partial Q}{\partial x}\mathrm{d}x\Delta t
  =
  \rho\frac{\partial A}{\partial t}\mathrm{d}x\Delta t
\end{equation*}
化简后得到明槽一维非恒定流连续性方程
\begin{equation}
  \frac{\partial A}{\partial t}
  +
  \frac{\partial Q}{\partial x}
  =
  0
  \label{EqCGe_SVe_Ce}
\end{equation}

如果在该流段内有旁侧入流或出流,且单位长度旁侧入流流量为$q$($q>0$为入流,$q<0$为出流
),考虑旁侧入流得明槽一维非恒定流连续性方程为
\begin{equation}
\frac{\partial A}{\partial t}
+
\frac{\partial Q}{\partial x}
=
q
\end{equation}

\subsection{一维运动方程}
设坐标轴$x$方向与水流流动方向一致,根据牛顿第二定律建立运动方程。为分析简单起见
,首先考虑棱柱体明槽(如图)的情况。

作用在1-1断面的所有外力在$x$方向的分力有:

(1)总动水压力。

设压强分布服从静水压强分布,则作用在1-1断面的水压力
\begin{equation}
P
=
\int_{0}^{h}\! \rho g(h-y)\xi(y)\mathrm{d}y
\end{equation}
式中,$\xi(y)$为过水断面上距渠底$y$处的宽度。
作用于2-2断面的水压力为$P+\frac{\partial P}{\partial x}\mathrm{d}x$。则沿$x$方向
的总动水压力
\begin{equation}
  \begin{aligned}
    \sum P
    =&
    P - \left(P+\frac {\partial P} {\partial x}\mathrm{d}x\right)
    \\
    =&
    -\gamma \mathrm{d}x 
    \left[
      \frac{\partial h}{\partial x}
      \int_{0}^{h}\!
      \zeta(y)
      \mathrm{d}y
      +
      \int_{0}^{h}\!
      (h-y)
      \frac{\partial \zeta(y)}{\partial x}
      \mathrm{d}y
    \right]
  \end{aligned}
  \label{EqCGe_SVe_Me_Pressure}
\end{equation}
因假定明槽为棱柱体明槽,有$\frac{\partial \xi(y)}{\partial x}=0$。则有
\begin{equation}
\sum P
=
-\gamma A\frac{\partial h}{\partial x}\mathrm{d}x
\end{equation}

(2)重力
\begin{equation}
  \mathrm{d}G_{x}
  =
  \mathrm{d}G\sin\alpha
  =
  -\gamma A\mathrm{d}x\frac{\partial z}{\partial x}
\end{equation}
式中,$\alpha$为坐标轴$x$与水平方向的夹角,$A$为过水断面面积。

(3)侧壁面上的阻力
\begin{equation}
\mathrm{d}T
=
\tau_{0}\chi\mathrm{d}x
=
\gamma RJ\chi\mathrm{d}x
=
\gamma AJ\mathrm{d}x
\end{equation}
式中,$\chi$为过水断面湿周,$R$为过水断面水力半径,$J$为水力坡度,
$\tau_{0}=\gamma RJ$为侧壁表面平均切应力。

其次,由于流速$U$是$x$和$t$的函数,则水流沿$x$方向的加速度$a_{x}$为
\begin{equation}
  a_{x}
=
\frac{\mathrm{d} U}{\mathrm{d} t}
=
\frac{\partial U}{\partial t}
+
U
\frac{\partial U}{\partial x}
\end{equation}
微小流段内的水体质量为$\mathrm{d}m=\rho A\mathrm{d}x$。

根据牛顿第二定律,有$\sum F_{x}=\mathrm{d}ma_{x}$,即
\begin{equation}
-\gamma A\frac{\partial h}{\partial x}\mathrm{d}x
-\gamma A\frac{\partial z}{\partial x}\mathrm{d}x
-\gamma AJ\mathrm{d}x
=
\rho A\mathrm{d}x
\left(
\frac{\partial U}{\partial t}
+
U
\frac{\partial U}{\partial x}
\right)
\end{equation}
上式两边同除以$\gamma A\mathrm{d}x$并整理得:
\begin{equation}
\frac{\partial z}{\partial x}
+
\frac{1}{g}
\frac{\partial U}{\partial t}
+
\frac{U}{g}
\frac{\partial U}{\partial x}
+
J
=
0
\label{EqCGe_SVe_Me}
\end{equation}
式\eqref{EqCGe_SVe_Me}即为棱柱体明槽非恒定流运动方程得一般形式。对于非棱柱体明槽
(比如河槽向下游缩窄或展宽),则两岸壁将对微小流段水体作用一附加压力,该附加压力
可表示为
\begin{equation}
  Vp^{\prime}
  =
  \int_{0}^{h}\!
  \left[
    \rho g(h-y)
    \frac{\partial \xi(y)}{\partial x}
    \mathrm{d}x
  \right]
  \mathrm{d}y
\end{equation}
将附加压力代式\eqref{EqCGe_SVe_Me_Pressure}入中,恰好与该式最后一项抵消。因此对
于非棱柱体明槽,式\eqref{EqCGe_SVe_Me}仍适用。

\subsection{圣维南方程不同形式}

