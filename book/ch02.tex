\chapter{控制方程}
\section{纳维-斯托克斯方程}
\subsection{模型}
\subsection{连续性方程}
\subsection{运动方程}
\section{雷诺应力平均方程}
\subsection{紊流基础特征}
\subsection{紊流连续性方程}
\subsection{紊流运动方程}
\subsection{紊流模型}
\section{平面二维浅水方程}
天然河道水流运动一般都属于三维流动,运动要素即沿程变化,又沿水深和河宽方向变化。
由于三维水流运动比较复杂,河流数值模拟常用的一种简化方法是将运动要素沿水深方向平
均,把三维问题转化为平面二维问题。本节基于一定条件将三维流动的雷诺平均运动微分方
程简化为平面二维浅水方程。
\subsection{浅水假设和水深平均积分法则}
\subsubsection{浅水假设}
三维流动的雷诺平均运动微分方程如式\eqref{EqRaeC}-\eqref{EqRaeMez}所示。

\begin{subequations}
\begin{align}
  \frac{\partial \widebar{u_{x}}}{\partial x} +
  \frac{\partial \widebar{u_{y}}}{\partial y} +
  \frac{\partial \widebar{u_{z}}}{\partial z} 
  &
  =
  0
 \label{EqRaeC}
  \\
  \frac{\partial \widebar{u_{x}}}{\partial t} +
  \frac{\partial (\widebar{u_{x}}\widebar{u_{x}})}{\partial x} +
  \frac{\partial (\widebar{u_{x}}\widebar{u_{y}})}{\partial y} +
  \frac{\partial (\widebar{u_{x}}\widebar{u_{z}})}{\partial z} 
  &
  =
  \widebar{f_{x}} -
  \frac{1}{\rho}\frac{\partial \widebar{p}}{\partial x} +
  \nu_{t}
  \left(
    \frac{\partial^{2} \widebar{u_{x}}}{\partial x^{2}} +
    \frac{\partial^{2} \widebar{u_{x}}}{\partial y^{2}} +
    \frac{\partial^{2} \widebar{u_{x}}}{\partial z^{2}}
  \right)
 \label{EqRaeMex}
  \\
  \frac{\partial \widebar{u_{y}}}{\partial t} +
  \frac{\partial (\widebar{u_{y}}\widebar{u_{x}})}{\partial x} +
  \frac{\partial (\widebar{u_{y}}\widebar{u_{y}})}{\partial y} +
  \frac{\partial (\widebar{u_{y}}\widebar{u_{z}})}{\partial z} 
  &
  =
  \widebar{f_{y}} -
  \frac{1}{\rho}\frac{\partial \widebar{p}}{\partial y} +
  \nu_{t}
  \left(
    \frac{\partial^{2} \widebar{u_{y}}}{\partial x^{2}} +
    \frac{\partial^{2} \widebar{u_{y}}}{\partial y^{2}} +
    \frac{\partial^{2} \widebar{u_{y}}}{\partial z^{2}}
  \right)
 \label{EqRaeMey}
  \\
  \frac{\partial \widebar{u_{z}}}{\partial t} +
  \frac{\partial (\widebar{u_{z}}\widebar{u_{x}})}{\partial x} +
  \frac{\partial (\widebar{u_{z}}\widebar{u_{y}})}{\partial y} +
  \frac{\partial (\widebar{u_{z}}\widebar{u_{z}})}{\partial z} 
  &
  =
  \widebar{f_{z}} -
  \frac{1}{\rho}\frac{\partial \widebar{p}}{\partial z} +
  \nu_{t}
  \left(
    \frac{\partial^{2} \widebar{u_{z}}}{\partial x^{2}} +
    \frac{\partial^{2} \widebar{u_{z}}}{\partial y^{2}} +
    \frac{\partial^{2} \widebar{u_{z}}}{\partial z^{2}}
  \right) \label{EqRaeMez}
\end{align}
\end{subequations}

在浅水湖泊或水库中,如果垂向加速度与重力加速度相比很小,则可以忽略垂向加速度,并
假定沿水深方向的动水压强分布符合静水压强分布,则三维流动的运动微分方程可简化为:
\begin{subequations}
\begin{align}
  \frac{\partial \widebar{u_{x}}}{\partial x} +
  \frac{\partial \widebar{u_{y}}}{\partial y} +
  \frac{\partial \widebar{u_{z}}}{\partial z} 
  &
  =
  0
 \label{EqRaeCSimp}
  \\
  \frac{\partial \widebar{u_{x}}}{\partial t} +
  \frac{\partial (\widebar{u_{x}}\widebar{u_{x}})}{\partial x} +
  \frac{\partial (\widebar{u_{x}}\widebar{u_{y}})}{\partial y} +
  \frac{\partial (\widebar{u_{x}}\widebar{u_{z}})}{\partial z} 
  &
  =
  -\frac{1}{\rho}\frac{\partial \widebar{p}}{\partial x} +
  \nu_{t}
  \left(
    \frac{\partial^{2} \widebar{u_{x}}}{\partial x^{2}} +
    \frac{\partial^{2} \widebar{u_{x}}}{\partial y^{2}} +
    \frac{\partial^{2} \widebar{u_{x}}}{\partial z^{2}}
  \right)
 \label{EqRaeMexSimp}
  \\
  \frac{\partial \widebar{u_{y}}}{\partial t} +
  \frac{\partial (\widebar{u_{y}}\widebar{u_{x}})}{\partial x} +
  \frac{\partial (\widebar{u_{y}}\widebar{u_{y}})}{\partial y} +
  \frac{\partial (\widebar{u_{y}}\widebar{u_{z}})}{\partial z} 
  &
  =
  -\frac{1}{\rho}\frac{\partial \widebar{p}}{\partial y} +
  \nu_{t}
  \left(
    \frac{\partial^{2} \widebar{u_{y}}}{\partial x^{2}} +
    \frac{\partial^{2} \widebar{u_{y}}}{\partial y^{2}} +
    \frac{\partial^{2} \widebar{u_{y}}}{\partial z^{2}}
  \right)
 \label{EqRaeMeySimp}
  \\
  \frac{\partial \widebar{p}}{\partial z} 
  =
  -\rho g
  \label{EqRaeMezSimp}
\end{align}
\label{EqRae}
\end{subequations}

\subsubsection{水深积分平均法则}
在河道水流中,水平尺度一般远大于垂向尺度,流速等水力参数沿垂向的变化常采用其垂向
平均值,并假定沿水深方向的动水压强分布符合静水压强分布。将式\ref{EqRae}沿水深积
分平均,即可得到沿水深平均的平面二维流动的基本方程。

在沿水深积分平均过程中,采用以下定义和公式:

(1)定义水深为
\begin{equation}
  H = \zeta - Z_{0}
\end{equation}
式中,$H$为水深,$\zeta=\zeta(x,y,t)$、$Z_{0}=Z_{0}(x,y,t)$分别为某一基准面下的
水面高程和河床高程(见图)

\begin{figure}
  \begin{tikzpicture}
  \end{tikzpicture}
  \caption{水位基准示意图}
\end{figure}

(2)定义沿水深平均流速$U_i$为
\begin{equation}
  U_i
  =
  \frac{1}{H}
\int_{Z_{0}}^{\zeta}\!\overline{u_{i}}\mathrm{d}z
\end{equation}
式中,下标$i$取1,2和3分别对应$x$,$y$和$z$方向的速度分量。

(3)莱布尼兹公式
\begin{equation}
  \frac{\partial}{\partial x_{i}}
  \int_{a}^{b}\!f\mathrm{d}z
  =
  \int_{a}^{b}\!
  \frac{\partial f}{\partial x_{i}}\mathrm{d}z
  +
  \left.f\right|_{b}\frac{\partial b}{\partial x_{i}}
  -
  \left.f\right|_{a}\frac{\partial a}{\partial x_{i}}
\end{equation}
式中,$a$、$b$和$f$都是$x_{i}$的函数。

(4)自由表面及河床底部运动学条件为:
\begin{equation}
  \left.\overline{u_{z}}\right|_{z=\zeta}
    =
    \frac{D\zeta}{Dt}
    =
    \frac{\partial\zeta}{\partial t}
    +
    \left.\overline{u_{x}}\right|_{z=\zeta}\frac{\partial\zeta}{\partial x}
      +
    \left.\overline{u_{y}}\right|_{z=\zeta}\frac{\partial\zeta}{\partial y}
\end{equation}
\begin{equation}
  \left.\overline{u_{z}}\right|_{z=Z_{0}}
    =
    \frac{DZ_{0}}{Dt}
    =
    \frac{\partial Z_{0}}{\partial t}
    +
    \left.\overline{u_{x}}\right|_{z=Z_{0}}\frac{\partial Z_{0}}{\partial x}
      +
    \left.\overline{u_{y}}\right|_{z=Z_{0}}\frac{\partial Z_{0}}{\partial y}
\end{equation}

\subsection{沿水深平均的连续性方程}
采用上述定义和公式对连续性方程\eqref{EqRaeCSimp}沿水深积分平均得:
\begin{equation*}
  \begin{aligned}
  &\int_{Z_{0}}^{\zeta}\!
  \left(
  \frac{\partial \widebar{u_{x}}}{\partial x} +
  \frac{\partial \widebar{u_{y}}}{\partial y} +
  \frac{\partial \widebar{u_{z}}}{\partial z} 
  \right)
  \mathrm{d}z
  \\
  =&
  \frac{\partial}{\partial x}
  \int_{Z_{0}}^{\zeta}\!
  \widebar{u_{x}}
  \mathrm{d}z
  -
  \left.\widebar{u_{x}}\right|_{z=\zeta}
    \frac{\partial\zeta}{\partial z}
  +
  \left.\widebar{u_{x}}\right|_{z=Z_{0}}
    \frac{\partial Z_{0}}{\partial z}
   + \\
  &
  \frac{\partial}{\partial y}
  \int_{Z_{0}}^{\zeta}\!
  \widebar{u_{y}}
  \mathrm{d}z
  -
  \left.\widebar{u_{y}}\right|_{z=\zeta}
    \frac{\partial\zeta}{\partial z}
  +
  \left.\widebar{u_{y}}\right|_{z=Z_{0}}
    \frac{\partial Z_{0}}{\partial z}
    + \\
  &
  \left.\widebar{u_{z}}\right|_{z=\zeta}
    -
    \left.\widebar{u_{z}}\right|_{z=Z_{0}}
      \\
  =&
  \frac{\partial HU_{x}}{\partial x} +
  \frac{\partial HU_{y}}{\partial x} +
  \frac{\partial \zeta}{\partial t} -
  \frac{\partial Z_{0}}{\partial t}
  =
  0
  \end{aligned}
\end{equation*}
最后得
\begin{equation}
  \frac{\partial H}{\partial t} +
  \frac{\partial HU_{x}}{\partial x} +
  \frac{\partial HU_{y}}{\partial y}
  =
  0
\end{equation}

\subsection{浅水运动方程}

\subsection{浅水方程形式}

\section{一维圣维南方程}

\subsection{一维连续性方程}

\subsection{一维运动方程}

\subsection{圣维南方程形式}
