\chapter{水动力学问题差分解法}
\section{一维非恒定流求解}
\subsection{矩形明渠一维圣维南方程组}
对矩形断面渠道,一维圣维南方程组的守恒形式为:
\begin{equation}
  \frac{\partial \mathbf{U}}{\partial t} +
  \frac{\partial \mathbf{F}}{\partial x} =
  \mathbf{S}
\end{equation}
其中,$\mathbf{U}$为未知量矢量,$\mathbf{F}$为通量矢量,$\mathbf{S}$为源项矢量,
分别为:

\begin{equation}
  \mathbf{U} = 
  \begin{bmatrix}
    h \\
    v
  \end{bmatrix}
  ,
  \mathbf{F} = 
  \begin{bmatrix}
    hv \\
    hv^{2} + \frac{1}{2}gh^{2}
  \end{bmatrix}
  ,
  \mathbf{S} = 
  \begin{bmatrix}
    0 \\
    ghi - C_{f}v|v|
  \end{bmatrix}
\end{equation}
式中,$v(x,t)$为断面平均流速,$h(x,t)$为断面水深,$i$为底坡,$g$为重力加速度,
$C_{f}$为阻力系数。

\subsection{初始条件}
为了计算圣维南方程的瞬态解,必须要先知道所有网格节点上的水深和流速值。这些值不能
随意给定,而应该按照实际物理情况来给定初始条件,通常来说有两种选择。
\begin{enumerate}
  \item 渠道处于静止状态。在该状态下,水面为水平线,所有节点上的速度都为零。干燥区域也可以被包含在计算域中。
  \item 渠道处于恒定流状态。在该状态下,水深和流速必须通过恒定流水流求解手段来获得。这个求解过程主要是求解渐变流水面曲线方程来获得。如果渠道中存在急变流段,还需要求解水跃跃前跃后水深关系来获得。
    \begin{equation}
    \frac{\mathrm{d}h}{\mathrm{d}x}
    =
    \frac{i - J }{1 - Fr^{2 }}
    \end{equation}
    \begin{equation}
      \frac{h_{2 }}{h_{1 }}
      =
      \frac{1}{2 }
      [(1+8{Fr}_{1}^{2})^{1/2} - 1]
    \end{equation}
\end{enumerate}

\subsection{边界条件}

\subsection{显式格式}

\subsubsection{FTCS格式}
\begin{equation}
  \mathbf{U}_{i}^{k+1} 
  =
  \mathbf{U}_{i}^{k} -
  \frac{\Delta t}{2\Delta x}(\mathbf{F}_{i+1}^{k}-\mathbf{F}_{i}^{k}) + 
  \mathbf{S}_{i}^{k}\Delta t
\end{equation}

\subsubsection{Lax扩散格式}
\begin{equation}
  \mathbf{U}_{i}^{k+1} 
  =
  \frac{(\mathbf{U}_{i-1}^{k} + \mathbf{U}_{i+1}^{k})}{2} -
  \frac{\Delta t}{2\Delta x}(\mathbf{F}_{i+1}^{k}-\mathbf{F}_{i}^{k}) + 
  \mathbf{S}_{i}^{k}\Delta t
\end{equation}

%\subsubsection{迎风格式}

\subsubsection{MacCormack预测校正格式}
MacCormack格式是一种两步(预测步和校正步)有限差分格式,具备激波捕捉能力。在预测
步,通量$\mathbf{F}$采用向前差分格式,用$k$时间层信息来计算。
\begin{equation}
  \mathbf{U}^{\mathrm{yc}} =
  \mathbf{U}^{k} -
  \frac{\Delta t }{\Delta x }
  (\mathbf{F}_{i}^{k} - \mathbf{F}_{i}^{k}) +
  \mathbf{S}_{i}^{k}\Delta t
\end{equation}
预测步给出了$k+1$时间层的流动的近似值,记为预测值。在校正步,通量$\mathbf{F}$采
用向后差分格式,用$k+1$时间层信息来计算。
\begin{equation}
  \mathbf{U}^{\mathrm{jz}} =
  \mathbf{U}^{k} -
  \frac{\Delta t }{\Delta x }
  (\mathbf{F}_{i}^{\mathrm{yc}} - \mathbf{F}_{i}^{\mathrm{yc}}) +
  \mathbf{S}_{i}^{\mathrm{yc}}\Delta t
\end{equation}
$k+1$时间层的最终结果是预测步和校正步得出的结果取平均得到的。
\begin{equation}
  \mathbf{U}^{k+1} =
  \mathbf{U}^{\mathrm{pj}} =
  \frac{1}{2}
  (\mathbf{U}^{\mathrm{yc}} +\mathbf{U}^{\mathrm{jz}})
\end{equation}

MacCormack格式在时间和空间上都是二阶精度,也是条件稳定的,因此受到CFL条件的限制。
根据Godunov理论,精度高于一阶的格式将在$\mathbf{U}$梯度较高的区域内产生非物理振
荡,比如在激波附近,例如溃坝波的波头。一阶精度格式虽然不会引入非物理振荡,但是过
度的扩散抹平了波头。因此,在实际使用中,二阶格式常被采用,控制非物理振荡的策略是
在局部区域内根据需要引入一定量的数值扩散。

\subsubsection{带人工粘性的MacCormack格式}
MacCormack格式中非物理振荡可以通过引入人工扩散来降低。在预测和校正步后,人工粘性
的引入是通过求解下面的附加方程
\begin{equation}
  \frac{\partial \mathbf{U}}{\partial t} =
  \varepsilon
  \frac{(\Delta x)^{2}}{\Delta t}
  \frac{{\partial}^{2}\mathbf{U}}{\partial x^{2}}
\end{equation}
式中,$\varepsilon$是一个扩散系数,用来控制格式中的耗散量。当$\varepsilon$取常数
时,上式中扩散项用二阶中心格式离散,可得
\begin{equation}
  \mathbf{U}_{i}^{k+1} =
  \mathbf{U}_{i}^{\mathrm{pj}} +
  \varepsilon
  (
  \mathbf{U}_{i+1}^{\mathrm{pj}} - 
  2\mathbf{U}_{i}^{\mathrm{pj}} +
  \mathbf{U}_{i-1}^{\mathrm{pj}}
  )
\end{equation}
基于这一思想,Jameson等人提出一种更容易实现的方法。上式改写为:
\begin{equation}
  \mathbf{U}_{i}^{k+1} =
  \mathbf{U}_{i}^{\mathrm{pj}} +
  {\varepsilon}_{i+1/2}
  (
  \mathbf{U}_{i+1}^{\mathrm{pj}} - 
  \mathbf{U}_{i}^{\mathrm{pj}}
  )
  -
  {\varepsilon}_{i-1/2}
  (
  \mathbf{U}_{i}^{\mathrm{pj}} -
  \mathbf{U}_{i-1}^{\mathrm{pj}}
  )
\end{equation}
其中,
\begin{equation}
  {\varepsilon}_{i+1/2} =
  K\max({\varepsilon}_{i+1}-{\varepsilon}_{i})
\end{equation}
${\varepsilon}_{i}$是基于水深计算得出的参数。
\begin{equation}
  {\varepsilon}_{i} =
  \frac{|h_{i+1}-h_{i}+h_{i-1}|}{|h_{i+1}|-|h_{i}|+|h_{i-1}|}
\end{equation}
$K$为人工粘性系数,需要校验,通常取值范围为0.5\textasciitilde 3。这一格式只在振荡发展区域内引入
数值扩散,而其他区域内的解不受影响。

\subsubsection{TVD MacCormack格式}

\subsection{隐式格式}


\section{二维泊松方程求解}

\section{二维输运扩散方程求解}
%\section{二维浅水方程求解}
%\subsection{网格基础}
%\subsection{求解算法}

