\chapter{绪论}
\section{引言}
计算水力学作为一门独立的学科,形成于20世纪60年代中期,它的出现标志着工程流体力学
发展的一个新阶段,是世界经济迅猛发展和科学技术水平不断提高,特别是高速、大容量电
子计算机的快速更新换代为计算技术的发展提供了雄厚技术基础的一个硕果。计算水力学已
经深入到几乎所有的技术科学领域,甚至某些社会科学方面。

自19世纪Froude奠定了物理比尺模型的理论基础,并首次应用于船舶阻力的比尺模型实验以
后,物理比尺模型曾经是验证流体力学理论和研究工程水力学问题的唯一手段。但是随着工
程的不断大型化,以及其他生物、化学过程的引入,物理比尺模型难以应用。数学模型也就
应运而生了。

数学模型是将已知的水动力学及以水为载体的物质的传输过程的基本定律用数学方程进行描
述。在一定的定解条件(初始条件和边界条件)下求解这些数学方程,从而达到模拟某个水
动力学及以水为载体的物质输运过程的理论问题或工程实际问题。传统的水力计算也是一种
数学模拟,但由于只涉及初等数学理论,能解决的问题较简单,还不能算是真正的数学模型
。在20实际50年代以后,数学模拟的基本理论已经建立,运用这些理论也曾解决过一些工程
问题。但是,这些基本理论的真正应用是在电子计算机发明以后。1952~1954年Isaacson和
Troesch首次建立了俄亥俄河和密西西比河的部分河段的数学模型,并进行了实际洪水过程
的模拟。但这以后,这方面的研究并没有得到发展。直到20世纪60年代中期,为了解决各种
各样的设计和规划问题,数学模型再次得到重视。人们针对个别的工程问题建立了大量用途
单一的数学模型并编制了相应的计算程序,这些模型可以用于解决一系列的工程问题。在此
工作的基础上,70年代中期,一种功能更加完善的系统数学模型出现了。这种系统模型综合
了过去的用途单一的模型的功能,可以对整个流域、河湖、近海、建成的或规划中的水利工
程等进行系统的模拟。这种系统模型的功能已远远超过了物理比尺模型。

计算水力学可以说是研究如何用计算机进行实验的科学。正确处理的边界条件相当于物理比
尺模型中的固体边界与自由边界;数学模型中用以表征流体性质的参数,如粘度,密度等,
则相当于物理比尺模型中所用的试验流体;而通过数值运算求得的计算域内每一个特定的空
间点处的待求函数值,如速度、压强,浓度等,则与物理比尺模型中利用测量仪器测得的值
相对应。相对于物理比尺模型,数学模拟的优点有:
\begin{itemize}
  \item 可以完全自由地改变或控制流体的性质;
  \item 可以进行严格的一维、二维和三维流动的试验;
  \item 可以完全自由地选择流体参数,如初始速度剖面等;
  \item 可以进行全尺度试验,不存在比尺效应;
  \item 可以进行物理和生物化学现象的各种理论近似处理的有效性和敏感性实验;
  \item 可以验证流体的本构方程的可靠性,如非牛顿流体的本构方程的验证。
\end{itemize}

数值模型所依据的是数学方程的离散形式,计算区域和方程的离散不仅引起量上的误差,而
且处理不当常常改变了方程的性质。尽管目前多种紊流模型已被引入使用的计算,但都是基
于一系列的假定的技术上得出的,模型越复杂,引入的实验常数越多,给出的也只是紊动的
平均效应。紊动的结构及其基本特性仍是数学模型难以模拟的,必须依赖于物理模型的试验
。所以、数值模型不能完全替代物理实验,更不能替代理论分析。实际上,理论分析、数学
模型和物理实验的紧密结合才是解决理论问题和实际工程问题的有效途径。

计算水力学是涉及经典水力学、计算方程、数值分析、程序编制和资料处理等学科的一门综
合性的交叉学科。它既不是水力学、也不是数值分析,具有自身的概念与特点,有其确定的
应用领域。计算水力学用的偏微分方程组,但更注重方程中每一项所代表的物理意义,在什
么条件下可以略去方程中的某些项而仍能确切地模拟某些特定的水力现象。方程中涉及的物
理、生化参数的正确确定,边界条件的合理给定,均需要对所模拟的水动力学现象及以水为
载体的物质的传输过程的物理和生化本质有深刻的理解。为了证明一个数学模型是水动力学
现象及以水为载体的物质的传输过程的可靠模型,则又需要利用数学分析中的线性理论,对
所用数学模式的收敛性、相容性和稳定性进行证明。而计算结果的质量又与资料的收集、整
理和正确利用紧密相关,这往往取决于从事数学模拟的人的经验与技巧。

\section{课程内容}
本课程的主要内容分为三个部分:第二章主要介绍各类数学模型,包括三维纳维-斯托克斯
方程,雷诺平均方程,二维浅水方程和一维圣维南方程组的推导;第三章主要介绍有限差分
法;第四章主要介绍线性方程组的求解方法。
