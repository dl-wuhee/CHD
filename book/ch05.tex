\chapter{代数方程组的求解方法}
\section{三对角矩阵直接求解方法}
为方便讨论,将方程组\eqref{EqBD_1dht_ia}改写为
  \begin{equation}
    A_{i}T_{i} = B_{i}T_{i-1} + C_{i}T_{i+1} + D_{i}
    \label{EqLA_TDMA_array}
  \end{equation}
  假设共有$N$个方程,即$i=1,2,\cdots,N$。当$i=1$时,$B_{i}=0$;当$i=N$时,
  $C_{i}=0$。

  这个方程组的求解过程分为消元和回代两个步骤。消元过程从系数矩阵的第二行开始。设
  消元过程完成后的方程表示为:
  \begin{equation}
    T_{i} = P_{i}T_{i+1}+Q_{i}
    \label{EqLA_xiaoyuan_1}
  \end{equation}
  或
  \begin{equation}
    T_{i-1} = P_{i-1}T_{i} + Q_{i-1}
    \label{EqLA_xiaoyuan_2}
  \end{equation}
  将式\eqref{EqLA_xiaoyuan_2}乘以$B_{i}$后,与式\eqref{EqLA_TDMA_array}相加,可
  得
 \begin{equation}
   A_{i}T_{i} + B_{i}T_{i-1} = B_{i}T_{i-1}+C_{i}T_{i+1}+D_{i} +
   B_{i}P_{i-1}T_{i} + B_{i}Q_{i-1}
 \end{equation} 
整理的
\begin{equation}
  T_{i} = 
  \frac{C_{i}}{A_{i}-B_{i}P_{i-1}}T_{i+1} +
  \frac{D_{i}+B_{i}Q_{i-1}}{A_{i}-B_{i}P_{i-1}}
\end{equation}
将上式与式\eqref{EqLA_xiaoyuan_1}对比,可得
\begin{equation}
\begin{aligned}
  P_{i} 
  &=
\frac{C_{i}}{A_{i}-B_{i}P_{i-1}} \\
Q_{i}
  &=
  \frac{D_{i}+B_{i}Q_{i-1}}{A_{i}-B_{i}P_{i-1}}
\end{aligned}
\end{equation}
这是一个递推关系式,需要确定$P_{1}$和$Q_{1}$。

当$i=1$时,式\eqref{EqLA_TDMA_array}为
\begin{equation}
  A_{1}T_{1} = B_{1}T_{0} + C_{1}T_{2} + D_{1}
\end{equation}
其中,$B_{1}=0$。对比式\eqref{EqLA_xiaoyuan_1},可得
\begin{equation}
  P_{1} = \frac{C_{1}}{A_{1}},\quad
  Q_{1} = \frac{D_{1}}{A_{1}}
\end{equation}
在$P_{i}$和$Q_{i}$序列的另一端,消元进入最后一行时,由式\eqref{EqLA_xiaoyuan_1}
得
\begin{equation}
  T_{N} = P_{N}T_{N+1}+Q_{N}
\end{equation}
且$P_{N}=0$,得
\begin{equation}
  T_{N} = Q_{N}
\end{equation}
到此消元过程结束,然后按照式\eqref{EqLA_xiaoyuan_2}进行回代。

下面给出TDMA方法的计算步骤:
\begin{enumerate}
  \item 计算系数$P_{i}$和$Q_{i}$
    \begin{equation*}
      P_{1} = \frac{C_{1}}{A_{1}}, \quad Q_{1} = \frac{D_{1}}{A_{1}}
    \end{equation*}
  \item 对$i=1,2,\cdots,N$用递推关系式求系数$P_{i}$和$Q_{i}$
    \begin{equation*}
      P_{i} 
      =
      \frac{C_{i}}{A_{i}-B_{i}P_{i-1}}
      ,\quad
      Q_{i}
      =
      \frac{D_{i}+B_{i}Q_{i-1}}{A_{i}-B_{i}P_{i-1}}
    \end{equation*}
  \item 令$T_{N}=Q_{N}$
  \item 对$i=N-1,N-2,\cdots,2,1$应用回代得$T_{N-1},T_{N-2},\cdots,T_{2},T_{1}$
    \begin{equation*}
      T_{i-1}=P_{i-1}T_{i} + Q_{i-1}
    \end{equation*}
\end{enumerate}

\section{代数方程组迭代求解方法}
下面用一个简单的例子来介绍代数方程组迭代求解方法中点迭代方法。
考虑一个简单的三元方程组为例,
\begin{equation}
  \sysdelim..
  \systeme{
    2x_{1} + x_{2} + x_{3} = 7,
    -x_{1} + 3x_{2} -x_{3} = 2,
    x_{1} - x_{2} + 2x_{3} = 5
  }
\end{equation}
首先,对第一个方程进行移项,使等号左边仅保留$x_{1}$,第二个方程的等号左边仅保留
$x_{2}$,依次类推。可得
\begin{equation}
\begin{aligned}
    x_{1} =(7-x_{2}-x_{3})/2 \\
    x_{2} =(2 +x_{1} +x_{3} )/ 3 \\
    x_{3} =(5- x_{1} + x_{2} )/ 2
\end{aligned}
\label{EqLA_pointiteration}
\end{equation}
上式可以通过将$x_{1}$,$x_{2}$和$x_{3}$的假定初始值带入后迭代求解得到。假定初始
值带入上式右侧后计算出新的$x_{1}$,$x_{2}$和$x_{3}$。然后再将新值带入,直到
整个过程收敛到整个代数方程组的真解。

需要注意的是,不是所有的代数方程组都可以采用这种迭代方法得到收敛解的。这种迭代方
法收敛的一个条件是线性方程组的系数矩阵必须是对角占优的,即每一行的对角线元素
$|a_{ii}|>\sum |a_{ij}|(i\ne j)$。因此,在采用迭代方法来求解线性方程组的时候可以
对方程组进行重排来确保对角占优。

雅可比迭代、高斯赛德尔迭代和松弛迭代的主要差别在带入右侧的值存在细微的差别。

\subsection{雅可比迭代法}
在雅可比迭代中,第$i$次迭代等式左侧的值$x_{1}^{k}, x_{2}^{k},x_{3}^{k}$通过将
$k-1$次迭代获得的$x_{1}^{k-1}, x_{2}^{k-1},x_{3}^{k-1}$带入等式右侧来获得。以上
面给出的线性方程组,第一步先假定$x_{1}^{0} = x_{2}^{0} = x_{3}^{0}=0$。将假定值
带入方程右侧,可得
\begin{equation*}
x_{1}^{1}=3.500,\quad x_{2}^{1}=0.667,\quad x_{3}^{1}=2.500
\end{equation*}
重复该过程,最后得到值见表\ref{TbLA_Jacobi_result}。经过17次迭代后,
$x_{1}=1.000,x_{2}=2.000,x_{3}=3.000$,并且结果在增加迭代次数后不会有新的变化。
与该方程组的解析解对比可知,迭代得出的保留四位有效数值的迭代解就是精确解。
\begin{table}[h!]
  \begin{center}
  \caption{雅可比迭代结果}
  \label{TbLA_Jacobi_result}
  \begin{tabular}{|c|r|r|r|r|r|r|r|r|}
    \hline
    迭代次数 & 0 & 1 & 2 & 3 & 4 & 5 & $\cdots$ & 17 \\
    \hline
    $x_{1}$ & 0 & 3.5000 & 1.9167 & 1.6250 & 1.2292 & 1.1563 & $\cdots$ & 1.000
    \\
    \hline
    $x_{2}$ & 0 & 0.6667 & 2.6667 & 1.6667 & 2.1667 & 1.9167 & $\cdots$ & 2.000
    \\
    \hline
    $x_{3}$ & 0 & 2.5000 & 1.0833 & 2.8750 & 2.5208 & 2.9688 & $\cdots$ & 3.000
    \\
    \hline
  \end{tabular}
  \end{center}
\end{table}

对一个包含$n$个方程和$n$各未知量的线性方程组,$\mathbf{A}\cdot \mathbf{x} =
\mathbf{b}$,或写成如下形式
\begin{equation}
  \sum_{j=1}^{n}a_{ij}x_{j} = b_{i}
\end{equation}

按雅可比迭代对上式进行移项,等式左侧仅保留与$x_{i}$相关的项,其他项都移到等式右
侧,得
\begin{equation}
  a_{ii}x_{i} = b_{i} - 
  \sum_{\substack{j=1\\j\ne i}}^{n}a_{ij}x_{j} (i=1,2,\cdots,n)
\end{equation}
等式两侧同除以$a_{ii}$,并用上标$k$来标识等式左侧为第$k$次迭代的值,上标${k-1}$
来表示等式右侧为第$k-1$次迭代的值。
\begin{equation}
  x_{i}^{k} 
  =
  \sum_{\substack{j=1\\ j\ne i}}^{n}
  \left(
    \frac{-a_{ij}}{a_{ii}}
  \right)
  x_{j}^{k-1}
  +
  \frac{b_{i}}{a_{ii}} 
  \quad
  (i=1,2,\cdots,n)
\end{equation}
该式也可写成矩阵形式:
\begin{equation}
  \mathbf{x}^{k} 
  =
  \mathbf{T}\cdot\mathbf{x}^{k-1} + \mathbf{c}
\end{equation}
其中,$\mathbf{T}$是迭代矩阵,其定义为
\begin{equation}
  T_{ij} 
  =
  \begin{cases}
    \displaystyle
    -\frac{a_{ij}}{a_{ii}} \quad & i\ne j 
    \\
    0 \quad & i=j
  \end{cases}
\end{equation}
$\mathbf{c}$为常数向量,
\begin{equation}
  c_{i} = \frac{b_{i}}{a_{ii}}
\end{equation}

\subsection{高斯赛德尔迭代}
在雅可比迭代中,等式右侧采上一个迭代的结果或初始值来进行计算。迭代方程求解过程中
基本都是按照方程顺序来进行的。还是以上面的方程组为例。第一个迭代过程中,
$x_{1}^{0}=0,x_{2}^{0},x_{3}^{0}$。带入第一个方程,得
\begin{equation}
  x_{1}^{1} = (7-x_{2}^{0}-x_{3}^{0})/2 = (7-0-0)/2=3.5
\end{equation}
接下来,求解第二个方程,$x_{2}=(2+x_{1}+x_{3})/3$。注意,等号左侧的$x_{1}$。在雅
可比迭代中,$x_{1}$还是用$x_{1}^{0}$来代入。然而,我们在求解第一个方程后,
$x_{1}$的值已经更新为$x_{1}^{1}$了。高斯赛德尔迭代就是直接使用已经更新的值代入计
算,即
\begin{equation}
  x_{2}^{1} = (2+x_{1}^{1}+x_{3}^{0})/3=(2+3.5+0)/3 = 1.8333
\end{equation}
在求解第三个方程$x_{3}=(5-x_{1}+x_{2})/2$时,用$x_{1}^{1}$和$x_{2}^{1}$代入计算
\begin{equation}
  x_{3}^{1} = (5-x_{1}^{1}+x_{2}^{2})/2=(5-3.5+1.8333)/2=1.6667
\end{equation}
重复这一过程,直至得到收敛解。计算过程如表\ref{TbLA_Gauss_result}所示。采用高斯
赛德尔迭代只需要13次迭代就可以收敛。通常来说,高斯赛德尔迭代方法比雅可比迭代方法
更快收敛。
\begin{table}[h!]
  \begin{center}
  \caption{雅可比迭代结果}
  \label{TbLA_Gauss_result}
  \begin{tabular}{|c|r|r|r|r|r|r|r|r|}
    \hline
    迭代次数 & 0 & 1 & 2 & 3 & 4 & 5 & $\cdots$ & 13 \\
    \hline
    $x_{1}$ & 0 & 3.5000 & 1.7500 & 1.3333 & 1.1181 & 1.0475 & $\cdots$ & 1.000
    \\
    \hline
    $x_{2}$ & 0 & 1.8333 & 1.8056 & 1.9537 & 1.9761 & 1.9922 & $\cdots$ & 2.000
    \\
    \hline
    $x_{3}$ & 0 & 1.6667 & 2.5278 & 2.8102 & 2.9290 & 2.9724 & $\cdots$ & 3.000
    \\
    \hline
  \end{tabular}
  \end{center}
\end{table}

从上面的列子容易得出通用的高斯赛德尔迭代,即:
\begin{equation}
  x_{i}^{k}
  =
  \sum_{j=1}^{i-1}
  \left(
    \frac{-a_{ij}}{a_{ii}}
  \right)
  x_{j}^{k}
  +
  \sum_{j=i+1}^{n}
  \left(
    \frac{-a_{ij}}{a_{ii}}
  \right)
  x_{j}^{k-1}
  +
  \frac{b_{i}}{a_{ii}}
  \quad
  (i=1,2,\cdots,n)
\end{equation}
写成矩阵形式,有
\begin{equation}
  \mathbf{x}^{k} 
  =
  \mathbf{T}_{1}\mathbf{x}^{k}
  +
  \mathbf{T}_{2}\mathbf{x}^{k-1}
  +
  \mathbf{c}
\end{equation}
其中,系数矩阵$T_{1}$和$T_{2}$的定义如下:
\begin{equation}
  {T_{1}}_{ij} 
  =
  \begin{cases}
    \displaystyle
    -\frac{a_{ij}}{a_{ii}} & i>j \\
    0 & i\le j
  \end{cases}
\end{equation}

\begin{equation}
  {T_{2}}_{ij} 
  =
  \begin{cases}
    0 & i\ge j
    \\
    \displaystyle
    -\frac{a_{ij}}{a_{ii}} & i<j 
  \end{cases}
\end{equation}

\subsection{松弛迭代}
仔细观察高斯赛德尔迭代的通用计算公式,可以看出
\begin{equation}
  x_{i}^{k}
  =
  x_{i}^{k-1}
  +
  \sum_{j=1}^{i-1}
  \left(
    \frac{-a_{ij}}{a_{ii}}
  \right)
  x_{j}^{k}
  +
  \sum_{j=i}^{n}
  \left(
    \frac{-a_{ij}}{a_{ii}}
  \right)
  x_{j}^{k-1}
  +
  \frac{b_{i}}{a_{ii}}
  \quad
  (i=1,2,\cdots,n)
\end{equation}
当我们引入一个松弛系数$\alpha$,可得
\begin{equation}
  x_{i}^{k}
  =
  x_{i}^{k-1}
  +
  \alpha
  \left[
  \sum_{j=1}^{i-1}
  \left(
    \frac{-a_{ij}}{a_{ii}}
  \right)
  x_{j}^{k}
  +
  \sum_{j=i}^{n}
  \left(
    \frac{-a_{ij}}{a_{ii}}
  \right)
  x_{j}^{k-1}
  +
  \frac{b_{i}}{a_{ii}}
    \right]
  \quad
  (i=1,2,\cdots,n)
\end{equation}
通过合理选择$\alpha$的取值,可以起到加速收敛的效果。
%\section{多重网格法}
