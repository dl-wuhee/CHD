\chapter{有限体积法}
有限体积法的基本思想是:把计算域分成许多互不重叠的控制体或控制容积,然后在每一个
控制容积上将微分方程进行积分,用表示网格节点之间的分段分布关系来计算所要求的积分
,进而得到一个仅包含网格节点处$\phi$值表示的离散化线性方程组。

\section{稳态传导方程的有限体积法}
以某一运动要素$\phi$的一维稳态传导问题为例:
\begin{equation}
  \frac{\mathrm{d}}{\mathrm{d} x}(\Gamma \frac{\mathrm{d} \phi}{\mathrm{d} x}) +
  S = 0
\end{equation}
其中,$\Gamma$为传导系数,$S$为源项。在边界点上$\phi$的值给定。这类问题的一个例
子就是第三章讨论的一根金属棒上一维热传导问题。

\subsection{步骤一:网格生成}
有限体积法的第一步是将计算区域划分为互不重叠的离散控制体。在A和B之间均匀的布置一
系列的节点。每个控制体的边界位于相邻节点的中线处。这样,每个节点都被一个控制体或
控制单元所包围。在计算域边界处设置控制体是比较常见的做法,这样可以让控制体的边界
与物理边界重叠。

首先,我们建立有限体积法网格系统的符号系统。图\ref{}给出了计算域中网格系统的示意
图。图中$P$为网格系统中的任意节点,与它相邻的西侧和东侧节点分别为$W$和$E$。节点
$P$所在控制体西边的边界面为$w$,东边的边界面为$e$。$W$节点与$P$节点的距离为
$\delta x_{WP}$,$P$节点与$E$节点的距离为$\delta x_{PE}$。控制体边界面$w$到$P$节
点的距离为$\delta x_{wP}$,$P$节点到边界面$e$的距离为$\delta x_{Pe}$,边界面$w$
到边界面$e$的距离为$\delta x_{we}$。

\subsection{步骤二:离散}
有限体积法的关键步骤是在控制体上对控制方程积分来得到控制体节点$P$上的离散方程。
对上面建立的网格系统,
\begin{equation}
  \int_{\Delta V}\!
  \frac{\mathrm{d} }{\mathrm{d} x}
  \left(
    \Gamma \frac{\mathrm{d} \phi}{\mathrm{d} x}
  \right)
  \mathrm{d}V
  +
  \int_{\Delta V}\!
  S
  \mathrm{d}V
  =
  \left(
    \Gamma A\frac{\mathrm{d} \phi}{\mathrm{d} x}
  \right)_{e}
  -
  \left(
    \Gamma A\frac{\mathrm{d} \phi}{\mathrm{d} x}
  \right)_{w}
  +
  \overline{S}\Delta V
  =
  0
  \label{EqFV_Diffusion_Discretisation}
\end{equation}
式中,$A$为控制体边界面的面积,$\Delta V$为控制体体积,$\overline{S}$为控制体上
$S$的平均值。有限体积法的一个非常吸引人的特性是离散方程具有明确的物理意义。式
\eqref{EqFV_Diffusion_Discretisation}表明:流出东边交界面的$\phi$的扩散通量减去流
入西边交界面的$\phi$的扩散通量等于$\phi$的减少量。

为了推导出可用的离散方程,式\eqref{EqFV_Diffusion_Discretisation}中控制体交界面上的
$\Gamma$和梯度$\mathrm{d}\phi/\mathrm{d}x$必须要先求得。
\begin{subequations}
  \begin{align}
  \Gamma_{w} 
  &=
  \frac{\Gamma_{W}+\Gamma_{P}}{2}
  \\
  \Gamma_{e} 
  &=
  \frac{\Gamma_{P}+\Gamma_{E}}{2}
  \end{align}
\end{subequations}
\begin{subequations}
  \begin{align}
  &\left(
    \Gamma A\frac{\mathrm{d} \phi}{\mathrm{d} x}
  \right)_{e}
  =
  \Gamma_{e}A_{e}
  \left(
    \frac{\phi_{E}-\phi_{P}}{\delta x_{PE}}
  \right)
    \\
  &\left(
    \Gamma A\frac{\mathrm{d} \phi}{\mathrm{d} x}
  \right)_{w}
  =
  \Gamma_{w}A_{w}
  \left(
    \frac{\phi_{P}-\phi_{W}}{\delta x_{WP}}
  \right)
  \end{align}
\end{subequations}
\begin{equation}
  \overline{S}\Delta V = S_{u} + S_{P}\phi_{P}
\end{equation}
\begin{equation}
  \Gamma_{e}A_{e}
  \left(
    \frac{\phi_{E}-\phi_{P}}{\delta x_{PE}}
  \right)
  -
  \Gamma_{w}A_{w}
  \left(
    \frac{\phi_{P}-\phi_{W}}{\delta x_{WP}}
  \right)
  +
  (S_{u} + S_{P}\phi_{P})
  =
  0
\end{equation}
\begin{equation}
  \left(
    \frac{\Gamma}{\delta x_{PE}}A_{e}
    +
    \frac{\Gamma}{\delta x_{WP}}A_{w}
    -
    S_{p}
  \right)
  \phi_{P}
  =
  \left(
    \frac{\Gamma_{w}}{\delta x_{WP}}A_{w}
  \right)\phi_{W}
  +
  \left(
    \frac{\Gamma_{e}}{\delta x_{PE}}A_{e}
  \right)\phi_{E}
  +
  S_{u}
\end{equation}
\begin{equation}
  a_{P}\phi_{P} = a_{W}\phi_{W} + a_{E}\phi_{E}+S_{u}
  \label{EqFV_1dsd_fvm}
\end{equation}
其中
\begin{table}[H]
  \begin{center}
  %\caption{雅可比迭代结果}
  \label{TbFV_diffusion_coefficient}
  \begin{tabular}{|c|c|c|}
    \hline
    $a_{W}$ & $a_{E}$ & $a_{P}$
    \\
    \hline
    \makecell*[c]{
    $\displaystyle \frac{\Gamma_{w}}{\delta x_{WP}}A_{w}$
  }
            &
    $\displaystyle \frac{\Gamma_{e}}{\delta x_{PE}}A_{e}$
            &
    $a_{W} + a_{E} - S_{P}$
    \\
    \hline
  \end{tabular}
  \end{center}
\end{table}

\subsection{步骤三、求解}
式\eqref{EqFV_1dsd_fvm}必须在所有控制体的节点上都列出才能求解。对于毗邻计算域边
界的控制体,式\eqref{EqFV_1dsd_fvm}必须经过适当修正以包含边界条件。最后形成的线
性代数方程组可以通过上一章的求解方法来进行求解得到$\phi$的分布。

\section{示例:一维稳态热传导问题求解}
\subsection{例1:无源一维稳态热传导问题}
如图\ref{FgFV_1dsd_ex1}考虑一根等直径的绝热金属棒,两端温度分布保持在
$100^{\circ}\mathrm{C}$和$500^{\circ}\mathrm{C}$。该金属棒的导热系数为常数$k$,
截面面积为$A$。该问题的控制方程为:
\begin{equation}
\frac{\mathrm{d} }{\mathrm{d} x}
\left(
k\frac{\mathrm{d} T}{\mathrm{d} x}
\right)
=
0
\label{EqFV_1dsd_ex1_gov}
\end{equation}

首先将该金属棒分成5个体积相等的控制体,如图\ref{FgFV_1dsd_ex1_grid}所示。网格间
距$\delta x=0.1\mathrm{m}$。整个网格系统包括5个节点。对节点2,3和4,
\begin{equation}
  \left(
    \frac{k}{\delta x_{PE}}A_{e}
    +
    \frac{k}{\delta x_{WP}}A_{w}
  \right)
  T_{P}
  =
  \left(
    \frac{k}{\delta x_{WP}}A_{w}
  \right)
  T_{W}
  +
  \left(
    \frac{k}{\delta x_{PE}}A_{e}
  \right)
  T_{E}
\end{equation}
由于导热系数,截面面积和网格间距为常数,因此上式可以写出:
\begin{equation}
  a_{P}T_{P} = a_{W}T_{W} + a_{E}T_{E}
\end{equation}
其中
\begin{table}[H]
  \begin{center}
  %\caption{雅可比迭代结果}
  \label{TbFV_diffusion_coefficient_ex1_n234}
  \begin{tabular}{|c|c|c|}
    \hline
    $a_{W}$ & $a_{E}$ & $a_{P}$
    \\
    \hline
    \makecell*[c]{
    $\displaystyle \frac{k}{\delta x}A$
  }
            &
    $\displaystyle \frac{k}{\delta x}A$
            &
    $a_{W} + a_{E}$
    \\
    \hline
  \end{tabular}
  \end{center}
\end{table}
对节点1,将式\eqref{EqFV_1dsd_ex1_gov}在包围节点1的控制体上
积分,可得
\begin{equation}
kA
\left(
  \frac{T_{E}-T_{P}}{\delta x}
\right)
-
kA
\left(
  \frac{T_{P}-T_{A}}{\delta x/2}
\right)
=
0
\end{equation}
对上式进行移项,可得
\begin{equation}
  \left(
    \frac{k}{\delta x}A
    +
    \frac{2k}{\delta x}A
  \right)
  T_{P}
  =
  0\cdot T_{W}
  +
  \left(
    \frac{k}{\delta x}A
  \right)
  T_{E}
  +
  \left(
    \frac{2k}{\delta x}A
  \right)
  T_{A}
\end{equation}
写成离散形式:
\begin{equation}
  a_{P}T_{P}
  =
  a_{W}T_{W}
  +
  a_{E}T_{E}
  +
  S_{u}
\end{equation}
其中
\begin{table}[H]
  \begin{center}
  %\caption{雅可比迭代结果}
  \label{TbFV_diffusion_coefficient_ex1_n1}
  \begin{tabular}{|c|c|c|c|c|}
    \hline
    $a_{W}$ & $a_{E}$ & $a_{P}$ & $S_{p}$ & $S_{u}$
    \\
    \hline
    0
            &
    \makecell*[c]{
    $\displaystyle \frac{kA}{\delta x}$
  }
            &
          $a_{W}+a_{E}-S_{p}$
            &
    \makecell*[c]{
    $\displaystyle -\frac{2kA}{\delta x}$
  }
  &
    \makecell*[c]{
      $\displaystyle \frac{2kA}{\delta x}T_{A}$
  }
    \\
    \hline
  \end{tabular}
  \end{center}
\end{table}


对节点5,将式\eqref{EqFV_1dsd_ex1_gov}在包围节点5的控制体上
积分,可得
\begin{equation}
kA
\left(
  \frac{T_{B}-T_{P}}{\delta x/2}
\right)
-
kA
\left(
  \frac{T_{P}-T_{W}}{\delta x}
\right)
=
0
\end{equation}
对上式进行移项,可得
\begin{equation}
  \left(
    \frac{k}{\delta x}A
    +
    \frac{2k}{\delta x}A
  \right)
  T_{P}
  =
  \left(
    \frac{k}{\delta x}A
  \right)
  T_{W}
  +
  0\cdot T_{E}
  +
  \left(
    \frac{2k}{\delta x}A
  \right)
  T_{B}
\end{equation}
写成离散形式:
\begin{equation}
  a_{P}T_{P}
  =
  a_{W}T_{W}
  +
  a_{E}T_{E}
  +
  S_{u}
\end{equation}
其中
\begin{table}[H]
  \begin{center}
  %\caption{雅可比迭代结果}
  \label{TbFV_diffusion_coefficient_ex1_n1}
  \begin{tabular}{|c|c|c|c|c|}
    \hline
    $a_{W}$ & $a_{E}$ & $a_{P}$ & $S_{p}$ & $S_{u}$
    \\
    \hline
    \makecell*[c]{
    $\displaystyle \frac{kA}{\delta x}$
  }
            &
    0
            &
          $a_{W}+a_{E}-S_{p}$
            &
    \makecell*[c]{
    $\displaystyle -\frac{2kA}{\delta x}$
  }
  &
    \makecell*[c]{
      $\displaystyle \frac{2kA}{\delta x}T_{B}$
  }
    \\
    \hline
  \end{tabular}
  \end{center}
\end{table}
将$kA/\delta x=100$,$T_{A}=100$和$T_{B}=500$代入5个节点的离散方程,可得
\begin{table}[H]
  \begin{center}
  %\caption{雅可比迭代结果}
  \label{TbFV_diffusion_coefficient_ex1_coeff}
  \begin{tabular}{|c|c|c|c|c|c|}
    \hline
    节点 & $a_{W}$ & $a_{E}$ & $S_{p}$ & $S_{u}$ & $a_{P}$ \\
    \hline
    1 & 0 & 100 & -200 & 20000 & 300  \\
    \hline
    2 & 100 & 100 & 0 & 0 & 200  \\
    \hline
    3 & 100 & 100 & 0 & 0 & 200 \\
    \hline
    4 & 100 & 100 & 0 & 0 & 200  \\
    \hline
    5 & 100 & 0 & -200 & 100000 & 300 \\ 
    \hline
  \end{tabular}
  \end{center}
\end{table}
最后形成的线性方程组为:
\begin{equation}
  \begin{bmatrix}
    300 & -100 & 0 & 0 & 0 \\
    -100 & 200 & -100 & 0 & 0 \\
    0 & -100 & 200 & -100 & 0 \\
    0 & 0 & -100 & 200 & -100  \\
    0 & 0 & 0 & -100 & 300 \\
  \end{bmatrix}
  \begin{bmatrix}
    T_{1} \\
    T_{2} \\
    T_{3} \\
    T_{4} \\
    T_{5} \\
  \end{bmatrix}
  =
  \begin{bmatrix}
    20000 \\
    0 \\
    0 \\
    0 \\
    100000 \\
  \end{bmatrix}
\end{equation}
通过上一章介绍的TDMA方法可以求得:
\begin{equation}
  \begin{bmatrix}
    T_{1} \\
    T_{2} \\
    T_{3} \\
    T_{4} \\
    T_{5} \\
  \end{bmatrix}
  =
  \begin{bmatrix}
    140 \\
    220 \\
    300 \\
    380 \\
    460 \\
  \end{bmatrix}
\end{equation}

\section{稳态对热扩散方程的有限体积法}

\section{非稳态方程的有限体积法}

