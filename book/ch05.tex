\chapter{有限体积法}
有限体积法的基本思想是:把计算域分成许多互不重叠的控制体或控制容积,然后在每一个
控制容积上将微分方程进行积分,用表示网格节点之间的分段分布关系来计算所要求的积分
,进而来到一个仅包含网格节点处$\phi$值表示的离散化线性方程组。

\section{稳态传导方程的有限体积法}
以某一运动要素$\phi$的一维稳态传导问题为例:
\begin{equation}
  \frac{\mathrm{d}}{\mathrm{d} x}(\Gamma \frac{\mathrm{d} \phi}{\mathrm{d} x}) +
  S = 0
\end{equation}
其中,$\Gamma$为传导系数,$S$为源项。在边界点上$\phi$的值给定。这类问题的一个例
子就是第三章讨论的一根金属棒上一维热传导问题。

\subsection{步骤一:网格生成}
有限体积法的第一步是将计算区域划分为互不重叠的离散控制体。在A和B之间均匀的布置一
系列的节点。每个控制体的边界位于相邻节点的中线处。这样,每个节点都被一个控制体或
控制单元所包围。在计算域边界处设置控制体是比较常见的做法,这样可以让控制体的边界
与物理边界重叠。

首先,我们建立有限体积法网格系统的符号系统。图\ref{}给出了计算域中网格系统的示意
图。图中$P$为网格系统中的任意节点,与它相邻的
\section{稳态对热扩散方程的有限体积法}

\section{非稳态方程的有限体积法}

