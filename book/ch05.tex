\chapter{代数方程组的求解方法}
\section{三对角矩阵直接求解方法}
为方便讨论,将方程组\eqref{EqBD_1dht_ia}改写为
  \begin{equation}
    A_{i}T_{i} = B_{i}T_{i-1} + C_{i}T_{i+1} + D_{i}
    \label{EqLA_TDMA_array}
  \end{equation}
  假设共有$N$个方程,即$i=1,2,\cdots,N$。当$i=1$时,$B_{i}=0$;当$i=N$时,
  $C_{i}=0$。

  这个方程组的求解过程分为消元和回代两个步骤。消元过程从系数矩阵的第二行开始。设
  消元过程完成后的方程表示为:
  \begin{equation}
    T_{i} = P_{i}T_{i+1}+Q_{i}
    \label{EqLA_xiaoyuan_1}
  \end{equation}
  或
  \begin{equation}
    T_{i-1} = P_{i-1}T_{i} + Q_{i-1}
    \label{EqLA_xiaoyuan_2}
  \end{equation}
  将式\eqref{EqLA_xiaoyuan_2}乘以$B_{i}$后,与式\eqref{EqLA_TDMA_array}相加,可
  得
 \begin{equation}
   A_{i}T_{i} + B_{i}T_{i-1} = B_{i}T_{i-1}+C_{i}T_{i+1}+D_{i} +
   B_{i}P_{i-1}T_{i} + B_{i}Q_{i-1}
 \end{equation} 
整理的
\begin{equation}
  T_{i} = 
  \frac{C_{i}}{A_{i}-B_{i}P_{i-1}}T_{i+1} +
  \frac{D_{i}+B_{i}Q_{i-1}}{A_{i}-B_{i}P_{i-1}}
\end{equation}
将上式与式\eqref{EqLA_xiaoyuan_1}对比,可得
\begin{equation}
\begin{aligned}
  P_{i} 
  &=
\frac{C_{i}}{A_{i}-B_{i}P_{i-1}} \\
Q_{i}
  &=
  \frac{D_{i}+B_{i}Q_{i-1}}{A_{i}-B_{i}P_{i-1}}
\end{aligned}
\end{equation}
这是一个递推关系式,需要确定$P_{1}$和$Q_{1}$。

当$i=1$时,式\eqref{EqLA_TDMA_array}为
\begin{equation}
  A_{1}T_{1} = B_{1}T_{0} + C_{1}T_{2} + D_{1}
\end{equation}
其中,$B_{1}=0$。对比式\eqref{EqLA_xiaoyuan_1},可得
\begin{equation}
  P_{1} = \frac{C_{1}}{A_{1}},\quad
  Q_{1} = \frac{D_{1}}{A_{1}}
\end{equation}
在$P_{i}$和$Q_{i}$序列的另一端,消元进入最后一行时,由式\eqref{EqLA_xiaoyuan_1}
得
\begin{equation}
  T_{N} = P_{N}T_{N+1}+Q_{N}
\end{equation}
且$P_{N}=0$,得
\begin{equation}
  T_{N} = Q_{N}
\end{equation}
到此消元过程结束,然后按照式\eqref{EqLA_xiaoyuan_2}进行回代。

下面给出TDMA方法的计算步骤:
\begin{enumerate}
  \item 计算系数$P_{i}$和$Q_{i}$
    \begin{equation*}
      P_{1} = \frac{C_{1}}{A_{1}}, \quad Q_{1} = \frac{D_{1}}{A_{1}}
    \end{equation*}
  \item 对$i=1,2,\cdots,N$用递推关系式求系数$P_{i}$和$Q_{i}$
    \begin{equation*}
      P_{i} 
      =
      \frac{C_{i}}{A_{i}-B_{i}P_{i-1}}
      ,\quad
      Q_{i}
      =
      \frac{D_{i}+B_{i}Q_{i-1}}{A_{i}-B_{i}P_{i-1}}
    \end{equation*}
  \item 令$T_{N}=Q_{N}$
  \item 对$i=N-1,N-2,\cdots,2,1$应用回代得$T_{N-1},T_{N-2},\cdots,T_{2},T_{1}$
    \begin{equation*}
      T_{i-1}=P_{i-1}T_{i} + Q_{i-1}
    \end{equation*}
\end{enumerate}

\section{代数方程组迭代求解方法}
\subsection{雅可比迭代法}
\subsection{高斯赛德尔迭代}
\subsection{松弛迭代}

%\section{多重网格法}
